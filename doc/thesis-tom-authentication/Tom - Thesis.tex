\documentclass[paper=a4,fontsize=11pt,parskip=half]{scrartcl}


\usepackage[utf8]{inputenc}
\usepackage[T1]{fontenc}
\usepackage{graphicx}
\usepackage{titlesec}
\usepackage{listings}
\usepackage{setspace}
\usepackage{xcolor}
\usepackage{color}
\usepackage{pgf}
\usepackage{enumitem}
\usepackage[ngerman]{babel}
\usepackage[autostyle=true,german=quotes]{csquotes}
\usepackage[acronym,hyperfirst=false]{glossaries}
\usepackage[colorlinks]{hyperref}
\usepackage{acronym}
\usepackage[colorinlistoftodos,prependcaption]{todonotes}
\usepackage{soul}
\usepackage{subcaption}
\usepackage{fancyhdr}
\usepackage{hyperref}
\usepackage[backend=biber]{biblatex}

\hypersetup{
	colorlinks=true,% make the links colored
}

% Besseres highlighting von Worten
% https://tex.stackexchange.com/questions/343458/
\makeatletter
\if@todonotes@disabled
\newcommand{\hlnote}[2]{#1}
\else
\newcommand{\hlnote}[2]{\todo{#2}\texthl{#1}}
\fi
\makeatother

\titlespacing*{\section} {0pt}{0ex}{0ex}
\titlespacing*{\subsection} {0pt}{0ex}{0ex}
\titlespacing*{\subsubsection} {0pt}{0ex}{0ex}

\lstdefinestyle{CodeView} {
	columns=flexible,
	numbers=left,
	basicstyle=\footnotesize,
	tabsize=4,
	frame=single,
	backgroundcolor=\color{white},
	keywordstyle=\color{blue},
}

\colorlet{punct}{red!60!black}
\definecolor{background}{HTML}{EEEEEE}
\definecolor{delim}{RGB}{20,105,176}
\colorlet{numb}{magenta!60!black}

\lstdefinelanguage{JavaScript}{
	keywords={typeof, new, true, false, catch, function, return, null, catch, switch, var, if, in, while, do, else, case, break},
	keywordstyle=\color{blue}\bfseries,
	ndkeywords={class, export, boolean, throw, implements, import, this},
	ndkeywordstyle=\color{darkgray}\bfseries,
	identifierstyle=\color{black},
	sensitive=false,
	comment=[l]{//},
	morecomment=[s]{/*}{*/},
	commentstyle=\color{purple}\ttfamily,
	stringstyle=\color{red}\ttfamily,
	morestring=[b]',
	morestring=[b]"
}


\lstdefinelanguage{JSON}{
	basicstyle=\normalfont\ttfamily,
	numbers=left,
	numberstyle=\scriptsize,
	stepnumber=1,
	numbersep=8pt,
	tabsize=4,
	showstringspaces=false,
	breaklines=true,
	frame=lines,
	backgroundcolor=\color{background},
	literate=
	*{0}{{{\color{numb}0}}}{1}
	{1}{{{\color{numb}1}}}{1}
	{2}{{{\color{numb}2}}}{1}
	{3}{{{\color{numb}3}}}{1}
	{4}{{{\color{numb}4}}}{1}
	{5}{{{\color{numb}5}}}{1}
	{6}{{{\color{numb}6}}}{1}
	{7}{{{\color{numb}7}}}{1}
	{8}{{{\color{numb}8}}}{1}
	{9}{{{\color{numb}9}}}{1}
	{:}{{{\color{punct}{:}}}}{1}
	{,}{{{\color{punct}{,}}}}{1}
	{\{}{{{\color{delim}{\{}}}}{1}
	{\}}{{{\color{delim}{\}}}}}{1}
	{[}{{{\color{delim}{[}}}}{1}
	{]}{{{\color{delim}{]}}}}{1},
}

\fancyhf{}
\chead{\fancyplain{}{\nouppercase{\rightmark}}} 
\cfoot{\pagemark}

\onehalfspacing

\title{Abschlussarbeit}
\author{Tom Hilge}

\makeglossaries

\newcommand{\JS}{JavaScript}
\newcommand{\TS}{TypeScript}

\newacronym{JWT}{JWT}{JSON Web Token}
\newacronym{URI}{URI}{Uniform Resource Identifier}
\newacronym{HTTP}{HTTP}{Hypertext Transfer Protocol}
\newacronym{CSS}{CSS}{Cascading Style Sheets}
\newacronym{JSON}{JSON}{JavaScript Object Notation}
\newacronym{MVC}{MVC}{Model View Controller}
\newacronym{HTML}{HTML}{Hypertext Markup Language}
\newacronym{DOM}{DOM}{Document Object Model}
\newacronym{SPA}{SPA}{Single Page Application}
\newacronym{CLI}{CLI}{Command Line Interface}
\newacronym{API}{API}{Application Programming Interface}
\newacronym{RFC}{RFC}{Request for Comments}
\newacronym{MIME}{MIME}{Internet Media Type}
\newacronym{HMAC}{HMAC}{Keyed-Hash Message Authentication Code}
\newacronym{SHA256}{SHA256}{Secure Hash Algorithm 256 Bit}
\newacronym{oAuth2}{oAuth2}{Open Authorization 2.0}
\newacronym{UUID}{UUID}{Universally Unique Identifier}
\newacronym{STI}{STI}{Single Table Inheritance}
\newacronym{XSS}{XSS}{Cross-Site-Scripting}
\newacronym{CSRF}{CSRF}{Cross-Site-Request-Forgery}
\newacronym{HTTPS}{HTTPS}{Hypertext Transfer Protocol Secure}
\newacronym{URL}{URL}{Uniform Resource Locator}
\newacronym{SQL}{Structured Query Language}{Structured Query Language}

\addbibresource{quellen.bib} 

\begin{document}
	\hypersetup{linkcolor=black, urlcolor=black}
	\clearpage
	\thispagestyle{empty}

\begin{figure}[h]
	\hfill\includegraphics[scale=0.5]{graphics/ptl-logo.png}
\end{figure}
\vspace*{15ex}

\begin{center}
	\textbf{{\LARGE Abschlussarbeit 2019}}\\
	\vspace*{2ex}
	{\large Implementierung von Authentifizierung und Autorisierung in eine bereits vorhandene Webanwendung}\\
\end{center}

\vspace*{15ex}
\begin{tabbing}
	\hspace{8em} \= \hspace{14em} \= \hspace{8em} \= \kill
	Eingereicht am: \>02.09.2019\\\\
	
	Autor:\> Tom Hilge\\
	Matr.-Nr.:\> 101735\\
	E-Mail:\> \href{mailto:tom.hilge@gmail.com}{tom.hilge@gmail.com}\\\\
	
	Betreuer:\> M.Sc Marcus Riemer\\
	E-Mail:\> \href{mailto:mri@fh-wedel.de}{mri@fh-wedel.de}
\end{tabbing}
	\clearpage
   	\tableofcontents
  	\clearpage
 	\pagestyle{fancy}
	\section{Einleitung}
\label{sec:introduction}

SQLino ist bisher auf die Erstellung von Seiten mit Texten und Tabellen begrenzt. Praktisch relevante Webseiten verwenden allerdings fast ausnahmslos auch grafische Gestaltungselemente. Darunter fallen zum Beispiel Fotos von im Text beschriebenen Sachverhalten, bei denen auf Autos gefallene, entwurzelte Bäumes, Gebirgspanoramen, Strände, ... zu sehen sind. Gemäß dem Motto \enquote{ein Bild sagt mehr als 1000 Worte} profitiert hier der Leser von den Bildern. Darüber hinaus gibt es auch Sachverhalte die sich kurz und prägnant durch Diagramme, nicht aber durch Texte erfassen lassen. Dementsprechend ist die Ergänzung um grafische Inhalte für SQLino sowohl praktisch relevant als auch (vermutlich) für die Schülerinnen und Schüler \footnote{Zugunsten der leichteren Lesbarkeit wird in dieser Thesis auf die Ausdifferenzierung des Geschlechts von Personen im Folgenden verzichtet, da es für den Sachverhalt irrelevant ist und den Lesefluss stört.  Wenn in dieser Thesis Personen genannt werden, steht das grammatikalische Geschlecht in keinem Zusammenhang zum tatsächliche Geschlecht der jeweilig genannten Personen und es sind immer alle Personen gemeint, auf die die restliche Beeschreibung zutrifft, unabhängig des Geschlechts.} motivierend.

\todo[inline]{Beispielbilder welche Szenarien uns für die Bildnutzung vorschweben.}

Die Verwendung von Bildern auf Webseiten bereitet jedoch auch immer wieder Probleme. Neben deutlich zu hohen Dateigrößen, welche Ladezeiten in die Länge ziehen und bei
Mobilgeräten das in Deutschland im Vergleich zu anderen Ländern um
Größenordnungen geringere Datenvolumen unötig schnell aufbrauchen - von der
Unbenutzbarkeit solcher Webseiten in der ``Edge-Wüste'' der ländlichen
Mobilfunknetzte und Breitbandanschlüsse, die ihrem Namen nicht gerecht werden
mal ganz abgesehen - \todo{Technische Gründe}


ist ein weiteres Problem die mangelnde auseinandersetzung
mit dem Urheberrecht und Nutzungslizenzen\todo{}. Sogenannte Soziale Netzwerke und
Diskussionsplatformen verleiten den Nutzer Bilder einzustellen und dem Anbieter der
Platform laut AGB diverse Rechte an diesen Bildern zu übertragen. Da der
durchschnittliche Nutzer weder die AGB gelesen hat, noch sich dessen bewusst
ist, dass die Befugnis die Rechte an den Bildern Dritter an den
Platformbetreiber weiterzugeben bei besagten Dritten liegt und nicht bei der
eigenen Person.\todo{Rechtliche Gründe}

\todo[inline]{Anspruch: Technisch und rechtlich "standarmäßig" korrekt / "sicher".}

Ein weiteres Problem ist die Verwaltung dieser Bilder. Ein Bild wird bei
irgendeinem gerade angesagtem Imagehoster abgelegt, eingebunden und vergessen.
Andere Bilder werden anderswo abgelegt, eingebunden und vergessen. Einige der
Imagehoster liefern nach tausend Aufrufen im Monat das Bild nicht mehr aus ohne
sich dafür bezahlen zu lassen, andere ändern nach einiger Zeit ihr Konzept und
lassen einbetten ihrer Bilder auf fremden Seiten nicht mehr zu und zeigen die
Bilder nur noch auf ihrer eigenen Werbefinanzierten Webseite an und wiederum
andere haben in der zwischenzeit Insolvenz angemeldet und alle dort
gespeicherten Bilder sind nicht mehr verfügbar.\todo{Also selber machen.}

Diese Thesis beschreibt den Prototypen einer Bildverwaltung für SQLino, die auf
diese Probleme eingeht, indem verwendete Bilder an einem Ort gesammelt und
verwaltet werden, den Bildern Metadaten wie Urheber und Lizenz zugeordnet
werden und dem Betrachter der Webseite dargestellt werden und Bilder beim
ausliefern auf die für das Endgerät geeignete Größe herunterskaliert werden.

Ein weiteres bearbeites Problem ist das Verwalten von Programmabhängigkeiten, die beim Entwickeln und Testen der Software sowie deren Produktivbetrieb zu berücksichtigen sind. \todo{Wichtig hier: Einstieg für neue Mitarbeiter, Testen mit Cloudservice}

%%% Local Variables:
%%% mode: latex
%%% TeX-master: "thesis"
%%% End:

	\clearpage
	\section{Technologien}
\label{sec:technology}
Im Verlauf dieser Sektion werden die Technologien und deren Verwendungszweck
kurz erläutert.

\subsection{Blattwerkzeug}
\label{sec:blattwerkzeug}

\begin{figure}
	\includegraphics[scale=0.4]{graphics/blattwerkzeug.png}
\end{figure}


Blattwerkzeug ist ein quelloffenes Projekt, dass Informatik-Interessierten das Programmieren von \gls{HTML} Grundgerüsten und SQL Statements per \enquote{drag and drop} näher bringen kann. Dabei versteckt Blattwerkzeug die Syntax nicht vor dem Nutzer, sondern gibt ihm die Möglichkeit diesen gleich mit ein zu sehen. Dennoch ist es dem Nutzer einfach gemacht, mit visuellen Elementen teile der Informatik kennen zu lernen.

Dabei hat es sich Blattwerkzeug vor allem als Aufgabe gemacht an Schulen aufzutreten. Mit Blattwerkzeug wird Lehrern ein Werkzeug in die Hand gelegt, mit dem einfacher und informativer Informatik Unterricht gestaltet werden kann. Somit kann der veraltete und doch sehr Office-lastige Informatik Unterricht komplett erneuert und interessanter gestaltet werden\todo{Zuviel: BlattWerkzeug ist ein Zusatz, keine Ersetzung}.

\subsection{Passwort Hashing}
\label{sec:password_hashing}

Sobald eine Software mit Nutzerdaten geführt wird, ergibt sich das Problem des Speicherns der Passwörter jeweiliger Nutzer.
Denn sollten die Daten der Nutzer im Klartext in der Datenbank gespeichert werden und ein Angreifer erlangt Zugriff auf die Datenbank, so ist es für ihn ein leichtes weitere Konten der Nutzer zu infiltrieren. Der Grund dafür sind die anwendungsübergreifenden, vom Nutzer größtenteils identischen, Passwörter.

\todo[inline]{Nicht nur Super-GAU für eigene Seite, sondern auch für Nutzer auf anderen Seiten. Konkretes Beispiel ergänzen (Liesschen Müller ist mit ihrer EMail
  \enquote{lieeschen@müller.de} und dem Passwort \enquote{Milch} bei \enquote{Milchkanne.de} registriert, ...}

An diesem Punkt kommt das Hashen von Passwörtern zum Einsatz. Passwort Hashing soll dem Nutzer Sicherheit gewährleisten und es einem Angreifer nicht möglich machen mit erlangten Daten weitere Konten der Nutzer zu infiltrieren. Dabei wird aus einem Passwort ein Hash generiert\todo{.}, dieser Hash macht es einem unmöglich, das Passwort wiederherzustellen. Jedoch ergibt sich bei gleicher Eingabe, der gleiche Hash. Um ein gehashtes Passwort zu erhalten, muss ein Hashing Algorithmus auf das jeweilige Klartext Passwort angewendet werden.

Mittlerweile gibt es verschiedene Hash-Funktionen\todo{Satz führt nirgendwo hin}. Rainbowtables in denen Hashes mit dazugehörigem Klartext Passwort (Abbildung \ref{fig:unhashed_table}) stehen sorgen jedoch dafür, dass manche dieser Hash-Funktionen als nicht mehr riches\todo{?} gelten. Dies hat zur Folge, dass eine Sicherheitslücke ensteht falls ein Angreifer die Nutzerdaten erlangt und die Passwörter mit einer solchen Hash-Funktion gehasht wurden. Die Möglichkeit besteht das gehashte Passwort mit einer Rainbowtable (Abbildung \ref{fig:rainbowtable}) abzugleichen und dabei das jeweilige Klartext Passwort zu erhalten. Weshalb MD5 (Abbildung \ref{fig:md5_hashed_table}) und SHA zwei der bekanntesten Hash-Funktionen, seit geraumer Zeit nicht mehr zum Passwort hashen verwendet werden \todo{Stattdessen?}.

\begin{figure}[h]
	\includegraphics[width=\textwidth]{graphics/unhashed_table.pdf}
	\caption{Tabelle mit ungehashten Passwörtern}
	\label{fig:unhashed_table}
\end{figure}

\begin{figure}[h]
	\includegraphics[width=\textwidth]{graphics/md5_table.pdf}
	\caption{Tabelle mit MD5 gehashten Passwörtern}
	\label{fig:md5_hashed_table}
\end{figure}


\begin{figure}[h]
	\includegraphics[width=\textwidth]{graphics/salted_table.pdf}
	\caption{Beispiel einer Rainbowtable.}
	\label{fig:rainbowtable}
\end{figure}

Aus diesem Grund werden sogenannte Salts~\ref{fig:salted-hash}, zufällig generierte Zeichenketten, an das Passwort angehängt und darauffolgend die Hashfunktion angewandt.

\begin{figure}[h]
	\includegraphics[width=\textwidth]{graphics/salted_hash.png}
	\caption{Hashfunktion auf Klartext und Salt angewandt.}
	\label{fig:salted-hash}
\end{figure}

\subsection{Sessions}
\label{sec: sessions}

Das \gls{HTTP} ist ein zustandsloses Protokoll, dass sich keine Informationen der jeweiligen Aufrufe zwischenspeichert. Dies ist unpraktisch\todo{In vielen Fällen ist das auch praktisch}, da so keine Informationen des Nutzers kurzzeitig gespeichert werden können. Ein Verwendungszweck wäre beispielsweise der Warenkorb, da dieser nur temporär vorhanden sein soll. Genau dieses Problem kann mit der Session gelöst werden\todo{Warum lösen Cookies das Problem nicht?}.

Die Session ist eine serverseitige Daten-Speichermöglichkeit. Dabei wird bei der Anfrage von einem Client an den Server ohne Session-ID eine Session und Session-ID erstellt. Diese Session-ID wird bei der Antwort des Servers mit an den Client ausgeliefert. Ab diesem Punkt wird bei jeder Anfrage vom Client an den Server die Session-ID mit gesendet. Dies kann über einen Cookie oder über die \gls{URI} erfolgen. Aufgrund dessen kann der Server dem Client Daten aus der jeweiligen Session zur Verfügung stellen.

IMAGE

\subsection{JSON Web Token}
\label{sec: jwt}
\enquote{\gls{JWT} sind auf \gls{JSON} basierende \gls{RFC} 7519 genormte Access-Token.} \gls{RFC} ist eine Sammlung aus Dokumenten in denen das Verhalten der Technologien des Internets beschrieben ist. Einige davon gehören zum Standard und werden somit in den meisten Fällen vorausgesetzt. In speziellen Fällen möchte beispielsweise ein Unternehmen eigene Protokolle verwenden die nicht zum Standard gehören.

Diese Tokens werden zur eindeutigen Identifizierung von Nutzern verwendet und können die Session ersetzen. Dabei ist es bei einem \gls{JWT} nicht vonnöten die Daten auf dem Server zu speichern. Dies hat zur Folge, dass die Pflege des Speichers an diesem Punkt entfällt. Jedoch haben \gls{JWT} einen großen Nachteil, sobald der Server einen \gls{JWT} ausgestellt hat, ist dieser bis zum Ablauf des Tokens gültig. Das heißt, sollte ein Server die Berechtigung eines Nutzers nach Ausstellung eines \gls{JWT} ändern, ist diese Änderung erst bei erneutem Erstellen eines \gls{JWT} gültig.

Ein \gls{JWT} besteht aus Header, Payload und Signatur. Dabei ist der Header und die Payload jeweils ein \gls{JSON} Objekt.

\subsubsection{Header}
\label{sec: jwt_header}

\begin{description}
	\leftskip=1em
	\item[typ] Der typ Claim beschreibt den \gls{MIME} des \gls{JWT}, dieser wiederum teilt dem Client oder Server mit um welche Art von Medium an Daten es sich handelt. Der Standardwert dieses Claims beläuft sich auf \enquote{JWT}, übersetzt \enquote{application/jwt}.
	\item[alg] Der alg Claim beschreibt die Verschlüsselungsmethode. Ein Beispiel ist \gls{HMAC} mit \gls{SHA256}, HS256 abgekürzt.
\end{description}

\begin{figure}[h]
	\centering
	\includegraphics[width=\textwidth]{graphics/jwt-header.png}
	\caption{Beispiel eines \gls{JWT} Headers }
	\label{fig:jwt-header}
\end{figure}

\subsubsection{Payload}
\label{sec: jwt-payload}

Die Payload beinhalteten Schlüssel-Wert Paare werden Claims genannt. Dabei handelt es sich um ein JSON Objekt, bei dem bestimmte Schlüssel des Objektes bereits reserviert sind. Diese nennen sich registrierte Claims. Außerdem gibt es öffentliche und private Claims. Hierbei wird zwischen öffentlichen und privaten differenziert.

\noindent
\textbf{Beispiel registrierter Claims}

\begin{description}
	\leftskip=1em
	\item[iss]
	Der iss Claim steht für den Austeller des Tokens, beispielsweise eine Domain.
	\item[exp] Der exp Claim kennzeichnet den \gls{JWT} mit einem Ablaufdatum.
\end{description}

\noindent
\textbf{Öffentliche Claims}

Öffentliche Claims sind zusätzlich zum Standard nutzbar und ihre Namen sollten Semantisch dem dazugehörigen Wert entsprechen. Außerdem sollten die Namen der Claims Netzwerkübergreifend verständlich sein.~\ref{fig:public-claim}

\noindent
\textbf{Private Claims}

Private Claims werden nur innerhalb eines Netzwerkes verwendet. Aus diesem Grund gibt es keine implizite Beschränkung in der Namensgebung.~\ref{fig:private-claim}

\begin{minipage}{\linewidth}
	\lstinputlisting[language=JSON, caption=Beispiel eines Öffentlichen Claims, captionpos=b]{snippets/public_claim.json}

	\lstinputlisting[language=JSON, caption=Beispiel eines Privaten Claims, captionpos=b]{snippets/private_claim.json}
\end{minipage}

\subsubsection{Signatur}
\label{sec: jwt_signature}

Um die Signatur zu erhalten muss die Payload und der Header Base64 kodiert werden. Außerdem müssen diese beiden kodierten Zeichenfolgen mit einem Punkt als Trennzeichen verknüpft werden. Darauffolgend wird eine Hashfunktion auf das jeweilige Ergebnis mit zusätzlich sicherer Zeichenfolge als Parameter angewandt. Da diese sichere Zeichenfolge, auch Private Key genannt, nur auf dem Server hinterlegt ist, ist es dem Client zwar möglich den \gls{JWT} zu verändern, ihn jedoch mit korrekter Signatur zu versehen nicht.

\subsubsection{Zusammengesetzes Token}
\label{sec: jwt_result}
Schlussendlich ergibt sich der \gls{JWT} aus kodiertem Header, kodierten Payload und der Signatur. Dabei steht der Header am Anfang (Abbildung \ref{fig:jwt-encoded}, Rot gekennzeichnet). Darauffolgend mit einem Punkt getrennt die Payload und zum Schluss die Signatur, ebenfalls mit einem Punkt getrennt.

\begin{figure}[h]
	\centering
	\includegraphics[scale=0.9]{graphics/jwt-encoded.png}
	\caption{Beispiel eines kodierten \gls{JWT} }
	\label{fig:jwt-encoded}
\end{figure}

\subsection{Ruby on Rails}
\label{sec: rails}
Ruby on Rails ein quelloffenes Webframework für die Programmiersprache Ruby. Das Webframework nutzt das \gls{MVC} Muster und stellt bereits ein sehr umfangreiches \gls{CLI} zur Verfügung. Mittels des generate Werkzeugs kann beispielsweise Model, View und Controller erstellt werden. Jeder dieser Komponenten wird automatisch in die erstellte Rails Anwendung eingebunden. Außerdem stellt Rails eine umfangreiche Test-Architektur und einen Service zum Versenden von Mails. Dabei kann der Inhalt der E-Mail im Textformat oder als \gls{HTML} versendet werden. Einer der wesentlichen Vorteile von Ruby on Rails ist jedoch die Datenbankanbindung. Hierbei bietet Rails einen Nachhaltigen und Rücksichtsvollen Umgang mit der Datenbank, beispielsweise Migrationen. Migrationen erlauben die Datenbank, ohne explizite SQL-Statements, zu verändern. Außerdem erleichtern Migrationen die Implementierung einer Datenbankstruktur auf einem anderen System.

\subsubsection{Routen}
\label{sec: routen}
Die Routen in Rails verweisen auf einen Controller und auf eine Funktion innerhalb des Controllers. Dabei wird die Route meistens mit der Anfragemethode eingeleitet, beispielsweise \enquote{get}. Routen können in sogenannte \enquote{scopes}~\ref{fig:routes-scope} unterteilt werden. Somit ist es nicht vonnöten bei einer Verschachtelten \gls{URI} redundant zu werden.~\ref{fig:routes-redundant}

\begin{figure}
	\includegraphics[width=\textwidth]{graphics/routes-redundant.png}
	\caption{Beispiel einiger redundanter Routen }
	\label{fig:routes-redundant}
\end{figure}

\begin{figure}
	\includegraphics[width=\textwidth]{graphics/routes-news.png}
	\caption{Beispiel einiger Routen mit scope }
	\label{fig:routes-scope}
\end{figure}

\subsubsection{Controller}
\label{sec: rails_controller}
Der Controller dient hierbei zur Kapselung von bestimmten Prozessen. Jede Route verweist in irgendeiner Weise auf eine Controller Funktion. In der der jeweiligen Controller Funktion wird dann meistens mit einem Model interagiert. Es wird beispielsweise eine Benutzerberechtigung abgefragt und individuell auf die Berechtigung reagiert. Um auf die jeweilige Berechtigung zu reagieren gibt es mehrere Möglichkeiten. Eine der Möglichkeiten wäre, direkt ein View Template auf dem Server zu rendern und an den Client auszuliefern. Eine andere Möglichkeit wäre ein \gls{JSON} Objekt zurück zu geben und darauf mit dem Client zu agieren.

\subsubsection{Model}
\label{sec: rails_model}
Das Model in Rails stellt jeweils eine Datenbanktabelle dar. Die Attribute des Models sind jeweilige Spalten der Datenbanktabelle. Jeweilige Datenbankeinträge die über das Model erstellt werden, können mittels Validatoren auf ihre Gültigkeit geprüft werden. Diese Validatoren werden innerhalb des Models festgelegt und auf ein Attribut des Models zugewiesen. Rails bietet dabei bereits verfügbare Validatoren, beispielsweise \enquote{presence: true} \hlnote{IMAGE-REF}{Bild Referenz}. Dieser Validator sorgt für das Vorhandensein eines Wertes ungleich \textit{nil}. Jedes Model kann zusätzliche Funktionen beinhalten, die direkt auf den jeweiligen Datenbankeintrag angewandt werden kann. Außerdem bietet Rails die Möglichkeit die Beziehungen zwischen Datenbanktabellen direkt in den Modellen festzulegen.

\subsubsection{View}
\label{sec: rails_view}
Die View stellt in Rails die Möglichkeit \gls{HTML} Template auf dem Server zu rendern. Dabei kann beim rendern das \gls{HTML} Template dynamisch verändert werden. Da diese Komponente während dieser Thesis keine Rolle gespielt hat, wird diese nicht weiter erläutert.

\subsection{Zusammenfassung}
\label{sec: rails_resuemee}
Schlussendlich wird über die Route auf den jeweiligen Controller zugegriffen. Dieser fragt in den meisten Fällen nach einem bestimmten Eintrag eines Models. Darauffolgend wird mit dem Ergebnis der Anfrage interagiert. Es werden Veränderungen oder abfragen bestimmter Daten getätigt. Danach wird ein Ergebnis dem Client ausgeliefert.

\todo[inline]{Grafik mit Routen, Controllern, Models}



\subsection{Angular}
\label{sec: angular}
Angular ist ein TypeScript basiertes Front-End Webframework, dass in vielen Fällen für \gls{SPA} verwendet wird. \gls{SPA}s laden ihren Inhalt lediglich in ein einziges \gls{HTML} Dokument. Der Inhalt dieses \gls{HTML} Dokumentes wird dynamisch, von beispielsweise einem Framework wie Angular, verändert. Der wesentliche Vorteil von Angular sind die klaren Entwurfsmuster. Jede Komponente in Angular hat im wesentlichen die gleiche Struktur. Dies hat zur Folge, dass Angular eine sehr gute Codekonsistenz bietet.


\subsubsection{Component}
\label{sec: ang-component}
Komponenten in Angular bieten die Möglichkeit \gls{HTML}, \gls{CSS} und TypeScript zu kapseln. Das bedeutet, dass jede Komponente unabhängig von einer anderen Komponente arbeiten kann.

\subsubsection{Services}
\label{sec: ang-service}
Zur Kommunikation mit einem Server und/oder zum Datenaustausch zwischen unterschiedlichen Komponenten wird meistens ein Service verwendet. Jedoch bei einem Datenaustausch zwischen Eltern- und Kind-Komponente ist es einfacher dies mittels der Kind-Komponente durchzuführen. Services werden beim laden der Module instanziiert und dem Konstruktor der Komponente als instanziiertes Objekt übergeben.

\subsubsection{Module}
\label{sec: ang-modul}
Zusätzlich bietet Angular außerdem die Möglichkeit eigene Module zu erstellen in denen dann beispielsweise Services und Komponenten zusätzlich abgekapselt werden können. Ein Vorteil von Angular gegenüber anderen JavaScript Frameworks, sind die bereits von Angular mitgelieferten Module, beispielsweise das Routing- oder das HTTP-Modul. Das Routing-Modul wird für jegliche Navigation auf der Anwendung genutzt. Das \gls{HTTP}-Modul hingegen bietet die Möglichkeit mittels jeglicher Anfragemethoden, mit dem Server zu kommunizieren.

\subsection{oAuth2}
\label{sec: oauth2}
\gls{oAuth2} ist ein offenes \gls{RFC} 6749 Protokoll welches verwendet wird um eine Authentifizierung einer Anwendung mittels Drittanbieter zu ermöglichen. Hierbei wird der Nutzer zuerst auf die jeweilige Seite des Drittanbieters weitergeleitet. Dort muss der Nutzer sich authentifizieren und den Zugriff auf die Daten seines Kontos bestätigen. Nachdem der Zugriff auf die Daten bestätigt wurde, erhält die jeweilige Anwendung von dem Drittanbieter einen Autorisierungs-Token. Dieser Autorisierungs-Token wird darauffolgend von der Anwendung genutzt um einen Zugriffs-Token von dem Drittanbieter zu erhalten. Dieser ermöglicht schlussendlich den Zugriff auf die spezifischen Nutzerdaten des Drittanbieters.~\ref{fig:oauth2}

In Blattwerkzeug wird genau dieser umfangreiche Vorgang von Omniauth übernommen. Aus diesem Grund wird oAuth2 in dieser Thesis nicht weiter erläutert.

\begin{figure}[h]
	\includegraphics[width=\textwidth]{graphics/oauth2.png}
	\caption{oAuth2 verfahren}
	\label{fig:oauth2}
\end{figure}

\subsection{Omniauth\todo{Später}}
\label{sec:omniauth}
Omniauth ist eine quelloffene Library für Ruby on Rails und ermöglicht einem, eine Anmeldung mittels unterschiedlicher Anbieter über \gls{oAuth2}. Bei der Anmeldung mittels oAuth2 werden bereits viele Funktionen von Omniauth selber übernommen. Sobald der Nutzer sich bei dem jeweiligen Anbieter angemeldet hat, wird die Antwort des jeweiligen Anbieters automatisch über die von Omniauth festgelegte Route verarbeitet. Jedoch muss vorher das spezifische Gem des Anbieters für Omniauth installiert werden. Hierbei stellt jedes Gem eine eigene Strategie für Omniauth bereit. Eine Strategie stellt die Möglichkeit bereit sich mit einem speziellen Provider zu authentifizieren.

Omniauth selber verfügt nur über die Developer Strategie, diese ermöglicht eine Anmeldung ohne spezifische Überprüfung der angegebenen Daten. Das hat zur Folge, dass diese Art von Anmeldung auf keinen Fall im Produktiv System vorhanden sein darf.

Den Vorteil den Omniauth bietet ist die Kapselung zwischen den spezifischen Providern und der Hauptfunktionalität von Omniauth. Zusätzlich das von Omniauth festgelegte Schemata. Dies hat zur Folge, dass der Server nur explizit mit den installierten Providern kommunizieren kann. Außerdem bietet Omniauth eine lange Liste an zu installierenden Providern.~\ref{fig:provider-list}

\todo[inline]{Beispiel gut, aber lieber als Liste statt als Grafik}
\begin{figure}[h]
        \includegraphics[width=\textwidth]{graphics/provider-list.png}
	\caption{Beispiele einiger zu installierender Provider }
	\label{fig:provider-list}
\end{figure}

\subsection{Pundit\todo{Später}}
\label{sec: pundit}
Pundit ist eine Ruby on Rails Library die ein Designpattern zur Autorisierung bietet. Bei diesem Pattern wird zu einem jeweiligen Controller eine Policy angelegt. Eine Policy ist hierbei nur eine Klasse. Dabei setzt sich der Name der Policy, aus dem Namen des Models und dem Schlüsselwort Policy als Suffix zusammen. Dem Konstruktor der Policy wird beispielsweise ein Nutzer und das jeweilige Objekt übergeben, welches auf den Zugriff geprüft werden soll. Innerhalb der Policy werden die jeweiligen Controllerfunktionsköpfe in denen eine Autorisierung stattfinden soll mit einem \enquote{?} als Suffix ergänzt und definiert. Diese Funktionen müssen zwingend einen Boolean als Rückgabewert haben um eine gültige Auswirkung als Policy zu haben. Sobald die aufgerufene Funktion der Policy fehlschlägt wird eine Exception geworfen. Diese Exception kann an jeweiliger Position beispielsweise im Controller abgefangen und verarbeitet werden.

Da es sich bei Policies um Klassen handelt, können diese auch instanziiert und jeweilige Funktionen dynamisch abgerufen werden. Dies hat zur Folge, dass explizit nach einer bestimmten Policy-Funktion gefragt werden kann, selbst wenn der Funktionsname nicht dem der aufgerufenen Policy-Funktion entspricht.

\subsection{Rolify\todo{Später}}
\label{sec: rolify}
Rolify ist eine Ruby on Rails Library zur Verwaltung von Rollen. Hierbei liefert Rolify bereits zwei Datenbanktabellen im Design der polymorphen Assoziation. Bei einer Eins-zu-viele-Assoziation hat beispielsweise ein Nutzer verschiedene Rollen, diese Rollen beinhalten verschiedene Fremdschlüssel aus verschiedenen Tabellen. Dabei ergibt sich das Problem, dass nicht mehr sicher gestellt werden kann aus welcher Tabelle der Fremdschlüssel stammt. Um dieses Problem zu lösen gibt es drei bewährte Methoden. In dieser Thesis gehen wir jedoch nur auf die von Rolify mitgelieferte Methode ein.

Bei dieser Methode handelt es sich um eine Kindtabelle \enquote{roles} und einer Elterntabelle \enquote{users\_roles}\todo{Grafisch}. Dabei stehen in der Roles-Tabelle die jeweiligen Informationen der Rolle und auf welche Ressource diese Rolle sich bezieht. Rolify unterscheidet hierbei zwischen globalen und Ressourcen spezifische Rollen. Eine globale Rolle beinhaltet keine Informationen einer Ressource und kann somit als beispielsweise allgemeine \enquote{user} Rolle dienen.

Die \enquote{users\_roles} beinhaltet wiederum die jeweilige \enquote{user\_id} und die zu dem Nutzer dazugehörigen Primärschlüssel einer Rolle. Dabei können verschiedene Nutzer dieselbe Rolle und ein Nutzer verschiedene Rollen haben.

Außerdem liefert Rolify bereits vordefinierte Funktionen mit denen es möglich ist, jeweilige Rollen eines Nutzers abzufragen, hinzuzufügen oder zu entfernen.


	\clearpage
	\section{Anforderungsanalyse}
\label{sec: analyze}
Das Ziel dieser Thesis ist, ein standardisiertes Registrierungsverfahren zu erstellen, zuzüglich einer Authentifikation über \gls{oAuth2} oder ein Passwort. Hinzu kommt eine Möglichkeit, den Nutzer über den Server zu autorisieren und ihm nach spezifischen Benutzerrollen unterschiedlichen Inhalt zu präsentieren. Dabei soll es dem Nutzer nicht gestattet sein, durch Manipulation seines Clients mit verfälschten Daten, Zugriff auf für ihn nicht zugreiffbare Daten zu erhalten. Um das Ziel dieser Thesis zu erreichen, muss man sich vorerst mit Ruby und dem Webframework Rails auseinandersetzen. Zudem ist es vonnöten, sich intensiver mit Angular zu beschäftigen, da dieses bisher nur oberflächlich behandelt wurde.

\subsection{Derzeitiges Projekt}
\label{sec: current_project}
Zum derzeitigen Zeitpunkt ist Blattwerkzeug eine Lernplattform, in der jeder Nutzer jedes Projekt bearbeiten kann. Dafür wurde serverseitig ein Benutzername und Passwort, in beiden Fällen \enquote{user}, hinterlegt. Zusätzlich ist es möglich, das Adminpanel ohne weitere Autorisierungsabfrage zu betätigen.

\subsubsection{Client}
Der clientseitige Teil von Blattwerkzeug basiert auf Angular, Angular-Material und Bootstrap, hierbei sind Angular-Material und Bootstrap Gestaltungsframeworks. Bootstrap, welches hauptsächlich auf \gls{CSS} und \gls{HTML} basiert und Angular-Material, dass explizit Module für Angular bereitstellt.

\subsubsection{Server}
Der serverseitige Teil baut auf Ruby on Rails und bietet ein \gls{API} zur Kommunikation mit dem Server. Die Daten werden mittels unterschiedlicher Anfragemethoden abgefragt, übermittelt und vom Server verarbeitet.

\subsection{Anforderungen}
\label{sec: requirement}
Im Verlauf dieser Sektion werden die Anforderungen, die diese Thesis erfüllen soll, detailliert erläutert.

\subsubsection{Unterschiedliche Anmeldemöglichkeiten}
Ein Benutzer sollte die Möglichkeit haben sich mit mehreren Konten zu verknüpfen. Das bedeutet, einem Benutzer würde die Möglichkeit gegeben sein, sich mit Passwort und zum Beispiel Google anzumelden. 

\subsubsection{Anmeldung mittels \gls{oAuth2}}
Eine Authentifizierung mittels oAuth2 soll über Google und GitHub möglich sein. Die von Google oder GitHub zurückgelieferten Daten sollen in der Datenbank abgespeichert werden. Darüber hinaus müsste bei dem Vorhandensein spezifischer Daten, wie beispielsweise die einer E-Mail, eine automatische Zuweisung spezieller Datenbankfelder des Nutzers erfolgen.

\subsubsection{Anmeldung mittels Passwortes}
Für eine Anmeldung mittels Passwortes muss zuerst eine Möglichkeit verfügbar sein ein Konto zu erstellen. Bei der Erstellung eines Kontos sollten vier Felder geben sein. Das Erste für den Benutzernamen, das Zweite für die E-Mail, das Dritte und Vierte für das Passwort und die Passwort Bestätigung. Sobald der Benutzer seine Daten erfolgreich abgeschickt hat, würde das vom Client als Klartext verschickte Passwort auf dem Server verschlüsselt werden. Nachdem der Benutzer sein Konto erstellt hat, muss eine Bestätigungsmail an die angegebene E-Mail gesendet werden. Diese Bestätigungsmail sollte einen Hyperlink beinhalten mit dem das vom Nutzer erstellte Konto bestätigt werden kann. Empfängt diese E-Mail den Nutzer nicht, muss die Möglichkeit bestehen, eine erneute Bestätigungsmail zu versenden. Erst nachdem das Konto bestätigt wurde, soll es dem Nutzer gestattet sein, dieses Konto zu verwenden. Falls ein Nutzer sein Passwort vergessen hat, muss es zusätzlich eine Funktion zum Passwort wiederherstellen geben.

\subsubsection{Authentifizierung nach Anmeldung}
Um vom Server als angemeldeter Benutzer authentifiziert zu werden, müssen die Benutzerdaten in einem \gls{JWT} gespeichert werden. Dieser \gls{JWT} wird bei der Anmeldung eines Benutzers an den Client übermittelt. Ab dem Zeitpunkt wird dieser \gls{JWT} bei jeder Anfrage an den Server mit gesendet. Sobald eine Anfrage den Server erreicht, muss zwangsläufig der \gls{JWT} auf seine Gültigkeit geprüft werden. Hat dieser \gls{JWT} eine nicht gültige Signatur oder ist bereits abgelaufen, darf keine weitere Aktion auf dem Server erfolgen.

\subsubsection{Sicherheit und Login}
Die Einstellungen im Bereich Sicherheit und Login sollen einem Benutzer erlauben sein bereits erstelltes Konto mit weiteren Konten zu verknüpfen. Hierbei muss der Benutzer eingeloggt sein und einen bestimmten Provider auf seiner Einstellungsseite auswählen können. Nachdem sich der Benutzer bei dem ausgewählten Provider authentifiziert hat, sollte dieses Konto dem Nutzer hinzugefügt werden. Ebenso müssen die Benutzereinstellungen eine Verwaltung der verknüpften Konten beinhalten. Das bedeutet der Benutzer kann zu jeder Zeit entscheiden, welche dieser verknüpften Konten beständig bleiben oder welche gelöscht werden.

Hat sich ein Benutzer mittels Passwortes auf Blattwerkzeug registriert, muss es für diesen Nutzer eine Möglichkeit geben sein Passwort zu ändern, selbst wenn dieser bereits sein Konto zusätzlich mit Google oder GitHub verknüpft hat. Besteht für den Benutzer bereits ein vorhandenes Konto mit einem Passwort, sollte das neu zu verknüpfende Konto automatisch das Passwort des bereits Vorhandenen annehmen. Falls das zu verknüpfende Konto das erste mit einem Passwort sein sollte, muss dafür in den Benutzereinstellungen eine extra Passworteingabe dargestellt werden, in die das zu verwendende Passwort eingegeben wird.

Da es in Blattwerkzeug bei der Benutzernamensgebung zum jetzigen Stand keine einmaligen Benutzernamen gibt, muss es dem Benutzer in den Benutzereinstellungen zusätzlich möglich sein, seinen Benutzernamen zu ändern.

\subsubsection{Rollen und Autorisierung}
Die Rollen müssen sich in globale und Ressourcen spezifische Rollen unterteilen. Dabei sind Globale-Rollen wie in Sektion ~\ref{sec: rolify} beschrieben, keiner spezifischen Ressource zugewiesen. Zwei Beispiele globaler Rollen wären \enquote{user} und \enquote{admin} oder \enquote{guest}, \enquote{user} und \enquote{admin}. Ressourcen spezifische Rollen beziehen sich zum Beispiel auf ein Projekt. Bei der Erstellung eines Projektes muss ein Benutzer eine Rolle oder eine Datenbank-Beziehung zu dem jeweiligen Projekt erhalten. Mit dieser Rolle ist es dann dem Benutzer möglich, sein Projekt zu bearbeiten oder zu löschen. Zusätzlich muss die Möglichkeit gegeben sein, einem anderen Benutzer eine Rolle zuzuweisen mit der das Bearbeiten eines von ihm nicht erstellten Projektes ermöglicht wird. Administratoren mit der \enquote{admin} Rolle sollten jedoch Zugriff auf jedes Projekt haben.

\todo[inline]{Auflistung von nötigen Rollen und Rechten}

Für eine Autorisierung mittels Rollen müssen bestimmte Regeln für die jeweiligen Controller Funktionen festgelegt werden. Diese Regeln werden mittels Pundit erstellt werden. Innerhalb dieser Regeln sollte die Überprüfung der Rollen des angemeldeten Nutzers stattfinden.


\subsubsection{Bedienelemente und Routen}
Die Bedienelemente, die ein Benutzer zu sehen hat, müssen jeweils von den Rollen abhängen. Dazu ist es vonnöten, bei jedem Bedienelement den Server nach der Berechtigung zu fragen oder jedoch Clientseitig Pundit nachzubauen. Besucht ein Benutzer eine Route mit nicht ausreichender Berechtigung, wird ihm der Zugriff verwehrt. Die Bedienelemente sollten benuzterfreundlich sein, da es sich bei Blattwerkzeug, wie in Sektion \ref{sec:blattwerkzeug} beschrieben, um ein Werkzeug zum Lernen bestimmter Bereiche der Informatik handelt.

\todo[inline]{Code-Beispiel für Verwendung als Angular-Komponente (Template)}

	\clearpage
	\section{Implementierung}
\label{sec:implementierung}
Nachdem die Anforderungsanalyse beendet, und die Ziele bestimmt wurden, wurden alle Funktionalitäten implementiert. Dabei wurde in Richtung Front-End zu Back-End gearbeitet. Diese Reihenfolge wird auch im folgenden Kapitel eingehalten. Zum einem um den Entwicklungsprozess wieder zu spiegeln, zum anderen um die Gedankengänge besser darzustellen, die sich durch die Entwicklungsreihenfolge entwickelt haben.

\subsection{Client - Darstellung}
\label{subsec04:client_viz}
In dem ersten Abschnitt wird die Entwicklung des Clients beschrieben. Dieser dient letzten Endes zu der Bedienung und Darstellung der Software. Neben der Funktionalität, muss zusätzlich auf Bedienbarkeit und Verständlichkeit der Software geachtet werden.

\subsubsection{Informationen vom Server}
\label{subsubsec04:info_server}

In der schon bereits vorhandenen Software gibt es eine erste Implementierung, die die Informationen vom Server an den Client sendet.
Der Client speichert diese Informationen in einem ``Interface'' (siehe: Listing~\ref{lst:desc_schema}). 

\lstinputlisting[
  language=javascript,
  caption=Interface einer Tabelle und Spalte,
  label=lst:desc_schema,
  float=h,
  numbers=left
]{snippets/schema.description.ts}

Eines der Grundprinzipien von Typescript, ist die Typprüfung. Interfaces erfüllen dabei die Rolle der Benennung der Typen und bestimmen Schnittstellen innerhalb des Projektes und zum Code außerhalb.~\cite{typescript_interfaces}

Damit ist festgeschrieben welche Eigenschaften eine Tabelle und die dazugehörigen Spalten besitzen müssen. Bis zu diesem Zeitpunkt wurden die Informationen der Datenbank, nur verwendet um diese, beim Erstellen von SQL-Statements, auszuwählen. Mit der dazu kommenden Funktionalität der Migration, werden Funktionen auf einzelne Tabellen oder Spalten angewendet. Diese Funktionen lassen sich nicht in einem Interface abspeichern. Für eine leichtere Anwendung und der Verhinderung globale Funktionen zu definieren, wurden zwei Klassen implementiert. In einer Tabellen und Spalten Klasse konnten die einzelnen Funktionen für die weitere Implementierung programmiert werden.


\subsubsection{Fehlende Foreign Keys}
\label{subsubsec04:miss_fk}

Bei der Analyse der Daten die vom Server versendet werden (siehe: Listing~\ref{lst:desc_schema}), wurde festgestellt, dass die Foreign Keys der einzelnen Tabellen nicht übermittelt wurden.
Für die akkurate Darstellung des Schemas sind die Foreign Keys der einzelnen Tabellen notwendig, um die einzelnen Relationen anzeigen zu können.
Dafür mussten die Informationen zuerst von der Datenbank erfragt und dann versendet werden (siehe dafür:~\ref{subsubsec04:fk_table_server}).
Da im Gegensatz zu Tabellen und Spalten keine direkten Veränderungen an Foreign Keys möglich sind, sondern nur das Hinzufügen und Entfernen dieser, müssen diese nicht in einer eigenen Klasse abgebildet werden.

\lstinputlisting[
  language=javascript,
  caption=Interface: Foreign Keys,
  label=lst:fk,
  float=h,
  numbers=left
]{snippets/foreign_key.description.ts}

\subsubsection{Anzeige des Schemas}
\label{subsubsec04:anz_schema}

Die Darstellung des Schemas ist der Einstiegspunkt der Komponente und soll neben der Darstellung auch die Bedienung und Weiterleitung zu den weiteren Funktionen liefern.

\begin{description}
\item[Angular 2] \hfill\\
Im ersten Schritt wurden die Daten, die vom Server zur Verfügung gestellt wurden, in simpler Art auf dem Bildschirm dargestellt. 
Dafür bietet Angular 2 ``Templates'' an. Templates definieren damit das Aussehen einer Komponente. Templates sehen dabei aus wie \texttt{HTML}-Dokumente, mit ein paar Unterschieden~\cite{angular2_flow}.

\begin{figure}[ht]
    \frame{\includegraphics[width=\textwidth]{images/kap-4-angular2_overview.png}}
    \centering
    \caption{Angular 2 Architektur}
    \label{pic:angular2_architecture}
\end{figure}

Bei diesen Unterschieden handelt es sich um Erweiterungen, die bestimmte Funktionalitäten zu dem \texttt{HTML} Standard hinzufügen. Die wichtigsten und in diesem Projekt am häufigsten angewandten Erweiterungen sind folgende:
\begin{itemize}
\item Direkter Zugriff auf Variablen einer Komponente (Property Binding)
\item Funktionen direkt an Events zu verknüpfen (Event Binding)
\item If-Anweisungen 
\item Schleifen
\end{itemize} 


\item[Bootstrap] \hfill\\
Bootstrap ist ein \texttt{CSS}-Framework, welches in dem Projekt bereits eingebunden ist. Damit lassen sich auf schnelle und einfache Art \texttt{HTML}-Dokumente gestalten. 
Eines der in Bootstrap vorhandenen Gestaltungsvorlagen sind die ``Cards''\footnote{\url{https://v4-alpha.getbootstrap.com/components/card/}}, die für die Gestaltung der einzelnen Tabellen benutzt wurde.

\begin{figure}[ht]
    \includegraphics[width=\textwidth]{images/kap-4-tableView.png}
    \centering
    \caption{Darstellung einer Tabelle mit Bootstrap}
    \label{pic:table_boot}
\end{figure}

\item[Font Awesome] \hfill\\
Um die Darstellung ein wenig angenehmer für die Benutzer zu machen, wurden auf Symbole zurückgegriffen, um einige Informationen in der Tabelle oder Buttons darzustellen. In der ursprünglichen Entwicklung der Software, wurde dafür der Icon-Font\footnote{\url{https://de.wikipedia.org/wiki/Font_(Informationstechnik)}} ``Font Awesome'' eingebunden. Dieser Font wurde unter anderem dann auch zur Visualisierung der Primärschlüssel verwendet.(siehe: Grafik~\ref{pic:table_boot}) 

\item[Graphviz] \hfill\\
Im bereits vorhandenen Server war eine Schnittstelle zu einem Graphviz-Service\footnote{\url{http://www.graphviz.org/}} integriert. Graphviz ist ein Programmpaket um Grafiken darzustellen. Dieses kann dazu verwendet werden, um eine Darstellung des Datenbankschemas zu erzeugen und anzuzeigen. Speziell ermöglicht Graphviz eine weitere Möglichkeit das ganze Schema, mit nur den nötigsten Informationen, auf einen Blick sehen zu können.
Ein Beispiel ist in der Grafik~\ref{pic:graphviz_schema} zu sehen.

\begin{figure}[ht]
    \frame{\includegraphics[width=\textwidth]{images/visual_schema.png}}
    \centering
    \caption{Graphviz Schema Darstellung}
    \label{pic:graphviz_schema}
\end{figure}

\item[Zusätzliche Darstellung der Foreign Keys] \hfill\\
Zusätzlich zu der Darstellung mit Graphviz wird, wenn der Benutzer mit dem Mauszeiger über einer Spalte ist, die Tabelle und Spalte eingeblentet, auf die die Spalte verweist. (siehe Grafik:~\ref{pic:table_fk_view})

\begin{figure}[ht]
    \frame{\includegraphics[width=\textwidth]{images/kap-4-table-fk.png}}
    \centering
    \caption{Darstellung eines Foreign Keys}
    \label{pic:table_fk_view}
\end{figure}

\end{description}

\subsubsection{Tabelleninhalte}
\label{subsubsec04:table_content}

In Blattwerkzeug bestand bereits die Möglichkeit SQL-Statements auszuführen. Es ist also möglich eine Abfrage der Art ``\texttt{Select * from \textsc{Tabellenname}}'' auszuführen, und sich damit alle Daten aus einer Tabelle anzeigen zu lassen. In Anbetracht darauf, dass Blatt-werkzeug eine Software zum Erlernen von Datenbanken und deren Umgang ist, soll es möglich sein die Daten einer Tabelle in einer übersichtlichen Ansicht anzeigen zulassen, sowie es in anderen ähnlichen Programmen~\ref{sec02:vergleichbare_arbeiten}, die dieses zur Verfügung stellen, üblich ist.
Die Daten kommen vom Server in ein 2-Dimensionales Array, eine Dimension für jede vorhandene Spalte und die zweite Dimension für die einzelnen Zeilen der Spalte.
In Abschnitt~\ref{subsubsec04:anfragen_darstellung} wird dann im Detail darauf eingegangen, wie die Daten im Server von der Datebbank abgefragt werden.

Ein Beispiel einer solcher Darstellung ist in der Grafik~\ref{pic:table_content} dargestellt.

\begin{figure}[ht]
    \frame{\includegraphics[width=\textwidth]{images/table_content.png}}
    \centering
    \caption{Darstellung der Tabelleninhalte}
    \label{pic:table_content}
\end{figure}

\begin{description}
\item[Verwendung Angular 2 \& Bootstrap] \hfill\\
In der Darstellung der Daten konnte das 2-Dimensionale Array direkt in dem Template der Komponente eingebunden werden. Im folgendem sieht man die Verwendung der Angular 2 Templates mit der zusätzlichen Verwendung von Bootstrap, die im Abschnitt~\ref{subsubsec04:anz_schema} erwähnt wurden.
Dabei ist in Zeile 1 die \texttt{class}-Eigenschaft auf ``table table-striped'' gesetzt um die von Bootstrap zur Verfügung gestellte Tabellendarstellung zu verwenden. Dabei sorgt die zweite \texttt{class}-Eigenschaft, dafür dass die Tabellenzeilen, gestreift angezeigt werden, zur besseren Übersicht.
In den Zeilen 5,13,14 sind von Angular 2 Schleifen zu sehen und in Zeile 6 und 15, dass mit ``Property Binding'' direkt auf eine Variable aus der ``Component'' zugegriffen wird.

\lstinputlisting[
  language=HTML,
  caption=Code zur Darstellung von Tabelleninhalte,
  label=lst:table_entry,
  float=h,
  numbers=left
]{snippets/table_content.ts}

\item[Pagination] \hfill\\
Es ist zu berücksichtigen, dass einige Tabellen gefüllt sein können mit einer großen Datenmenge. Damit ein endlos langes Scrollen verhindert wird und den Datentransfer zwischen Datenbank-Server und Server-Client zu minimieren, wurde eine ``Pagination'' eingebaut. Damit kann die Anzahl an gleichzeitig angezeigten Daten gewählt werden, und mit den Pfeil-Buttons, kann zwischen den Pages~\footnote{Seiten - hier Satz an Daten} geschaltet werden. 

\end{description}

\newpage

\subsection{Client - Editor}
\label{subsec04:client_edit}

Im folgendem Kapitel soll die Implementierung des Editors auf Clientsseite erklärt werden. Dieser und der korrespondierende serverseitige Abschnitt werden deutlich und ausführlich behandelt, da es der Kern dieser Ausarbeitung ist.
Die Migration eines Schemas einer Datenbank, ist im Allgemeinen das Verändern der Datenbank, mit der Absicht diese auf neue Gegebenheiten anzupassen oder zu erweitern. Zusätzlich ist im allgemeinen Sinne der Migration, auch die Möglichkeit gegeben, das Schema auf eine ältere Version zurückzuführen.
In Anbetracht der hier vorliegenden Software, wird nur die Möglichkeit geboten das Schema anzupassen, ohne einer Versionskontrolle.
Da die Thematik der Schemamigration ein sehr komplexes ist, mit möglichen ir­re­ver­si­blen Fehlern, wird hier eine ausführliche Erklärung gegeben, die neben der Art der Implementierung, auch die Funktionsweise und Bedienung beschreibt.

\subsubsection{Änderungen und mögliche Folgen}
\label{subsubsec04:editor_moegliche_aenderungen}
Kurz angesprochen in Kapitel~\ref{subsubsec03:zu_entwickeln} welche Änderungen alle möglich sein sollen, wird hier im Detail erklärt welche Änderungen implementiert wurden. Vom SQLite Standard sind nur die Tabellennamensänderung und das Hinzufügen einer Spalte direkt vorgesehen und stehen als Funktionen zur Verfügung. Dabei wird zu allen Änderungen darauf hingewiesen, welche Konsequenzen oder negative Eigenschaften entstehen könnten.

\begin{description}
\item[Tabellen erstellen/entfernen] \hfill\\
Das zusätzliche erstellen von Tabellen, wird zu keinen Fehlern führen, im Gegensatz zu dem entfernen von Tabellen. Dies kann zu inkonsistenten Schemata führen. Damit eine Tabelle entfernt werden kann, müssen zuerst, mit hilfe des Editors, alle Foreign Keys entfernt werden, die auf die zu löschende Tabelle verweisen. Eine mögliche Automatisierung dieses Vorganges wurde bewust nicht eingebaut, da eine Automatisierung zu Missgeschicken führen kann und in Anbetracht des Lernerfolges, sollen die Benutzer darauf hingewiesen werden, dass in Datenbanken dafür selbst gesorgt werden muss, dass Verweise auf eine Tabelle mit entfernt werden müssen. So wurde nur ein Hinweis eingebaut, der den Benutzer darauf hinweist, dass Spalten auf die zu löschende Tabelle verweisen. Normalerweise, kann das Löschen, mit Triggern erreicht werden, jedoch unterstützt Blattwerkzeug derzeit keine Trigger, sondern möchte das Verständnis und Wissen des Users für die Situation aufbauen. Zudem können Trigger nur mit dem nötigen Verständnis effektiv benutzt werden. 

\item[Tabellennamen ändern] \hfill\\
Jede Tabelle erhält beim erstellen einen einzigartigen Namen. SQLite besitzt eine Funktion zum verändern des Namens.
Durch die Veränderung eines Tabellennamens können mehrere Probleme auftreten:
\begin{itemize}
    \item Trigger die auf die Tabelle verweisen funktionieren nicht mehr
    \item Indizes die auf die Tabelle verweisen funktionieren nicht mehr
    \item Fremdschlüsselbeziehungen verweisen auf den alten Tabellennamen
\end{itemize}
In Blattwerkzeug sind Indizes und Trigger zurzeit nicht vorgesehen, und sind damit in diesem Projekt nicht weiter behandelt worden. Dies muss bei der weiteren Entwicklung der Software gegebenenfalls beachtet werden.
Fremdschlüsselbeziehungen sind notwendig und vorhanden. Es muss dafür gesorgt werden, dass beim Umbenennen einer Tabelle alle Verweise auf diese Tabelle mit verändert werden. SQLite bietet dafür eine Lösung: \\
Wenn bei der Verbindung mit der Datenbank die ``foreign\_key\_constraints''\footnote{\url{https://sqlite.org/foreignkeys.html\#fk_enable}} eingeschaltet sind, werden alle Verweise auf diese Tabelle beim Umbenennen mit angepasst.\cite{sqlite_doc_alter}

\item[Spalten löschen] \hfill\\
Eine Funktion die von SQLite nicht direkt unterstützt wird. Dabei ist zu beachten, ähnlich wie beim Entfernen von Tabellen, dass das Schema weiterhin konsistent bleibt.

\item[Spalten hinzufügen] \hfill\\
Eine Spalte in eine bereits vorhandene Tabelle einzufügen ist in SQLite bereits integriert.
Dabei gibt es einige Einschränkungen die man dabei beachten muss. In einer neu angelegten Spalte sind keine Daten vorhanden, damit sind alle Werte dieser Spalte ``\texttt{NULL}''.
Somit ist ein ``NOT NULL Constraint'' in einer neuen Spalte nicht möglich, außer man gibt dieser Spalte einen Standardwert. \texttt{NULL}-Werte im Zusammenhang mit Primärschlüsseln haben in SQLite eine Besonderheit, dies wird im nächsten Abschnitt beschrieben.

\item[Primärschlüssels setzen/entfernen] \hfill\\
Beim entfernen eines Primärschlüssels, führt dies nicht zu Fehlern. Es kann zu inkonsistenten Zuständen führen, beim einpflegen von Daten die die Einzigartigkeit der Werte in dieser Spalte verletzen.
Beim Hinzufügen eines Primärschlüssels, können Fehler auftreten, sollten die Werte der Spalte die Einzigartigkeit verletzen.
Ein weiteres Problem besteht in der besonderen Verhaltensweise von \texttt{NULL}-Werten in Primärschlüsseln bei SQLite.
SQLite unterstützt ``\texttt{NULL}'' als einen Wert einer Primärschlüsselspalte, wenn die Spalte nicht vom Typ Integer ist. Wobei der SQL Standard ein implizites ``NOT NULL Constraint'' vorsieht.\\
Das Unterstützen von \texttt{NULL}-Werten als Primärschlüssel, ist auf einen Bug zurückzuführen und aus ``backward compatibility''~\footnote{Kompatibilität mit älteren Versionen} Gründen bis heute nicht behoben worden.\cite{sqlite_doc_alter}

\item[Typen ändern] \hfill\\
Der Typ einer Spalte ist in SQLite, wie schon in Kapitel~\ref{subsubsec03:zu_entwickeln-Datentypen} angesprochen, nur ein Hinweis, aber keine Restriktion. Somit hat SQLite keine Funktion vorgesehen um den Typen einer Spalte zu verändern. Blattwerkzeug sieht eine statische Typisierung vor, somit wurde darauf geachtet, dass beim Ändern des Typen die dazugehörigen Werte übereinstimmen.

\item[Not Null Constraint verändern] \hfill\\
Sollte die jeweilige Spalte keine \texttt{NULL}-Werte bereits besitzen, ist beim Hinzufügen kein Fehler zu erwarten.
Beim entfernen werden keine Fehler auftreten.

\item[Standardwert setzen/verändern] \hfill\\
Standardwerte werden erst bei einem \texttt{INSERT}-Statement wirksam und werden daher nicht direkt zu Fehlern führen.

\item[Reihenfolge der Spalten verändern] \hfill\\
Die Reihenfolge der Spalten ist in SQL nicht von großer Bedeutung.
In der \texttt{SELECT}-Anweisung kann die Reihenfolge der Ausgabe unabhängig von der Spaltenreihenfolge bei der Erstellung, gewählt werden.
In Blattwerkzeug sollen die Tabellen als Tabellen in der Schemadarstellung angezeigt werden, dabei ist die Reihenfolge der Spalten die die bei Erstellung der Tabelle. Als Einsteigersoftware ist solch eine Funktion zur besseren Darstellung eine Erleichterung für den Benutzer. So tendieren Anfänger häufig dazu, die Primärschlüsselspalte als erste Spalte zu definieren, dies kann nach dem Verändern des Primärschlüssels, mit Veränderung der Reihenfolge erzielt werden.

\item[Foreign Keys setzen/entfernen] \hfill\\
Mit der Veränderung von Foreign Keys können nur inkonsistente Zustände im Schema erstellt werden.\\
\textit{Das hier entwickelte Backend unterstützt die Erstellung von zusammengesetzen Foreign Keys, doch nach Absprache mit dem Erfinder von Blattwerkzeug und fehlenden Ideen eines Designs diese zu erstellen, sind nur ``einfache'' Foreign Keys möglich.}
\label{fk_disclaimer}  
\end{description}


\subsubsection{Aufbau des Editors}
\label{subsubsec04:editor_aufbau}

Zuerst wird der generelle Aufbau des Editors beschrieben und welche Veränderungsmöglichkeiten bestehen. Dies wird zum leichteren Verständnis im weiteren Verlauf des Kapitels führen, wenn auf die einzelnen Elemente des Editors verweist wird. 

\begin{description}
\item[Layout] \hfill\\
Das drei Spalten Layout, welches in der Software bereits verwendet wird (siehe Abbildung:~\ref{pic:layout}) wird hier als Grundgerüst verwendet. Dabei wird der Editor Bereich horizontal getrennt. Der obere Bereich der die zu veränderte Tabelle anzeigt, wird genutzt um die Veränderungen an der Tabelle durchzuführen. Der untere Bereich wird die Tabelle mit ihren Daten, ähnlich der Darstellung der Tabelleninhalte~\ref{subsubsec04:table_content}, darstellen. \\
Der Toolbox Bereich wird ebenfalls horizontal getrennt. Der obere Bereich ist, wie in der schon vorhandenen Software, für die Bedienelemente zuständig, darunter das Drag \& Drop. Der untere Bereich stellt einen Stack da. Dies ist eine Liste die die einzelnen Veränderungen anzeigt, die an der Tabelle durchgeführt wurden.
Zur besseren Visualisierung des Layouts siehe folgende Grafik:

\begin{figure}[ht]
    \includegraphics[width=\textwidth]{images/kap-4-editor-layout.png}
    \centering
    \caption{Layout des Editors}
    \label{pic:layout_editor}
\end{figure}
\end{description}

%%%%%%%%%%%%%%%%%%%%%%%%%%%%%%%%%%%%%%%%%%%%%%%%%%%%%%
\subsubsection{Bedienung}
\label{subsubsec04:editor_bedienung}

\begin{description}
\item[Einzelne Veränderungen] \hfill\\
Wie im Abschnitt~\ref{subsubsec04:editor_moegliche_aenderungen}, am Beispiel der Erstellung einer neuen Spalte fest zu stellen ist, sind Kombinationen von Veränderungen Auslöser für Fehler. So ist das Hinzufügen von neuen Spalten kein Problem, ungleich zum Hinzufügen einer neuen Spalte die ein \texttt{NOT NULL}-Constraint besitzt. \\
Des Weiteren ist bei einer Kette von Veränderung nicht immer feststellbar wie die Ursprungstabelle zu der Zieltabelle entwickelt wurde, wenn man die Zwischentabellen nicht kennt.
Zur genaueren Erläuterung betrachte man folgende Grafik:~\ref{pic:table_chain_changes_example}. 

\begin{figure}[ht]
    \frame{\includegraphics[width=\textwidth]{images/kap-4-chain-changes-ex.pdf}}
        \centering
        \caption{Exemplarische Veränderung einer Tabelle}
        \label{pic:table_chain_changes_example}
\end{figure}

Ohne Betrachtung der Werte und der Zwischentabelle, könnte davon ausgegangen werden, dass nur die Spalte C gelöscht wurde.
Aus diesen Gründen ist es sinnvoll, jeden Schritt im einzelnen zu betrachten und abzuspeichern, anstelle nur der Zieltabelle.
Zusätzlich bietet dies die Möglichkeit dem Benutzer genau mitzuteilen welche Veränderung fehlgeschlagen ist.


\item[Vorschau] \hfill\\
Der Editor zeigt nur die Informationen der Tabelle an. Zum leichteren Verständnis was die einzelnen Veränderungen bewirken, auf Hinsicht der Tabelle und der Daten, wurde eine Vorschau eingebaut. Die Vorschau zeigt die Tabelle in der typischen tabellarischen Darstellung an, mit ein paar Beispieldaten aus der Tabelle.\\
Eine solche Darstellung wurde schon in der Darstellung der Tabelleninhalte~\ref{subsubsec04:table_content} entwickelt. Diese wurde in einer eigenen Angular 2 Component geschrieben. Dadurch wurde die Wiederverwendbarkeit von Components benutzt, und wird direkt in die Componente des Editors eingebunden. Damit die Veränderung der Tabelle auch gleichzeitig in beiden Components sichtbar sind, wird das Objekt der Tabelle in einem Service gespeichert auf das beide Components zugreifen. In der Grafik~\ref{pic:angular2_architecture} wurde die allgemeine Architektur dargestellt.

Damit die Vorschau noch übersichtlicher wird, werden die einzelnen Spalten farblich gekennzeichnet:
\begin{itemize}
    \item Rot - Spalte wurde gelöscht
    \item Grün - Spalte wurde neu hinzugefügt
    \item Gelb - Spalte wurde verändert
\end{itemize}

\item[Undo \& Redo] \hfill\\
Typescript unterstützt unter anderem das Objekt Orientierte Paradigma, damit kann man auf bereits bekannte Entwurfsmuster der Objekt Orientierten Sprachen zurückgreifen. 
Um die ``Undo \& Redo'' Funktion einzubauen wird das Verhaltensmuster ``Command''\footnote{Kommando oder auch Befehl} verwendet.(Mehr dazu:~\ref{subsubsec04:cmd_pattern})

\item[Stack] \hfill\\
Beim Verwenden von ``Undo \& Redo'' und das dabei eingesetzte Entwurfsmuster Kommando sind die einzelnen Objekte in einer Liste abgelegt(siehe Kapitel:~\ref{subsubsec04:cmd_pattern}).
Durch die Erweiterung der einzelnen Klassen mit einer Funktion die eine String-Repräsentation des Objekts liefern, ist das Erstellen eines Stacks (siehe Grafik:~\ref{pic:stack}) möglich. Dieser zeigt die einzelnen Schritte an, welche bereits gemacht wurden. Mit einem Pfeil wird dargestellt an welcher Stelle im Stack man sich zurzeit befindet. Eine zusätzlich eingebaute Funktion ist, die Möglichkeit durch einen Klick an eine bestimmte Stelle im Stack zu springen. Dabei wird die Undo oder Redo Funktion mehrmals hintereinander ausgeführt, bis die Zielstelle im Stack erreicht ist. Wenn man sich nicht am Ende des Stacks befindet, werden alle Elemente vom Stackpointer bis zum Ende gelöscht, bevor ein neues Element hinzugefügt wird. 

\begin{figure}[ht]
    \frame{\includegraphics[width=0.7\textwidth]{images/kap-4-stack.png}}
        \centering
        \caption{Stack Beispiel}
        \label{pic:stack}
\end{figure}
\end{description}

Der Stack wird zusätzlich verwendet, um bei aufgetretenen Fehler darzustellen, welcher Schritt fehlgeschlagen ist. In diesem Fall wird der Eintrag im Stack, rot markiert.


%%%%%%%%%%%%%%%%%%%%%%%%%%%%%%%%%%%%%%%%%%%%%%%%%%%%%%
\subsubsection{Command Pattern}
\label{subsubsec04:cmd_pattern}

Während der Entwicklung einer Software sollte eine Architektur gut durchdacht sein. In der Objekt Orientierten Programmierung, in der wir uns in diesem Projekt bewegen, treten einige Probleme immer wieder auf. Für solche Probleme wurden in früheren Projekten bereits Lösungen entwickelt. Um ein ``Neuerfinden'' von bereits vorhandenen Lösungen zu vermeiden, wurde eine Reihe von Entwurfsmustern erstellt. \\
Eine Zusammenfassung an Entwurfsmuster für Softwareentwickler, wurde von der so genannten ``Gang of Four'' zusammengestellt und unter dem Namen ``Design Patterns: Elements of Reusable Object-Oriented Software'' veröffentlicht.

\begin{description}
\item[Allgemein] \hfill\\
Für die Veränderungen werden einzelne Funktionen benötigt, die ``Undo \& Redo'' Funktionen, im Zusammenhang mit dem Stack, dass zu jeder Funktion eine Funktion besteht, die diese Rückgängig macht. Dabei müssen die gegebenenfalls veränderten Zustände gespeichert werden, um diese wieder herzustellen. \\
Für solch eine Gegebenheit ist in dem Werk das Command\footnote{Befehl} im Buch genannt.
Zitat: 
\begin{quote}
``support undo. The Command's Execute operation can store state for reversing
its effects in the command itself. The Command interface must have an added
Unexecute operation that reverses the effects of a previous call to Execute.
Executed commands are stored in a history list. Unlimited-level undo and
redo is achieved by traversing this list backwards and forwards calling
Unexecute and Execute, respectively.'' ~\cite{Gamma1994}
\end{quote}

Das Command Pattern erlaubt es bestimmte Funktionen in ein Objekt zu kapseln. Diese können dann verwendet werden ohne die genaue Funktionalität des einzelnen Objektes kennen zu müssen.
Eine häufige Anwendungsmöglichkeit ist in der Implementierung von GUI-Elementen. Indem das Command ähnlich einer ``callback-function'' dient. Zusätzlich können die einzelnen Befehle in einer Liste gespeichert werden, um diese zu einem späteren Zeitpunkt auszuführen.
Die Struktur dieses Entwurfsmusters wird in folgender Grafik visualisiert:~\ref{pic:struct_cmd_gof}.

\begin{figure}[ht]
    \frame{\includegraphics[width=\textwidth]{images/command_allg.pdf}}
        \centering
        \caption{Genreller Aufbau des Command}
        \label{pic:struct_cmd_gof}
\end{figure}

\item[Genereller Aufbau] \hfill\\
Anhand diesem Entwurfsmusters wurde für jede einzelne Veränderung an der Tabelle eine Klasse erstellt, die vom Typ Befehl ist. In jeder Klasse wurden die Funktionen zur Veränderung und deren Umkehrung implementiert. Zusätzlich speichert jede Klasse die nötigen Informationen und die alten Inhalte die verändert wurden. \\
In einer zusätzlichen Klasse, werden die einzelnen Befehle in einer Liste abgespeichert.
Für eine grafische Darstellung der Struktur siehe Grafik:~\ref{pic:struct_cmd_classes}
\begin{figure}[ht]
    \frame{\includegraphics[width=\textwidth]{images/cmd_imple.pdf}}
        \centering
        \caption{Strukturaufbau der Klassen}
        \label{pic:struct_cmd_classes}
\end{figure}

\item[Konvertierung Interfaces] \hfill\\
Jede Befehlsklasse besitzt eine Funktion zur Erstellung einer \texttt{JSON}-Repräsentation von sich selbst. Diese kann direkt an den Server versendet werden, um die Änderungen in der Datenbank durchzuführen.

\item[Textuelle Darstellung] \hfill\\
Zusätzlich kann jede einzelne Klasse eine textuelle Darstellung ausgeben, um diese auf dem Stack anzeigen zu können.
\end{description}

%%%%%%%%%%%%%%%%%%%%%%%%%%%%%%%%%%%%%%%%%%%%%%%%%%%%%%
\subsubsection{Kommunikation mit dem Server}
\label{subsubsec04:kommunikation_cs}
In diesem Abschnitt wird die Kommunikation beschrieben. Wie der Client Anfragen an den Server verschickt und welche Antwort erwartet wird.
Die Serverseitige Kommunikation wird im Kapitel:~\ref{subsubsec04:sinatra_client_reaction} beschrieben.

Für alle Anfragen braucht der Server einen Identifikator, der das Projekt definiert und den Namen der Datenbank. Die Anfragen werden per \texttt{HTTP}-Befehle an den Server geleitet.\footnote{Genaueres siehe: \url{https://marcusriemer.de/static/marcus-riemer-thesis-blattwerkzeug.pdf}}

\begin{description}
\item[Tabelleninhalte vom Server beantragen] \hfill\\
Für die Anfrage der Daten einer Tabelle wird der Tabellenname benötigt. Hinzu kommt, dass in der Ansicht der Daten, durch das Pagination (siehe Kapitel:~\ref{subsubsec04:anz_schema}), dem Server mitgeteilt werden muss, wie viele und ab welchen Tabelleneintrag die Daten erwartet werden.

Als Antwort ist ein 2-Dimensionales Array zu erwarten, welches die Spalten und all deren Zeilen enthält.

\item[Neue Tabellen erstellen] \hfill\\
Für das Erstellen neuer Tabellen muss in dem \texttt{Body} des \texttt{HTTP}-Befehls ein \texttt{JSON}-Objekt sich befinden, welches die zu erstellende Tabelle mit allen Informationen repräsentiert.

Als Antwort erhält der Client das neue komplette Schema zurückgeschickt. Damit kann der Client das Schema aktualisieren.
Dies erhöht den Datentransfer zwischen Server und Client. Eine andere Möglichkeit wäre, das Schema des Clients mit dem vorhandenen Objekt separat zu erweitern. Dies verringert den Datentransfer, kann aber bei Kommunikationsfehlern zu nicht synchronisierten Zuständen führen. Um dem gegen zu wirken, wird der Zustand des Servers als richtig angenommen und an den Client verschickt.

\item[Tabellen löschen] \hfill\\
In diesen Fall wird nur der Name der Tabelle verschickt. 

Die Antwort ist wieder das komplette aktualisierte Schema.

\item[Tabellen verändern] \hfill\\
In dem \texttt{Body} der Anfrage, ist eine Liste an einzelnen Befehlen~\ref{subsubsec04:cmd_pattern} die an einer Tabelle ausgeführt werden sollen.

Die Antwort, sollten keine Probleme auftreten, ist das aktualisierte Schema.

\item[Fehler beim Ausführen der Befehle] \hfill\\
Beim Erstellen, Löschen und Verändern wird die von der Datenbank mitgeteilte Fehlernachricht an den Client versendet.
Dies können Fehler sein, wie das Erstellen einer Tabelle mit einem Bereits vergebenen Namen oder auch Konsistenzverstöße des Schemas.

Beim Editieren einer Tabelle, wird zusätzlich der genaue Schritt mitgeteilt an welchem ein Fehler aufgetreten ist. Dies wird, wie im Abschnitt~\ref{subsubsec04:editor_bedienung} verwendet, um den Fehler im Stack darzustellen.
\end{description}


%%%%%%%%%%%%%%%%%%%%%%%%%%%%%%%%%%%%%%%%%%%%%%%%%%%%%%
\subsubsection{Problematiken}
\label{subsubsec04:problematik}

Während der Entwicklung des Clients kamen einige Probleme auf, die auf mehrere Weisen gelöst werden konnten. In diesem Abschnitt soll besprochen werden, welche Probleme aufkamen mit den möglichen Lösungen und einer Begründung warum die jeweilige Lösung gewählt wurde.

\begin{description}
\item[Multiple Tabellen simultan verändern] \hfill\\
Jede Veränderung einer Tabelle wird erst übernommen, wenn ein Speicher-Button gedrückt wird. Damit kam die Frage auf, wie damit umgegangen werden soll, wenn man während der Bearbeitung einer Tabelle, denn Editiervorgang verlässt und eine andere Tabelle bearbeitet.
Damit wäre es möglich zwei Tabellen simultan zu verändern. \\
Dies bietet eine Vielzahl an möglichen Problemen, die nur schwer vorhersehbar sind. Zusätzlich wäre eine sehr klare und detaillierte Kommunikation notwendig bei der Benutzung der Software. Sollte ein Benutzer eine Änderung an einer Tabelle gemacht haben und den Editorvorgang verlassen haben, muss zu jedem Zeitpunkt folgendes klar definiert sein. Hat der Benutzer den Editiervorgang verlassen, um eine andere Änderung durchzuführen, oder weil er den Vorgang abbrechen will. In Anbetracht der unvorhersehbaren Probleme, ist der Gewinn einer solchen Funktionalität recht gering. Zusätzlich sollte Einsteigern im Thema Datenbanken verständlich gemacht werden, dass die Migration von Datenbanken sich immer nur auf das Nötigste beziehen sollte, mit so wenig und kleinen Veränderungen wie möglich. Gerade bei den ersten Projekten, die keinen großen Umfang aufweisen, sollte eine Migration durch gute Vorausplanung verhindert werden. \\
Somit ist es nicht möglich mehrere Tabellen gleichzeitig zu verändern. Da aber während einer Veränderung, der Wunsch bestehen könnte, sich die anderen Tabellen anzusehen ohne das der Stack\footnote{die bereits durchgeführt, aber noch nicht gespeicherten Veränderungen} gelöscht wird, besteht die Möglichkeit den Editiermodus zu verlassen, dass Schema zu betrachten und den Editiervorgang fortzusetzen. Beim Betreten eines Editiermodus einer anderen Tabelle, wird der Benutzer darauf aufmerksam gemacht, dass das Betreten eines anderen Editiervorgangs, den bereits vorhandenen Stack löscht.

\item[Undo vs. selbst verändern] \hfill\\
Die Undo \& Redo Funktion (siehe Kapitel:~\ref{subsubsec04:editor_bedienung}) wurde eingebaut,als Erleichterung für den Nutzer, um eine falsche Aktion wieder Rückgängig zu machen. Der durch die Undo \& Redo Funktion entstandene Stack wird auch genutzt, um diesen zum Server zu schicken. \\
Neben der Benutzung von Undo ist es auch möglich die Änderung selbst Rückgängig zu machen. Als Beispiel wäre das Hinzufügen eines \texttt{NOT NULL} Constraints, und dieses wieder manuell zu entnehmen, im Gegensatz zur Benutzung der Undo-Funktion. 
Dies würde zu zwei verschiedenen Stacks führen, die auch anders auf Serverseite durchgeführt werden würden. (Siehe Grafik:~\ref{fig:self_vs_undo})

\begin{figure}[ht]
  \begin{subfigure}[b]{0.45\textwidth}
    \includegraphics[width=\textwidth]{images/kap-4-self_stack_change.png}
    \caption{Manuell Rückgängig}
    \label{fig:stack_self}
  \end{subfigure}\hfill
  \begin{subfigure}[b]{0.45\textwidth}
    \includegraphics[width=\textwidth]{images/kap-4-undo_stack_change.png}
    \caption{Nutzung von Undo}
    \label{fig:stack_undo}
  \end{subfigure}
  \caption{Vergleich manuell Rückgängig vs. Undo}
  \label{fig:self_vs_undo}
\end{figure}

Dadurch könnten Fehler auftreten, wenn Änderungen auf dem Server durchgeführt werden, von denen der Benutzer ausging sie wurden Rückgängig gemacht. Eine mögliche Lösung des Problems ist die ``Stack Optimierung'' die im nächsten Abschnitt besprochen wird und aus welchem Grund diese nicht von Vorteil ist. \\
Bei einigen Änderungen kann man die Absicht des Benutzers nicht abschätzen, und verhindert gegebenenfalls Änderungen die vom Benutzer gewollt sind. Mit der Anzeige des Stacks wird dem Benutzer angezeigt, welche Schritte durchgeführt werden.
Dies ist ein mögliches Thema, welches in weiterer Entwicklung der Software beachtet werden könnte. Mögliche Lösungen können sich ergeben nach Absprache mit Benutzern. 

\item[Stack Optimierung] \hfill\\
Stack Optimierung ist die Verallgemeinerung des Problems ``Undo vs. selbst verändern''. Es gibt viele Ketten an Veränderungen die zusammengefügt oder optimiert werden könnten. Einige Beispiele wären:
\begin{itemize}
    \item Verändern einer Spalte die danach gelöscht wird
    \item Manuelles rückgängig machen von Veränderungen
    \item Eine Spalte mehrmals Umbenennen
\end{itemize}

Die jeweiligen Lösungen dafür wären:
\begin{itemize}
    \item Veränderungen nicht durchführen, Spalte sofort löschen
    \item Änderung gar nicht erst durchführen
    \item Spalte direkt auf den letzten Namen ändern
\end{itemize}

Alle diese Änderungen könnten aber von dem Benutzer genau in dieser Weise gewollt sein. So könnte ein mehrfaches Umbenennen einer Spalte die Folge sein, wenn bei zwei Spalten die Namen vertauscht werden sollen und dafür der Spaltenname einer Spalte einen temporären Namen erhielt. \\
Sollte eine Optimierung des Stacks dynamisch schon während der Bearbeitung der Tabelle stattfinden, könnte dies zur Verwirrung des Benutzers führen, wenn einzelne Schritte aus dem Stack verschwinden und dann auch nicht mehr per Undo erreicht werden können.
Wenn die Optimierung nach dem Speichern angewendet wird, ist die Benachrichtigung während welchen Schrittes ein Fehler aufkam deutlich erschwert. \\
Die letzten beiden Punkte sind in Anbetracht des ersten Punktes, der Vorhersage was vom Benutzer wirklich gewollt ist, einfacher zu Lösen und weisen trotzdem einen hohen Aufwandswert auf, für den dafür gewonnen Nutzen. \\
Dieses Thema ist eine Optimierung die zu keiner besseren Benutzung der Software selbst führt und der Optimierungsgrad ist relativ gering im Vergleich zum Arbeitsaufwand, und wurde aus diesen Gründen in dieser Arbeit nicht weiter bearbeitet. In der weiteren Entwicklung der Software, kann dieses Thema ein Ansatzpunkt sein. 
\end{description}




\newpage
\subsection{Server}
\label{subsec04:server}
Im folgendem Kapitel, wird die Implementierung des Servers beschrieben. Der Aufbau von Blattwerkzeug besitzt eine starke Trennung zwischen Client und Server. Dies ermöglichte es, diese beiden Gebiete von einander getrennt zu implementieren. Die bereits vorhandene Testmöglichkeit des Server, erlaubt es alle Funktionalitäten zu testen, ohne die Verwendung eines Clients. \\
Der Server bietet eine direkte Schnittstelle für die Kommunikation mit der Datenbank, welche erlaubt die eigentlichen Aktionen auszuführen.

\subsubsection{Sinatra reagieren auf Anfragen}
\label{subsubsec04:sinatra_client_reaction}
In Sinatra werden die Anfragen durch ein \texttt{REST}-artigem System angesprochen.
Für die jeweiligen Anfragen wurden URLs bestimmt, mit denen die einzelnen Funktionen von Clients angesteuert werden konnten.
Die möglichen Anfragen an den Server sind:
\begin{itemize}
    \item Einen Satz von Daten einer Tabelle
    \item Die Anzahl von vorhandenen Daten in einer Tabelle
    \item Löschen einer Tabelle
    \item Anlegen einer neuen Tabelle
    \item Editieren einer Tabelle
\end{itemize}

Dabei wurde zur Vorbeugung von ``SQL-Injections''\footnote{\url{https://de.wikipedia.org/wiki/SQL-Injection}} darauf geachtet durch Prüfung, ob eine Tabelle zu der Informationen oder Änderungen angefragt wurden, auch wirklich besteht.

\subsubsection{Foreign Keys der Tabellen}
\label{subsubsec04:fk_table_server}
Das Schema wurde bereits an den Client versendet. Um die Foreign Keys dem Schema anzuhängen, wurde zuerst die Struktur der Foreign Keys bestimmt. \\
Die Informationen von Foreign Keys einer Tabelle kann man mit der Verwendung von \texttt{PRAGMAS} erhalten.
\texttt{PRAGMA}-Statement ist eine Erweiterung in SQLite, diese wird für die modifizierung der SQLite Operationen und zum erhalt von internen (nicht Tabellen) Informationen genutzt.\cite{sqlite_pragma} \\

Die Ausgabe des \texttt{PRAGMA} ``foreign\_key\_list''\footnote{\url{https://sqlite.org/pragma.html\#pragma_foreign_key_list}} enthält die nötigen Informationen. Sie bietet eine Liste einer Liste von zusammengesetzten Foreign Keys. Diese werden unter anderem durch drei Eigenschaften dargestellt:
\begin{itemize}
    \item den Spaltennamen der Tabelle die ein Foreign Keys Constraint besitzt
    \item auf welche Tabelle dieser Foreign Keys verweist
    \item zu welcher Spalte dieser Tabelle genau verwiesen wird
\end{itemize}

Foreign Keys auf Spaltenebene zu speichern erschien somit nicht sinnvoll. Wegen der Ausgabe des \texttt{PRAGMA} und das Foreign Keys aus mehreren Foreign Keys zusammengesetzt sein können, werden die Informationen zu Foreign Keys auf Tabellenebene gespeichert. \\
Diese werden mit Anlehnung an die Ausgabe des \texttt{PRAGMA} gespeichert. Es wird eine Liste einer Liste zusammengesetzter Foreign Keys gespeichert, die die drei Information direkt aus dem \texttt{PRAGMA} speichert. \\
Dies wird dann zusammen mit dem bereits versendeten Schema an den Client geliefert.   

\subsubsection{Typsicherung der Tabellenspalten}
\label{subsubsec04:typsicherung}
Die dynamische Typisierung von Spalten ist in SQLite eine Ausnahme verglichen mit anderen Datenbanken, wie schon im Kapitel:~\ref{subsubsec03:zu_entwickeln-Datentypen} erwähnt wurde. \\
Aus diesem Grund ist dafür gesorgt, dass die Typen in Blattwerkzeug sich ähnlicher zum Standard vom SQL verhalten. Dafür werden die Werte die in eine Tabelle eingetragen werden je nach Typ überprüft.
SQLite besitzt drei Möglichkeiten Werte nach einem bestimmten Muster zu prüfen. \\
Der \texttt{LIKE}-Operator bietet zwei Wildcards im zu vergleichenden Muster:
\begin{itemize}
    \item ``\_'' ein beliebiges Zeichen
    \item ``\%'' eine beliebige Anzahl an Zeichen
\end{itemize}
In Abfragen ist der \texttt{LIKE}-Operator häufig zu gebrauchen und nützlich.

Der \texttt{GLOB}-Operator ist sehr ähnlich zu dem \texttt{LIKE}-Operator, er ist zusätzlich Case-Sensitive und bietet Wildcards mit gleicher Funktionalität wie der \texttt{LIKE}-Operator und einer Möglichkeit ein Zeichen variabel aus einer Menge zu setzen. \\
Die Funktionalität vom \texttt{GLOB}-Operator und \texttt{LIKE}-Operator sind nicht ausreichend für die Überprüfung eines Wertes, ob dieser die Kriterien eines Typs erfüllt.

SQLite besitzt zusätzlich einen ``REGEXP''-Operator der prüfen kann, ob ein String\footnote{Zeichenkette} einem Regulären Ausdruck entspricht.
Der ``REGEXP''-Operator ist in SQLite nativ nicht implementiert, um die SQLite Datenbankbibliothek so klein wie möglich zu halten. Somit muss zur Laufzeit eine solche Funktion implementiert und hinzugefügt werden. \\
Die Schnittstelle von Ruby, in dem der Server geschrieben wurde, besitzt eine ``create\_\-function''-Funktion, um bei der Verbindung mit einer SQLite Datenbank, eine Funktion zu definieren, die dann verwendet werden kann. Damit wurde eine Funktion definiert für den ``REGEXP''-Operator. Der Quelltext dieser Funktion ist in Listing~\ref{lst:regexp_impl} dargestellt.

\lstinputlisting[
  language=ruby,
  caption=Implmentierung einer Funktion für den ``REGEXP''-Operator,
  label=lst:regexp_impl,
  float=h,
  numbers=left
]{snippets/regexp_impl.rb} 

Die Funktion überprüft im ersten Schritt, ob der übergebene Wert ein leerer String ist. Dies verhindert, dass \texttt{REGEXP} einen Fehler auslöst, da ein leerer String mit dem übergebenen Muster nicht übereinstimmt. Die Datenbank hat bereits mit dem \texttt{NOT NULL}-Constraint eine Überprüfung, ob ein Wert \texttt{NULL} sein darf oder nicht. Aus diesem Grund werden leere Strings nicht von der Funktion überprüft. \\
Der Rest der Funktion benutzt die Ruby Implementierung einer \texttt{REGEXP}-Funktion.\footnote{\url{https://ruby-doc.org/core-2.1.1/Regexp.html}}
Nachdem der ``REGEXP''-Operator verwendet werden kann, wird bei der Definition einer Spalte die Möglichkeit geboten, ein \texttt{Contraint}\footnote{\url{https://sqlite.org/lang_createtable.html}} zu definieren.
Für jeden der vorgesehenen Typen wurde ein \texttt{Contraint} erstellt, mit einer individuellen Fehlermeldung. Die vier Typen sind:
\begin{itemize}
    \item \texttt{TEXT} - Eine Menge an beliebigen Zeichen, dafür ist ein Constraint nicht nötig
    \item \texttt{BOOLEAN} - Der Wert darf nur von Wert von 1 oder 0 annehmen, dies kann mit Vergleichsoperatoren implementiert werden (siehe:~\ref{lst:regex_bool})
    \item \texttt{INTEGER} - Eine Vorzeichen gebundene ganze Zahl, dafür wird der ``REGEXP''-Operator verwendet, um sicher zustellen, dass nur Zahlen in dem Wert enthalten sind (siehe:~\ref{lst:regex_integer})
    \item \texttt{FLOAT} - Eine Vorzeichen gebundene reelle Zahl, dafür wird der ``REGEXP''-Operator verwendet, damit auch nur solche Zahlen angenommen werden (siehe:~\ref{lst:regex_float})
    \item \texttt{URL} - Ein zusätzlicher Typ, der in standard Datenbanken nicht vorhanden ist, soll aber für die junge Zielgruppe ein weiterer Zusatz sein, der die Möglichkeit bietet URLs in einer Datenbank zu speichern. (siehe:~\ref{lst:regex_url})
\end{itemize}

\lstinputlisting[
  language=sql,
  caption=Implementierung des \texttt{BOOLEAN}-Constraint,
  label=lst:regex_bool,
  float=p,
  numbers=left
]{snippets/regex_bool.sql}

\lstinputlisting[
  language=sql,
  caption=Implementierung des \texttt{INTEGER}-Constraint,
  label=lst:regex_integer,
  float=p,
  numbers=left
]{snippets/regex_integer.sql}

\lstinputlisting[
  language=sql,
  caption=Implementierung des \texttt{FLOAT}-Constraint,
  label=lst:regex_float,
  float=p,
  numbers=left
]{snippets/regex_float.sql}

\lstinputlisting[
  language=sql,
  caption=Implementierung des \texttt{URL}-Constraint,
  label=lst:regex_url,
  float=p,
  numbers=left
]{snippets/regex_url.sql}
  

\subsubsection{Anfragen - Darstellung}
\label{subsubsec04:anfragen_darstellung}

Wie schon im Kapitel~\ref{subsubsec04:anz_schema} wurde das Schema bereits an den Client versendet, mit der Erweiterung der Foreign Keys. (siehe Kapitel:~\ref{subsubsec04:fk_table_server}) \\
Für die Darstellung der einzelnen Tabelleninhalte, mit der Möglichkeit der Pagination~\ref{subsubsec04:table_content}, soll nur eine gewisse Anzahl an Daten aus einer Tabelle an den Client gesendet werden.

Die erste Möglichkeit nur eine bestimmte Anzahl an Daten an den Client zu schicken, wäre alle Daten von der Datenbank auf dem Server zu filtern und diese an den Client zu schicken. Dabei wäre das Problem, dass bei einer Tabelle mit vielen Inhalten viel Datentransfer statt finden würde. \\
Um den Datentransfer so niedrig wie möglich zu halten, werden die ``\texttt{LIMIT}''- und ``\texttt{OFFSET}''-Funktionen benutzt. 

Mit der ``\texttt{LIMIT}''-Funktion kann genau angegeben werden wie viele Daten aus der Tabelle wiedergegeben werden. Somit kann die Anzahl an Daten aus der Datenbank erfragt werden, die pro Page angezeigt werden sollen.

Mit der ``\texttt{OFFSET}''-Funktion, wird eine Anzahl an Daten übersprungen bei der Ausgabe. Somit werden bei einem \texttt{OFFSET 10} erst die Daten ab dem 11 Eintrag wiedergegeben. 

Zusätzlich zu den anzuzeigenden Daten auf dem Client, ist für die Pagination die Anzahl an vorhandenen Daten in der Tabelle nötig. Dafür wurde der ``\texttt{COUNT}''-Operator verwendet. Dieser wird in der Form ``\texttt{SELECT COUNT(*) FROM ...}'' verwendet. Dabei steht das ``\texttt{*}'' für alle Einträge, wobei das ``\texttt{COUNT}'' die Anzahl dieser zurückliefert.  

\subsubsection{Anfragen - Entfernen/Hinzufügen}
\label{subsubsec04:anfragen_editor_add_remove}

\begin{description}
\item[Tabellen entfernen] \hfill\\
Beim Entfernen einer Tabelle wurde keine weitere Funktionalität benötigt als die schon erwähnte Prüfung, ob die besagte Tabelle vorhanden ist. (siehe Kapitel:~\ref{subsubsec04:sinatra_client_reaction})\\
Die SQL-Datenbank meldet einen Fehler bei existenz eines Foreign Keys, der auf die zu löschende Tabelle, verweist. Der Fehler wird, an den Benutzer weitergeleitet. 

\item[Tabellen hinzufügen] \hfill\\
Bei der Anfrage zum Erstellen einer Tabelle erhält der Server im \texttt{Body} der Anfrage eine \texttt{JSON} Repräsentation der Tabelle. \\
Diese ist äquivalent zu der \texttt{JSON} Darstellung die bereits Verwendet wurde, um die einzelnen Tabellen vom Server an den Client zu senden. Damit wird die bereits implementierte Tabellen-Klasse verwendet, um das \texttt{JSON}-Objekt in ein Ruby Objekt zu konvertieren, für die weitere Verwendung. \\
Das Objekt einer Tabelle muss zu einem SQL-CREATE\_TABLE-Statement umgeformt werden, um dieses dann an die Datenbank zu senden. Die möglichen Fehler werden von der Datenbank geprüft und gegebenenfalls an den Client weiter gesendet. Einige typische Fehler, wie die Einzigartigkeit des Tabellennamens könnten bereits auf Clientsseite geprüft werden. Dies ist bei weiteren Optimierungen in Betracht zu ziehen. Die vorhandene Implementierung ermöglicht es solche Optimierung in der Zukunft einzubauen. Durch eine große Menge möglich auftretender Fehler und um eine Vermischung von Client- und Datenbankseitiger Fehlererkennung zu verhindern, wurde die ganze Fehlererkennung an die Datenbank ausgelagert.
\end{description}

\subsubsection{SQLite vs. Ruby-Migration}
\label{subsubsec04:sql_vs_ruby}
Für die Kommunikation mit der Datenbank gibt es die Möglichkeit der Verwendung von SQL oder der in Ruby integrierten ``Migration''.
Die Ruby Migration bietet die Möglichkeit Datenbanken zu verändern, dabei neue Tabellen erstellen, Tabellen zu verändern oder Tabellen zu löschen. Die Migration ist dabei Datenbanken unabhängig, somit wäre der Wechsel von SQLite zu einem anderen Datenbanksystem vereinfacht. In Hinblick auf die zur Verfügung gestellten Veränderungen (siehe Kapitel:~\ref{subsubsec04:editor_moegliche_aenderungen}), und der Typsicherung~\ref{subsubsec04:typsicherung} wurde sich für die Implementierung mit SQL-Statements entschieden. \\
Eine Veränderung des Tabellennamens und das Hinzufügen neuer Spalten oder kompletten Tabellen ist in SQLite und Ruby Migration möglich. Für eine Kategorisierung welche Veränderungen durch welche Schnittstellen nativ möglich sind siehe Tabelle:~\ref{tbl:sql_vs_ruby}

\begin{table}[]
\centering
\begin{tabular}{l|ll}
\hline
\multicolumn{1}{c}{\textbf{Veränderung}}           & \multicolumn{1}{c}{\textbf{SQLite}} & \multicolumn{1}{c}{\textbf{Ruby Migration}} \\ \hline
Reihenfolge vertauschen        &          &                         \\ \hline
Primary Key setzen/entfernen   &          &                         \\ \hline
Spaltentypen verändern         &          &                         \\ \hline
Check Constraint setzen        & \multicolumn{1}{c}{$\surd$}(im Create Table)         &                         \\ \hline
Not Null setzen/entfernen      &          &                         \\ \hline
Default Value setzen/entfernen &                 &  \multicolumn{1}{c}{$\surd$}                \\ \hline
Spalte entfernen               &                 &  \multicolumn{1}{c}{$\surd$}                \\ \hline
Foreign Keys setzen/entfernen  &                 &  \multicolumn{1}{c}{$\surd$}                \\ \hline
Tabellen Namen verändern       & \multicolumn{1}{c}{$\surd$}         &  \multicolumn{1}{c}{$\surd$}                \\ \hline
Spalte Hinzufügen              & \multicolumn{1}{c}{$\surd$}         &  \multicolumn{1}{c}{$\surd$}                \\ \hline
\end{tabular}
\caption{Native Funktionen in SQLite \& Ruby Migration}
\label{tbl:sql_vs_ruby}
\end{table}

Die Typsicherung ist ein wichtiger Aspekt in dieser Software (siehe:~\ref{subsubsec04:typsicherung}), dies ist von Ruby Migration nicht direkt unterstützt und müsste per zusätzliches SQL-Statement erreicht werden, ähnlich wie die Veränderung der Spaltenreihenfolge. \\
Es müssten somit für die Ruby Migration Sonderfälle integriert werden, um jede Veränderung abzubilden. \\
SQLite besitzt nativ nur die Möglichkeit Tabellennamen zu ändern und das Hinzufügen einer neuen Spalte. Jedoch bieten die SQLite Entwickler einen allgemeingültigen Algorithmus an, mit dem sich alle möglichen Veränderungen realisieren lassen:\cite{sqlite_doc_alter}

\begin{enumerate}\label{enum:sql_algo}
    \item Wenn Foreign Keys aktiviert sind, müssen diese deaktiviert werden
    \item Starten einer Transaktion
    \item Alle Indizes und Trigger speichern, die mit dieser Tabelle verbunden sind
    \item Eine neue Tabelle erstellen, die das gewünschte Format besitzt. Benenne diese mit einem temporären Namen
    \item Übermittle alle Inhalte der alten Tabelle in die neue Tabelle
    \item Lösche die alte Tabelle
    \item Verändere den Namen der neuen Tabelle in den gewünschten Namen
    \item Erstelle Indizes und Trigger nach dem Vorbild der alten Tabelle
    \item Lösche alle Views die durch die Veränderung betroffen sind und rekonstruiere diese
    \item Sollten Foreign Keys ursprünglich aktiviert sein, prüfe die Konsistenz der Datenbank durch ``PRAGMA foreign\_key\_check''
    \item Commit die Transaktion
    \item Aktiviere die zuvor deaktivierten Foreign Keys
\end{enumerate}

Da dieser Algorithmus bereits allgemeingültig ist und dieser für Veränderungen implementiert werden müsste, die durch Ruby Migration nicht realisiert werden können, wurde der SQLite Algorithmus für alle Veränderungen verwendet. Zusätzlich müssen keine ``Mischlösungen'' implementiert und weitere Ruby Bibliotheken eingebunden werden.    

\subsubsection{Anfragen - Editieren}
\label{subsubsec04:anfragen_editor_edit}
In diesem Abschnitt wird die Implementierung erläutert, wie einzelne Tabellen auf der Serverseite und somit in der Datenbank selbst verändert werden. Dabei wird der von SQLite vorgegebene Algorithmus zur eigentlichen Migration verwendet(siehe:~\ref{subsubsec04:sql_vs_ruby}).

Die Daten die vom Client bei einer Anfrage zum Verändern einer Tabelle versendet werden, sind in einer Liste bestehend aus JSON-Objekten, die die einzelnen Veränderungen darstellen. (genaueres siehe:~\ref{subsubsec04:kommunikation_cs})
Auf dem Server wird der gleiche Vorgang realisiert, der auf dem Client bereits realisiert wurde. Die einzelnen Befehle des Command Patterns, werden auf dem Server durch einzelne Funktionen abgebildet. Die zu jedem Command, auf Clientsseite, vorhandene ``Undo''-Funktion ist auf dem Server nicht abgebildet. Zu dem Zeitpunkt, wenn eine Änderung auf dem Server durchgeführt werden soll, sind die einzelnen Veränderungen bereits fest, und müssen nicht rückgängig gemacht werden.
Hierbei werden bereits vorhandene Funktionen verwendet. Es wird ein Objekt aus der Tabelle, die Verändert wird, von der Datenbank erstellt. Die Funktion Objekte aus Tabellen der Datenbank zu erschaffen, wurde bereits von Marcus Riemer erstellt. Die erstellten Funktionen in Anlehnung an die Commands vom Client, verändern das Objekt. Nachdem das Objekt verändert wurde, kann dieses auf die gleiche Weise wie beim Erstellen einer neuen Tabelle zu einem SQL-CREATE\_TABLE-Statement konvertiert werden.

\begin{description}
\item[Zuweisung der Spalten] \hfill\\
Damit der von SQLite vorgegebene Algorithmus verwendet werden kann, musste bei dem Überführen der Tabellendaten (siehe Schritt 5:~\ref{enum:sql_algo}) bekannt sein, welche Spalte der alten Tabelle mit welcher Spalte der neuen Tabelle korrespondiert.  
Der SQLite Befehl zum Hinzufügen von Daten aus einer Tabelle in eine andere ist in Listing~\ref{lst:insert_into} exemplarisch dargestellt:
\lstinputlisting[
  language=sql,
  caption=SQLite - Einträge aus einer Tabelle in eine andere übertragen,
  label=lst:insert_into,
  float=h,
  numbers=left
]{snippets/insert_into.sql}

Für die Zuweisung der alten Spalte zu der neuen Spalte wurde ein Hash in Ruby verwendet. Dieser wird zusammen mit der Tabelle in den einzelnen Funktionen verändert. Für Veränderungen die die Informationen einer Spalte verändern, muss der Hash nicht angepasst werden. Durch die Syntax des SQLite Befehls zum Übertragen der Tabelleneinträge, ist die Reihenfolge innerhalb des Befehls entscheidend, somit muss beim verändern der Reihenfolge der Spalten in der Tabelle, der Hash nicht angepasst werden. Die Fälle in denen der Hash angepasst wird ist, wenn ein Spaltenname sich verändert oder eine Spalte komplett entfernt wird.

\begin{figure}[ht]
    \frame{\includegraphics[width=0.5\textwidth]{images/hash_new.pdf}}
        \centering
        \caption{Die Darstellung der Spaltenzuordnung durch einen Hash}
        \label{pic:hash_new}
\end{figure}

\begin{figure}[ht]
  \begin{subfigure}[b]{0.45\textwidth}
    \includegraphics[width=\textwidth]{images/hash_rename.pdf}
    \caption{Umbenennen einer Spalte}
    \label{fig:hash_rename}
  \end{subfigure}\hfill
  \begin{subfigure}[b]{0.45\textwidth}
    \includegraphics[width=\textwidth]{images/hash_remove.pdf}
    \caption{Entfernen einer Spalte}
    \label{fig:hash_remove}
  \end{subfigure}
  \caption{Hash - Entfernen \& Umbenennen einer Spalte}
  \label{fig:hash_remove_rename}
\end{figure}

\item[Änderungen einzeln oder alle durchführen] \hfill\\
Mit der jetzigen Implementierung wird jede Änderung einzeln an der Datenbank durchgeführt. Durch die Implementierung des Hashs für die Spaltenzuweisung und der Abbildung der Tabelle auf dem Server, könnten alle Änderung auf dem Server durchgeführt werden, bevor diese an die Datenbank weitergeleitet werden. Dies würde die Zugriffe auf die Datenbank deutlich verringern, und somit die Synchronisierung auf die Datenbankzugriffe von verschiedenen Benutzern vereinfachen. Der Nachteil wäre, dass nicht festzustellen ist, welche Veränderung gegebenenfalls fehlgeschlagen ist. Da die Zielgruppe Anfänger im Thema Datenbanken ist, ist die Fehlerkommunikation ein wichtigerer Aspekt, der einen Vorrang hat.  

\item[Sicherung der Datenbank] \hfill\\
Bei der Veränderung der Datenbank in Einzelschritten, schlagen Veränderungen innerhalb der Kette von Veränderungen fehl, nachdem einige Änderungen bereits an der Datenbank durchgeführt wurden. Dabei muss die Datenbank beim Fehlschlagen der Veränderungskette auf den ursprünglichen Zustand zurückgeführt werden. In SQLite werden, anders als in anderen Datenbanksystemen, Verschachtelungen von Transaktionen nicht unterstützt. Dafür wird in SQLite eine Datenbank in einer einzelnen Datei gespeichert. \\ 
Es kann damit eine Kopie dieser Datei beim Anfang der Migration erstellt werden, um diese gegebenenfalls wiederherzustellen.

\end{description}

\subsubsection{Tests}
\label{subsubsec04:server_testing}
Um die einzelnen Funktionen zu testen, wurden die bereits von Marcus Riemer genutzten Möglichkeiten verwendet, um Unit-Tests zu schreiben. \\
Damit konnte sichergestellt werden, dass Funktionen sich in der Weise verhalten haben wie es erwartet war. Diese Tests waren die einzige Möglichkeit die Funktionalität zu testen, die nicht direkt per Oberfläche gesteuert werden konnten. Wie bereits in~\ref{fk_disclaimer} erwähnt, wurde keine Lösung gefunden für die Darstellung von zusammengesetzten Foreign Keys. Mit Tests wurde sichergestellt, dass der Server auch diese unterstützt, damit in Zukunft zusammengesetzte Foreign Keys im Client eingebaut werden können.

\subsubsection{Unerwartete Probleme}
\label{subsubsec04:server_problems}

Während der Entwicklung auf dem Server sind einige Schwierigkeiten aufgetreten, so muss unter anderem die \texttt{REGEXP}-Funktion bei jeder Verbindung mit der Datenbank definiert werden und das größte Problem lag in der Nutzung von \texttt{PRAGMA}-Funktionen.

\begin{description}
\item[PRAGMA foreign\_key\_check löst SQLException aus] \hfill\\
In dem Algorithmus zum Verändern der Tabelle wird die Konsistenz der Datenbank mit dem Befehl ``PRAGMA foreign\_key\_check'' geprüft. (Schritt 12:~\ref{enum:sql_algo}) \\
Das \texttt{PRAGMA} gibt eine Tabelle zurück, mit Einträgen die anzeigen welche Tabelle und welche Spalte die Konsistenz verletzen. 
Sollte die Datenbank inkonsistent sein, wird eine SQLException geworfen und in der Message der Exception ist die Tabelle mit den Einträgen enthalten. Die Exception vom Typ SQLException, wird bei den meisten Fehlern geworfen die mit SQL verbunden sind. Dies ist ein Problem, wenn alle geworfenen Fehler als Ergebnis der Konsistenzprüfung interpretiert werden. Aus diesem Grund musste die Prüfung in eine separate Funktion ausgelagert werden. \\
Ab der SQLite Version 3.16.0 die am 2. Januar 2017 erschienen ist, lassen sich \texttt{PRAGMA}-Funktionen, die eine Ausgabe liefern, innerhalb von \texttt{SELECT}-Statements benutzen. Diese werden als Tabellen interpretiert und sind erreichbar über den Funktionsnamen mit dem Präfix ``pragma\_''. Damit lässt sich das Auslösen einer SQLException mit dem Aufruf \texttt{SELECT * from pragma\_foreign\_key\_check} verhindern. Diese sehr junge Erweiterung, war auf der Entwicklermaschine noch nicht vorhanden, und ist nicht auf dem zukünftigen Server garantiert vorhanden, was dazu führt, dass vorerst die Implementierung per \texttt{PRAGMA}-Funktionen verwendet wird.
\end{description}

	\clearpage
	\section{Fazit}
\label{sec:conclusion}

Ein Blick in den Anhang zeigt, dass der Prototyp dem eingangs erwähnten Ziel\footnote{"`Mit dem Blattwerkzeug lassen sich gestützt durch \textit{Drag \& Drop-Editoren} für beliebige \texttt{SQLite}-Datenbanken \textit{Abfragen formulieren} und \textit{Oberflächen entwickeln}"', siehe Kapitel \fullref{sec:introduction}} durchaus gerecht wird: Zu praktisch beliebigen SQLite-Datenbanken können Abfragen und Oberflächen entwickelt werden.

Etwas detallierter lässt sich der Grad des Erfolges einer prototypischen Implementierung vor allem am finalen Entwicklungsstand festmachen: Welcher prozentuale Anteil der angestrebten Funktionalität wurde erreicht? Daher muss sich der "`fertige Prototyp"' an den in Kapitel \fullref{sec:principles} formulierten Zielen messen lassen. Auch einige spätere Kapitel, insbesondere \fullref{sec:sql-subset}, lassen sich als ein recht detaillierter Anforderungskatalog verstehen.

Dieses vermeintlich objektive Kriterium berücksichtigt aber nur einen Teil der für \idename{} zu relevanten Ziele. Da auf dem prototypischen Stand weiter aufgebaut werden soll, ist eine weitere Frage von Bedeutung: Kann man auf diesem Stand die Weiterentwicklung fortführen?

\subsection{Erreichte Ziele}

Der aktuelle Stand der Implementierung ist vor allem als ein Durchstich zu verstehen: Auch wenn in fast jedem Teilbereich noch Funktionaliät fehlt, kann das Zusammenspiel dieser Systeme schon gut erprobt werden. Insbesondere bei der Verbindung der Datenbank mit der Oberfläche, sowohl lesend als auch schreibend, haben sich im Laufe der konkreten Implementierung noch einige unerwartete Stolpersteine aufgetan. Das zugrundeliegende Fundament, also die internen und textuellen Darstellungen von \texttt{SQL} und \texttt{HTML}, sind nun aber stabil und darüber hinaus auch mächtiger, als der mit dem Drag \& Drop-Editor editierbare Stand.

Das Grundprinzip "`\textbf{Semantik vor Syntax}"' wurde erreicht. Der Drag \& Drop-Editor schließt Syntaxfehler kategorisch aus, sofern diese im kompilierten Quelltext doch auftreten wäre das eindeutig ein Fehler in der Codegenerierung, nicht jedoch des Entwicklers. Bei Fehlermeldungen für erkannte Fehlersituationen gibt es aber noch Verbesserungspotenzial: Aktuell werden fehlerhafte Eingaben einfach zugelassen und dann erst im Nachhinein als Fehler markiert. An dieser Stelle wäre das im Prinzip angesprochene "`kontinuierliche Feedback"' vermutlich am besten mit einer eingebauten, kontextsensitiven Hilfe unterstützen. Diese könnte auf Fehler mit einer kurzen Erläuterung reagieren und dabei demonstrieren, wied die richtige Vorgehensweise wäre.

Ob die erstellbaren Seiten durch "`\textbf{praktisch vorzeigbare Ergebnisse motivieren}"', kann nur ein praktischer Test mit der Zielgruppe zeigen. Ein Fortschritt gegenüber der nicht sonderlich hohen Hürde "`besser sein als Texteingaben in einer \texttt{SQL}-\texttt{IDE}"' ist aber durchaus zu erwarten.

Die "`\textbf{einfache Inbetriebnahme}"' ist eine Frage der Perspektive. Aus Sicht eines Schülers ist der einfache Aufruf einer \texttt{URL} in der Tat einfach, bisher hat \idename{} auch gut unter allen ad-hoc probierten Kombinationen aus Betriebssystemen (Windows, Mac\-OS, Linux) und Browsern (Firefox, Chrome, Edge) gut funktioniert. Für Lehrkräfte wird aktuell eine virtuelle Maschine bereitgestellt, der Umgang mit dieser ist aufgrund des fehlenden Webinterfaces noch relativ unbequem.

Eine Notwendigkeit des Wechsels auf eine "`normale"' Desktopanwendung hat sich zu keinem Zeitpunkt ergeben. Das technische Fundament aus Ruby mit Sinatra und Typescript mit Angular 2 hat sich ebenso bewährt, wie die Entscheidung, den größten Teil der Logik im Client zu belassen. Der Betrieb von \idename{} fühlt sich nach der initialen Ladezeit sehr flüssig an. Die Möglichkeit der Entwicklung von Unit- und End-to-End-Tests fügt sich gut in den Entwicklungszyklus ein.

\subsection{Nicht erreichte Ziele}

Das Ziel einer "`\textbf{schrittweise komplexeren Benutzeroberfläche}"' wurde zunächst hinten angestellt, da es mit einem ausgearbeiteten Konzept einhergehen sollte. Kapitel \fullref{sec:sql-subset-ranks} untersucht zwar die möglichen Einschränkungen für \texttt{SQL}, geht aber nicht auf \texttt{HTML} oder die möglichen Auswirkungen auf das Zusammenspiel beider Systeme ein. Mit dem aktuell noch überschaubaren Funktionsumfang ist der akute Bedarf nach diesem spezifischen Grundprinzip allerdings auch noch nicht gegeben.

Auch das Ziel der "`\textbf{Fortführung der entwickelten Projekte}"' mit externen Programmen wurde für den Prototypen hinten angestellt. Und zumindest die Unterstützung der Quelltext-Editoren sollte noch implementiert werden, bevor der Bedarf für einen Export überhaupt abgeschätzt werden kann.

Der Umfang der tatsächlich implementierten \texttt{SQL}-Funktionen ist noch nicht abgeschlossen. Zum jetzigen Zeitpunkt fehlen neben der Unterstützung der \texttt{AS}-Direktive im Editor (der Syntaxbaum unterstützt die Benennung schon) vor allem Funktionen und Gruppierungen. Dieser Umstand schränkt die umsetzbaren Projekte empfindlich ein: Stünden diese Funktionen schon zur Verfügung könnten zum Beispiel auch Webseiten für Sportvereine, inklusive der dynamischen Erzeugung von Tabellen, mit \idename{} umgesetzt werden.

Der Einsatz des Prototypen im Unterricht ist aktuell vor allem aufgrund der rudimentären Benutzerverwaltung nur schwer vorstellbar. Noch ist noch keine Registrierung von Benutzern implementiert, Projekte können noch nicht über die Webseite kopiert werden. Die Implementierung einer Benutzerverwaltung ist allerdings eine vor allem handwerkliche Aufgabe und wurde daher im Rahmen dieser Thesis nicht vorangetrieben.

Aber auch der Betrieb für Lehrkräfte gestaltet sich deutlich komplizierter als notwendig: Die Notwendigkeit einer eigenen (Sub-)Domain ist keine triviale Hürde. Die Annahme "`man wird doch wohl mal auf dem Nameserver der Schuldomain einen Eintrag für \idename{} anlegen können"' hat sich zwar nicht als haltlos erwiesen, behindert die initiale Inbetriebnahme aber merklich.

\subsection{Weiterentwicklung}

Eine wesentliche Frage bei der Weiterentwicklung von \idename{} ist die Frage nach dem Zeitpunkt, zu dem man die eigentliche Zielgruppe (sowie deren Lehrer) in die Entwicklung mit einbezieht. Oder anders ausgesdrückt: Es stellt sich die Frage nach dem minimalen Satz an implementierten Funktionen, mit denen man sinnvoll Feedback bei Lehrern und Schülern einholen kann.

\begin{description}
\item[SQL: Unterstützung von Funktionen] \hfill\\

\item[SQL: Unterstützung von \texttt{GROUP BY}] \hfill\\

\item[SQL: Schemaeditor] \hfill\\
  
\end{description}



%%% Local Variables:
%%% mode: latex
%%% TeX-master: "thesis"
%%% End:

	\clearpage 

	\section{Literaturverzeichnis}
	\nocite{*}
	\printbibliography
	
	\clearpage
	\section{Eidesstattliche Erklärung}
Hiermit erkläre ich an Eides Statt, dass ich die vorliegende Arbeit selbstständig und nur unter Zuhilfenahme der ausgewiesenen Hilfsmittel angefertigt habe.  Sämtliche  Stellen  der  Arbeit,  die  im  Wortlaut  oder dem  Sinn  nach  anderen  gedruckten  oder im  Internet  verfügbaren  Werken  entnommen  sind,  habe  ich  durch  genaue  Quellenangaben kenntlich gemacht. 

\vspace{2em}


Ort, Datum\hfill Unterschrift 
	\clearpage
	
\section{CD-ROM}
Alle auf der beigefügten CD-ROM enthaltenen Daten sind auch im Blattwerkzeug-Git-Repository unter \url{http://blattwerkzeug.de/forward/git-repository} verfügbar.

\end{document}
%%% Local Variables:
%%% mode: latex
%%% TeX-engine: xetex
%%% TeX-master: t
%%% End:
