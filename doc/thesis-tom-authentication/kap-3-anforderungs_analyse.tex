\section{Anforderungsanalyse}
\label{sec: analyze}
Das Ziel dieser Thesis ist ein standardisiertes Registrierungsverfahren. Außerdem eine Authentifikation über \gls{oAuth2} oder ein Passwort. Hinzu kommt eine Möglichkeit den Nutzer über den Server zu Autorisieren und ihm nach spezifischen Benutzerrollen unterschiedlichen Inhalt zu präsentieren. Dabei soll es dem Nutzer nicht gestattet sein durch Manipulation seines Clients mit verfälschten Daten Zugriff auf für ihn nicht zugreiffbare Daten zu erhalten. Um das Ziel dieser Thesis zu erreichen muss sich vorerst mit Ruby und dem Webframework Rails auseinandergesetzt werden. Außerdem ist es vonnöten sich intensiver mit Angular zu beschäftigen, da dieses bisher nur Oberflächlich behandelt wurde.

\subsection{Derzeitiges Projekt}
\label{sec: current_project}
Zum derzeitigen Zeitpunkt ist Blattwerkzeug eine Lernplattform in der jeder Nutzer jedes Projekt bearbeiten kann. Dafür wurde Serverseitig ein Benutzername und Passwort, in beiden Fällen \enquote{user}, hinterlegt. Zusätzlich ist es möglich das Adminpanel ohne weitere Autorisierungsabfrage zu betätigen.

\subsubsection{Client}
Der Clientseitige-Teil von Blattwerkzeug basiert auf Angular, Angular-Material und Bootstrap, hierbei sind Angular-Material und Bootstrap Gestaltungsframeworks. Bootstrap welches hauptsächlich auf \gls{CSS} und \gls{HTML} basiert und Angular-Material welches explizit Module für Angular bereitstellt.

\subsubsection{Server}
Der Serverseitige-Teil baut auf Ruby on Rails und bietet ein \gls{API} zur Kommunikation mit dem Server. Die Daten werden mittels unterschiedlicher Anfragemethoden abgefragt/übermittelt und vom Server verarbeitet.

\subsection{Anforderungen}
\label{sec: requirement}
Im Verlauf dieser Sektion werden die Anforderungen, die diese Thesis erfüllen soll, detailliert erläutert.

\subsubsection{Unterschiedliche Anmeldemöglichkeiten}
Ein Benutzer sollte die Möglichkeit haben sich mit mehreren Konten zu verknüpfen. Das bedeutet, er sollte sich mit Google authentifizieren können, jedoch auch mit einem Passwort.

\subsubsection{Anmeldung mittels \gls{oAuth2}}
Eine Authentifizierung mittels oAuth2 soll über Google und GitHub möglich sein. Die von Google oder GitHub zurückgelieferten Daten sollen in der Datenbank abgespeichert werden. Darüber hinaus soll beim vorhanden sein spezifischer Daten, wie beispielsweise der E-Mail, eine automatische Zuweisung spezieller Datenbankfelder des Nutzers geben.

\subsubsection{Anmeldung mittels Passwortes}
Für eine Anmeldung mittels Passwortes muss zuerst eine Möglichkeit gegeben sein ein Konto zu erstellen. Bei der Erstellung eines Kontos sollten vier Felder geben sein. Das erste für den Benutzernamen, das zweite für die E-Mail, das dritte und vierte für das Passwort und die Passwort Bestätigung. Sobald der Benutzer seine Daten erfolgreich abgeschickt hat, sollte das vom Client als Klartext verschickte Passwort auf dem Server verschlüsselt werden. Nachdem der Benutzer sein Konto erstellt hat soll eine Bestätigungsmail an die angegebene E-Mail gesendet werden. Diese Bestätigungsmail sollte einen Hyperlink beinhalten mit dem das vom Nutzer erstellte Konto bestätigt werden kann. Sollte diese E-Mail der Nutzer nicht empfangen, muss die Möglichkeit geben sein eine erneute Bestätigungsmail zu versenden. Erst nachdem das Konto bestätigt wurde, soll es dem Nutzer gestattet sein dieses Konto zu verwenden. Falls ein Nutzer sein Passwort vergessen haben sollte, muss es zusätzlich eine Funktion zum Passwort wiederherstellen geben.

\subsubsection{Authentifizierung nach Anmeldung}
Um vom Server als angemeldeter Benutzer authentifiziert zu werden, sollten die Benutzerdaten in einem \gls{JWT} gespeichert werden. Dieser \gls{JWT} wird bei der Anmeldung eines Benutzers an den Client übermittelt. Ab dem Zeitpunkt wird dieser \gls{JWT} bei jeder Anfrage an den Server mit übermittelt. Sobald eine Anfrage den Server erreicht, muss zwangsläufig der \gls{JWT} auf seine Gültigkeit geprüft werden. Sollte dieser \gls{JWT} eine nicht gültige Signatur haben oder bereits abgelaufen sein, darf keine weitere Aktion auf dem Server erfolgen.

\subsubsection{Sicherheit und Login}
Die Einstellungen im Bereich Sicherheit und Login sollten einem Benutzer erlauben sein bereits erstelltes Konto mit weiteren Konten zu verknüpfen. Hierbei sollte der Benutzer eingeloggt sein und einen bestimmten Provider auf seiner Einstellungsseite auswählen können. Nachdem sich der Benutzer bei dem ausgewählten Provider authentifiziert hat, sollte dieses Konto dem Nutzer hinzugefügt werden.  Außerdem sollten die Benutzereinstellungen eine Verwaltung der verknüpften Konten beinhalten. Das heißt der Benutzer sollte zu jeder Zeit entscheiden können welche dieser verknüpften Konten beständig bleiben oder welche gelöscht werden.

Sollte sich ein Benutzer mittels Passwortes auf Blattwerkzeug registriert haben, muss es für diesen Nutzer eine Möglichkeit geben sein Passwort zu ändern, selbst wenn dieser bereits sein Konto zusätzlich mit Google oder GitHub verknüpft hat. Sollte ein Benutzer bereits ein vorhandenes Konto mit einem Passwort haben, sollte das neu zu verknüpfende Konto automatisch das Passwort des bereits vorhandenen annehmen. Falls das zu verknüpfende Konto, das erste mit einem Passwort sein sollte, muss dafür in den Benutzereinstellungen eine extra Passworteingabe erscheinen, in dem das zu verwendende Passwort angegeben wird.

Da es in Blattwerkzeug bei der Benutzernamensgebung zum jetzigen Stand keine einmaligen Benutzernamen geben muss, soll es dem Benutzer in den Benutzereinstellungen außerdem möglich sein, seinen Benutzernamen zu ändern.

\subsubsection{Rollen und Autorisierung}
Die Rollen sollten sich in globale und Ressourcen spezifische Rollen unterteilen. Dabei sind Globale-Rollen wie in Sektion ~\ref{sec: rolify} beschrieben keiner spezifischen Ressource zugewiesen. Zwei Beispiele globaler Rollen wären \enquote{user} und \enquote{admin} oder \enquote{guest}, \enquote{user} und \enquote{admin}. Ressourcen spezifische Rollen beziehen sich auf beispielsweise ein Projekt. Sollte ein Projekt von einem angemeldeten Benutzer erstellt werden, sollte dieser eine jeweilige Rolle oder eine Datenbank-Beziehung für dieses Projekt erhalten. Mit dieser Rolle sollte es dem Benutzer möglich sein, sein Projekt zu bearbeiten oder zu löschen. Außerdem sollte die Möglichkeit gegeben sein, einem anderen Benutzer eine Rolle zuzuweisen mit der das Bearbeiten eines von ihm nicht erstellten Projektes ermöglicht wird. Administratoren mit der \enquote{admin} Rolle sollten jedoch Zugriff auf jedes Projekt haben.

\todo[inline]{Auflistung von nötigen Rollen und Rechten}

Für eine Autorisierung mittels Rollen sollten bestimmte Regeln für die jeweiligen Controller Funktionen festgelegt werden. Diese Regeln sollten mittels Pundit erstellt werden. Innerhalb der Regeln sollte die Überprüfung der Rollen des angemeldeten Nutzers stattfinden.


\subsubsection{Bedienelemente und Routen}
Die Bedienelemente die ein Benutzer zu sehen hat, müssen jeweils von den Rollen abhängen. Dazu ist es möglich bei jedem Bedienelement den Server nach der Berechtigung zu fragen oder jedoch Clientseitig Pundit nach zu bauen. Sollte ein Benutzer eine Route mit nicht ausreichender Berechtigung besuchen, muss ihm der Zugriff verwehrt werden. Die Bedienelemente sollten benuzterfreundlich sein, da es sich bei Blattwerkzeug, wie in Sektion \ref{sec:blattwerkzeug} beschrieben, um ein Werkzeug zum lernen bestimmter Bereiche der Informatik handelt.

\todo[inline]{Code-Beispiel für Verwendung als Angular-Komponente (Template)}
