\documentclass[paper=a4,fontsize=12pt,parskip=half,twoside]{scrartcl}

%% packages
\usepackage[ngerman]{babel}
\usepackage[utf8]{inputenc}
\usepackage[draft]{optional}
\usepackage[T1]{fontenc}
\usepackage{lmodern}
\usepackage{amsmath}
\usepackage[usenames,dvipsnames,svgnames,table]{xcolor}
\usepackage[absolute]{textpos}
\usepackage{hyperref}
\usepackage[format=plain,labelfont=bf]{caption}
\usepackage{subcaption}
\usepackage{listings}
\usepackage{tabularx}
\usepackage{wrapfig}
\usepackage{xargs}   % Use more than one optional parameter in a new commands
\usepackage[colorinlistoftodos,prependcaption]{todonotes}
\usepackage{dirtree}
\usepackage{float}
\usepackage{newfloat}
\usepackage{framed}
\usepackage{tikz}
\usepackage[simplified]{pgf-umlcd}
\usepackage[pass]{geometry} % Use `showframe` to test for overly full boxes
\usepackage[onehalfspacing]{setspace}
\usepackage{multicol}
\usepackage{enumitem}       % Narrower enumerations
\usepackage[normalem]{ulem} % Striking out stuff
\usepackage{datetime}       % \currenttime

%% additional ressources
\newcommand{\idename}{Blatt\-Werkzeug}

\newcommand{\doctitle}{A grammar based approach for syntax-free IDE generation}
\newcommand{\docsubtitle}{}
\newcommand{\docauthors}{Marcus Riemer}
\newcommand{\docpdfauthors}{Marcus Riemer}

\newcommand{\person}[1]{\textsc{#1}}

\newcommand{\fullref}[1]{\ref{#1}~\enquote{\nameref{#1}}}

\newcommand*\circled[1]{\tikz[baseline=(char.base)]{
    \node[shape=circle,draw,inner sep=2pt] (char) {#1};}}

\newcommand*\cleartoleftpage{%
  \clearpage
  \ifodd\value{page}\hbox{}\newpage\fi
}

\usepackage{color}
\definecolor{lightgray}{rgb}{.9,.9,.9}
\definecolor{darkgray}{rgb}{.4,.4,.4}
\definecolor{purple}{rgb}{0.65, 0.12, 0.82}

\lstset{
  numberstyle=\tiny
}

\lstdefinelanguage{SQL}{
  keywords={INSERT, UPDATE, DELETE, SELECT, FROM, WHERE, GROUP BY, SET, INTO, VALUES, LIMIT},
  keywordstyle=\color{blue}\bfseries,
  ndkeywords={strftime},
  ndkeywordstyle=\color{OliveGreen}\bfseries,
  identifierstyle=\color{black},
  sensitive=false,
  comment=[l]{//},
  morecomment=[s]{/*}{*/},
  commentstyle=\color{purple}\ttfamily,
  stringstyle=\color{red}\ttfamily,
  morestring=[b]',
  morestring=[b]"
}

\lstdefinelanguage{HTML}{
  keywords={select, option, h1, template, h2, div, ul, li, table, thead, tr, th, td, tbody, tr, value, innerHtml, class, id},
  keywordstyle=\color{blue}\bfseries,
  ndkeywords={for, in, include, endfor, endif, else, if, ngIf, ngFor, each},
  ndkeywordstyle=\color{OliveGreen}\bfseries,
  identifierstyle=\color{black},
  sensitive=false,
  comment=[l]{//},
  morecomment=[s]{/*}{*/},
  commentstyle=\color{purple}\ttfamily,
  stringstyle=\color{red}\ttfamily,
  morestring=[b]',
  morestring=[b]"
}

\lstdefinelanguage{JavaScript}{
  keywords={typeof, new, true, false, catch, function, return, null, catch, switch, var, let, const, if, in, while, do, else, case, break},
  keywordstyle=\color{blue}\bfseries,
  ndkeywords={class, export, boolean, number, string, throw, extends, implements, import, this, constructor, public, private},
  ndkeywordstyle=\color{darkgray}\bfseries,
  identifierstyle=\color{black},
  sensitive=false,
  comment=[l]{//},
  morecomment=[s]{/*}{*/},
  commentstyle=\color{purple}\ttfamily,
  stringstyle=\color{red}\ttfamily,
  morestring=[b]',
  morestring=[b]"
}

\lstdefinelanguage{Ruby}{
  keywords={do, def, end, if, else, return},
  keywordstyle=\color{blue}\bfseries,
  ndkeywords={each, map, zip},
  ndkeywordstyle=\color{darkgray}\bfseries,
  identifierstyle=\color{black},
  sensitive=false,
  comment=[l]{\#},
  morecomment=[s]{/*}{*/},
  commentstyle=\color{purple}\ttfamily,
  stringstyle=\color{red}\ttfamily,
  morestring=[b]',
  morestring=[b]"
}

\lstset{
   backgroundcolor=\color{lightgray},
   extendedchars=true,
   basicstyle=\footnotesize\ttfamily,
   showstringspaces=false,
   showspaces=false,
   numberstyle=\footnotesize,
   numbersep=9pt,
   tabsize=2,
   breaklines=true,
   showtabs=false,
   captionpos=b
}

\lstset{escapeinside={(*@}{@*)}}

\newcommandx{\unsure}[2][1=]{\todo[linecolor=red,backgroundcolor=red!25,bordercolor=red,#1]{\textbf{Unsure}: #2}}
\newcommandx{\change}[2][1=]{\todo[linecolor=blue,backgroundcolor=blue!25,bordercolor=blue,#1]{\textbf{Change}: #2}}
\newcommandx{\info}[2][1=]{\todo[linecolor=OliveGreen,backgroundcolor=OliveGreen!25,bordercolor=OliveGreen,#1]{\textbf{Info}: #2}}
\newcommandx{\missing}[2][1=]{\todo[linecolor=Plum,backgroundcolor=Plum!25,bordercolor=Plum,#1]{\textbf{Missing}: #2}}

\newcommand{\warning}[2][Achtung]{
  \begin{framed}
    \textbf{#1}: #2
  \end{framed}
}

% Using \DeclareFloatingEnvironment leads to a strange extra dot in the
% caption numbering, see http://tex.stackexchange.com/questions/330638/getting-rid-of-an-extra-dot-in-the-numbering-of-my-new-float-environment

\newfloat{diagram}{thp}{lop}
\floatname{diagram}{Diagramm}

%%% Local Variables:
%%% mode: latex
%%% TeX-master: "thesis"
%%% End:


\usetikzlibrary{calc,intersections,shapes}

% neue Kopfzeilen mit fancypaket
\usepackage{fancyhdr} %Paket laden
\pagestyle{fancy} %eigener Seitenstil
\fancyhf{} %alle Kopf- und Fußzeilenfelder bereinigen
\fancyhead[L]{\nouppercase{\leftmark}} %Kopfzeile links
\fancyhead[C]{} %zentrierte Kopfzeile
\fancyhead[R]{\thepage} %Kopfzeile rechts
\renewcommand{\headrulewidth}{0.4pt} %obere Trennlinie
%\fancyfoot[C]{Confidential}
%\renewcommand{\footrulewidth}{0.4pt} %untere Trennlinie

%% meta information
\hypersetup{
  pdftitle={\doctitle},
  pdfsubject={\docsubtitle},
  pdfauthor={\docpdfauthors},
  colorlinks=true,
  linkcolor=blue,
  urlcolor=blue,
}

\title{\doctitle}
\subtitle{\docsubtitle}
\author{\docauthors}
\date{\today{}}


 %% document
\begin{document}
\pagenumbering{Roman}
\newgeometry{hmarginratio=1:1}   %% make layout symmetric

\begin{titlepage}
  \vspace*{25ex}
  \begin{textblock*}{\paperwidth - 4cm}[1,0](\paperwidth - 2cm, 1cm)
    \centering
    \includegraphics[height=2.5cm]{images/fhlogo}
    \hfill
    \includegraphics[height=2.5cm]{images/logo-cau-kiel.png}
  \end{textblock*}
  \begin{center}
    \sffamily{}
    \includegraphics{images/blattwerkzeug-caption} \\[4ex]
    {\Large\docsubtitle}
  \end{center}
  \vspace*{20ex}
  \begin{tabbing}
    \hspace{8em} \= \hspace{14em} \= \hspace{8em} \= \kill

    Eingereicht am: \> 31. Oktober 2016 \\[5ex]
    Autor: \> Marcus Riemer, B.Sc. \\
    Matr.-Nr.: \> 100478 \\
    E-Mail: \> \href{mailto:mri@fh-wedel.de}{mri@fh-wedel.de} \\

    \\
    Betreuer: \> PD. Dr. Frank Huch  \> Prof. Dr. Ulrich Hoffmann \\
    E-Mail: \> \href{mailto:fhu@informatik.uni-kiel.de}{fhu@informatik.uni-kiel.de} \> \href{mailto:uh@fh-wedel.de}{uh@fh-wedel.de}
  \end{tabbing}
  \vfill
  \centering \href{https://creativecommons.org/licenses/by-sa/4.0/}{\includegraphics{images/licenselogo}}
\end{titlepage}

\restoregeometry                 %% restore the layout

%%% Local Variables:
%%% mode: latex
%%% TeX-master: "thesis"
%%% End:


\newpage

\section*{Unfertige Arbeit!}

An dieser Arbeit ist \sout{noch nichts fertig} das Ende unfertig! Ich habe sie online gestellt, um einigen hilfreichen Personen ein einfaches mitlesen zu ermöglichen. Du, lieber Helfer, darfst diese Arbeit in dem wohligem Bewusstsein lesen, dass du mir damit bestimmt weiterhilfst\footnote{... and then, there will be cake.}. Vielen Dank für deine Unterstützung!

Momentan lassen sich die unten aufgeführten Kapitel einigermaßen sinnvoll lesen. Sie sind inhaltlich weitestgehend vollständig (sofern ich das nicht einschränke), können allerdings Spuren von Rechtschreib-, Grammatik- oder Layoutfehlern enthalten.

Dieser Stand der Arbeit ist vom \today{} \currenttime{}, das Kürzel der dazugehörigen \texttt{git}-Revision lautet \texttt{\input{GIT_REV}}. Solltet ihr mir euer Feedback nicht in Form eines annotierten \texttt{PDF}-Dokumentes zukommen lassen, helfen mir diese beiden Informationen eure Seitenzahlen nachzuvollziehen.

\begin{itemize}[noitemsep]
\item \ref{sec:introduction}~\nameref{sec:introduction} fehlen noch Vorwärtsreferenzen und wird möglicherweise noch erweitert. Mir fehlt bloß noch eine zündende Idee was da noch fehlt.
\item \ref{sec:related-work}~\nameref{sec:related-work} ist vollständig.
\item \ref{sec:requirements}~\nameref{sec:requirements} fasert bei \ref{sec:drag-drop-ui-editor}~\nameref{sec:drag-drop-ui-editor} sehr abrupt aus, genau da ist noch Baustelle.
\item \ref{sec:implementation-analysis}~\nameref{sec:implementation-analysis} kann bis \ref{sec:data-model-query}~\nameref{sec:data-model-query} gelesen werden, dort ist aktuell einfach Schluss.
\end{itemize}

\subsubsection*{Änderungen vom 24.10.2016}
\begin{itemize}[noitemsep]
\item \fullref{sec:introduction} um Hinweis auf Prototypen im Web ergänzt.
\item \fullref{sec:principles} bei "`Einfache Inbetriebnahme"' um Smartphones und Tablets ergänzt.
\item Lektoriert bis \fullref{sec:design-sql-editor}.
\end{itemize}

\subsubsection*{Änderungen vom 20.10.2016}
\begin{itemize}[noitemsep]
\item \ref{sec:introduction}~\nameref{sec:introduction} und \ref{sec:related-work}~\nameref{sec:related-work} wurden lektoriert.
\item \ref{sec:pro-con-wysiwyg-editor}~\nameref{sec:pro-con-wysiwyg-editor} wurde ergänzt.
\end{itemize}

\subsubsection*{Änderungen vom 19.10.2016}
\begin{itemize}[noitemsep]
\item \ref{sec:editor-different-versions}~\nameref{sec:editor-different-versions} wurde aus \ref{sec:design-sql-editor}~\nameref{sec:design-sql-editor} verschoben und etwas erweitert.
\item \ref{sec:complex-ui-elements}~\nameref{sec:complex-ui-elements} ist neu.
\end{itemize}

\newpage
\tableofcontents{}

\cleardoublepage

\pagenumbering{arabic}

\cleardoublepage
\section{Einleitung}
\label{sec:introduction}

SQLino ist bisher auf die Erstellung von Seiten mit Texten und Tabellen begrenzt. Praktisch relevante Webseiten verwenden allerdings fast ausnahmslos auch grafische Gestaltungselemente. Darunter fallen zum Beispiel Fotos von im Text beschriebenen Sachverhalten, bei denen auf Autos gefallene, entwurzelte Bäumes, Gebirgspanoramen, Strände, ... zu sehen sind. Gemäß dem Motto \enquote{ein Bild sagt mehr als 1000 Worte} profitiert hier der Leser von den Bildern. Darüber hinaus gibt es auch Sachverhalte die sich kurz und prägnant durch Diagramme, nicht aber durch Texte erfassen lassen. Dementsprechend ist die Ergänzung um grafische Inhalte für SQLino sowohl praktisch relevant als auch (vermutlich) für die Schülerinnen und Schüler \footnote{Zugunsten der leichteren Lesbarkeit wird in dieser Thesis auf die Ausdifferenzierung des Geschlechts von Personen im Folgenden verzichtet, da es für den Sachverhalt irrelevant ist und den Lesefluss stört.  Wenn in dieser Thesis Personen genannt werden, steht das grammatikalische Geschlecht in keinem Zusammenhang zum tatsächliche Geschlecht der jeweilig genannten Personen und es sind immer alle Personen gemeint, auf die die restliche Beeschreibung zutrifft, unabhängig des Geschlechts.} motivierend.

\todo[inline]{Beispielbilder welche Szenarien uns für die Bildnutzung vorschweben.}

Die Verwendung von Bildern auf Webseiten bereitet jedoch auch immer wieder Probleme. Neben deutlich zu hohen Dateigrößen, welche Ladezeiten in die Länge ziehen und bei
Mobilgeräten das in Deutschland im Vergleich zu anderen Ländern um
Größenordnungen geringere Datenvolumen unötig schnell aufbrauchen - von der
Unbenutzbarkeit solcher Webseiten in der ``Edge-Wüste'' der ländlichen
Mobilfunknetzte und Breitbandanschlüsse, die ihrem Namen nicht gerecht werden
mal ganz abgesehen - \todo{Technische Gründe}


ist ein weiteres Problem die mangelnde auseinandersetzung
mit dem Urheberrecht und Nutzungslizenzen\todo{}. Sogenannte Soziale Netzwerke und
Diskussionsplatformen verleiten den Nutzer Bilder einzustellen und dem Anbieter der
Platform laut AGB diverse Rechte an diesen Bildern zu übertragen. Da der
durchschnittliche Nutzer weder die AGB gelesen hat, noch sich dessen bewusst
ist, dass die Befugnis die Rechte an den Bildern Dritter an den
Platformbetreiber weiterzugeben bei besagten Dritten liegt und nicht bei der
eigenen Person.\todo{Rechtliche Gründe}

\todo[inline]{Anspruch: Technisch und rechtlich "standarmäßig" korrekt / "sicher".}

Ein weiteres Problem ist die Verwaltung dieser Bilder. Ein Bild wird bei
irgendeinem gerade angesagtem Imagehoster abgelegt, eingebunden und vergessen.
Andere Bilder werden anderswo abgelegt, eingebunden und vergessen. Einige der
Imagehoster liefern nach tausend Aufrufen im Monat das Bild nicht mehr aus ohne
sich dafür bezahlen zu lassen, andere ändern nach einiger Zeit ihr Konzept und
lassen einbetten ihrer Bilder auf fremden Seiten nicht mehr zu und zeigen die
Bilder nur noch auf ihrer eigenen Werbefinanzierten Webseite an und wiederum
andere haben in der zwischenzeit Insolvenz angemeldet und alle dort
gespeicherten Bilder sind nicht mehr verfügbar.\todo{Also selber machen.}

Diese Thesis beschreibt den Prototypen einer Bildverwaltung für SQLino, die auf
diese Probleme eingeht, indem verwendete Bilder an einem Ort gesammelt und
verwaltet werden, den Bildern Metadaten wie Urheber und Lizenz zugeordnet
werden und dem Betrachter der Webseite dargestellt werden und Bilder beim
ausliefern auf die für das Endgerät geeignete Größe herunterskaliert werden.

Ein weiteres bearbeites Problem ist das Verwalten von Programmabhängigkeiten, die beim Entwickeln und Testen der Software sowie deren Produktivbetrieb zu berücksichtigen sind. \todo{Wichtig hier: Einstieg für neue Mitarbeiter, Testen mit Cloudservice}

%%% Local Variables:
%%% mode: latex
%%% TeX-master: "thesis"
%%% End:


\cleardoublepage
\section{Vergleichbare Arbeiten und Produkte}
\label{sec:related-work}

TODO

%%% Local Variables:
%%% mode: latex
%%% TeX-master: "thesis"
%%% End:


\cleardoublepage
\section{Anforderungsanalyse}
\label{sec:requirements}

In diesem Kapitel werden die Anforderungen beschrieben die an die
Containerisierung und die Bildverwaltung gestellt wurden.

Neben der Bildverwaltung stand auch die Verbesserung der Zugänglichkeit der
Hosting und Entwicklungsumgebung. Docker ist eine verbreitete Lösung um
Anwendungen in sogenannten Containern zu isolieren, die weitgehend
unabhängig von der verwendeten Hardware ist und auch auf den meisten
verbreiteten Betriebssystemen eingesetzt werden kann, ohne dass die auf dem
Hostsystem installierte zusätzlich installierte Software die Anwendung im
Container beeinflusst, sodass der Einfluss betriebssystemspeziefischer
Eigenheiten auf die Anwendung weitgehend vermieden werden kann.

Zur Bildverwaltung gehöhren ein API, die es ermöglicht, Bilder und
essentielle Metadaten anzulegen, auszugeben, zu bearbeiten und zu löschen,
eine Bibliothek, die die durch die API bereitgestellten Funktionen Serverseitig
umsetzt und eine Frontendintegration, die dem Verwalter eines Projekts das
Anlegen, Betrachten, Modifizieren und Löschen der Bilder und Metadaten
ermöglicht sowie die Verwendung der Bilder in Projektvorschau, dem
Seiteneditor und den Datenbanktabellen.

\subsection{Containerisierung}

Das Projekt verwendet drei Ausprägungen von Images, jeweils eins für
Produktivumgebung, Entwicklungsumgebung und Testumgebung. 

\subsubsection{Entwicklungsumgebung}

Das Entwicklungsimage stellt die während der Softwareentwicklung
benötigte Umgebung zur Verfügung und erwartet das Einbinden des
Quellcodes via Volume, sodass eine Änderung am Quellcode auf dem Hostsystem
sofort im Container sichtbar wird und zum neucompilieren des Codes durch den im
Entwicklungsmodus laufenden Server führt. So ist es möglich auf einem
Entwicklungsgerät zu arbeiten ohne die Abhängigkeiten im Betriebssystem
zu installieren und somit inkompatible Versionen von Abhängigkeiten mit
anderen Projekten zu vermeiden. Weiterhin ist es so möglich unter Windows
und MacOS zu entwickeln, da hier seitens Docker Werkzeuge zum Ausführen von
Containern mittels der Linux Distrubution boot2docker in einer sehr
minimalistischen durch die Werkzeuge verwalteten virtuellen Maschiene.

\subsubsection{Testumgebung}

Das Testimage weist Ähnlichkeiten zum Entwicklungsimage auf, enthält
aber noch den Chromium Browser, der im Headless Mode für Tests des Clients
genutzt wird.

\subsubsection{Produktivumgebung}

Das Produktivimage enthält eine fertig gebaute Version der Software sowie die
zum Betrieb nötigen Abhängigkeiten, die Versionierung des Images richtet sich
nach der Version der Software, ein Update wird durch das herunterladen/bauen des
Images mit der entsprechend höheren Version und austauschen des laufenden
Containers mit einem Container der neueren Version realisiert.

\subsection{Bildverwaltung}

Der erste zu klärende Punkt der Bilddatenbank war, wie Bilder zur Datenbank
hinzugefügt werden können. Zur Auswahl standen der Dateiupload im Projekteditor
durch den Autoren des Projekts, Import aus einer öffentlichen Bilddatenbank,
Import aus einer eigenen durch den Administrator gepflegten Bilddatenbank.
Eine dieser Möglichkeiten soll mit der Implementierung der Bildverwaltung direkt
umgesetzt werden, die anderen möglicherweise später folgen.

Vorteile der öffentlichen Bilddatenbank ist der geringe Aufwand beim Import von
Bildern, bei ausschließlicher Nutzung von Datenbanken die alle Bilder unter
Creative Commons Zero (CC0) veröffentlichen, wie beispielsweise openclipart,
entfällt zudem die Notwendigkeit Metadaten wie Urheber und Lizenz zu Speichern
oder anzuzeigen, bei Datenbanken die auch andere Creative Commons Lizenzen
(CC-BY, CC-BY-SA,CC-BY-NC, CC-BY-SA-NC, CC-BY-ND, CC-BY-ND-NC) oder
vergleichbare zulassen sind Urheber und Lizenz zumindest Maschienenlesbar
hinterlegt, sodass im einfachsten Fall die Auswahl der Bilddatenbank und die ID
des Bildes für den Import ausreichen, im angenehmen Fall ist im Projekts Editor
ein Suchinterface mit Anbindung an die API der Bilddatenbank sodass eine oder
alle Bilddatenbanken aus dem Projekteditor heraus durchsucht werden können ohne
dass dem Projektautoren die jeweilige Bilddatenbank bekannt sein muss.

Die Nachteile der öffentlichen Bilddatenbanken als initial einziges Verfahren
zum hinzugefügen von Bildern ist die Verwendung eigener Werke. Ein eigenes Werk
müsste erst einer angebundenen Bilddatenbank hinzugefügt werden um es verwenden
zu können. Dies erfordert zum einen die Bereitschaft des Urhebers das Werk unter
einer der von der Bilddatenbank erlaubten Lizenz zu veröffentlichen, sowie die
Eröffnung eines Accounts bei der jeweiligen Bilddatenbank. Da es sich bei der
Zielgruppe von SQLino um Schüler handelt, die auch zum Teil zu jung sind um
Verträge wie den zur Eröffnung eines Accounts einzugehen, wodurch sie von der
Verwendung eigener Werke ausgeschlossen sind bis der Dateiupload implementiert
ist, weshalb diese Importmethode als vorerst einzige ausscheidet.

\todo{warum die admindatenbank nicht genommen wird: kann auch aus oeffentlichen datenbanken importieren,
  freigabeprozess fuer eigene werke, vorauswahl, hoher aufwand}

Die zusammen mit der Bildverwaltung zu implementierende Variante ist daher der
Dateiupload im Projekteditor. Der Import von Bildern aus Bilddatenbanken ist so,
zwar unter höherem Aufwand, möglich, und durch manuell anzugebenden Urheber und
Lizenz können Schüler so für die Notwendigkeit der Auseinandersetzung mit
Urheberrecht bei Veröffentlichung von Bildern sensibilisiert werden.

\subsubsection{Lizenzen \& Metadaten}

Da Bilder nicht alle unterstützenswerten Bildformate Metadaten über
Urheber und Lizenz standatisiert haben, ist es nicht möglich die aus
rechtlicher Sicht Notwendigen Metadaten zum Urheber und zur Lizenz der Bilder
in der Datei selbst zu speichern und aus rechlicher Sicht, reicht die bloße
Anwesenheit der Daten in der Datei nicht um Beispielsweise die Anforderung der
Urhebernennung zu erfüllen, weshalb die aus rechlicher Sicht notwendigen
Metadaten in einer zusätzlichen Datenquelle gehalten werden müssen.

Die aus rechtlicher Sicht notwendigen Daten sind Name des Urhebers mit Link auf
eine Webpräsenz und der Name der Lizenz unter der das Bild verwendet wird
mit Link auf die Lizenz.

\subsubsection{Darstellungsformen}

Für die Bilder sollen zwei Darreichungsformen existieren: Die Abbildung
(Figure) und das gestalterische Element (Graphic), die sich durch ihre
Einbettung und den Ort der Quellenangaben unterscheiden. Die Abbildung
eingerahmt, die Metadaten am Bild dargestellt, das Gestalterische Element bettet
das Bild in einen Link ein, der auf die Quellenangabe am Ende der Seite verweist.

\subsubsection{Darstellung von Quellenangaben}

Bei Abbildungen werden Name des Bildes, Urheber des Bildes und Name der Lizenz
direkt unter dem Bild innerhalb des eingerahmten Bereichs dargestellt, der Name
des Urhebers ist ein Link auf die Webpräsenz, der Name der Lizenz ein Link
auf den Volltext der Lizenz.

Die Quellenangaben der gestalterischen Elemente findet tabellarisch im Footer der
Webseite statt, Alle dargestellten gestalterischen Elemente werden dort in der
Reihenfolge ihres Erscheinens im Quelltext der Seite aufgelistet. Die Tabelle
besteht aus den Spalten Nummer des Quellverweises, Name des Bildes,
Name des Urhebers und Name der Lizenz, wobei die Nummer des Quellverweises auf
das gestalterische in der Seite verlinkt, der Name des Bildes auf eine neue
Seite bestehend aus einer Abbildung des Bildes gemäß der obig genannten
Darstellungsform von Abbildungen, der Name des Autoren ein Link auf dessen
Webpräsenz und der Name der Lizenz ein Link auf den Volltext der Lizenz.

\subsubsection{Notwendigkeit von Quellenangaben}

Sofern notwendig kann die obig genannte Darstellung der Quellenangaben nicht
weiter beeinflusst werden als durch die Auswahl eines der genannten Typen. Liegt
jedoch eine Grafik vor, deren Lizenz weder die Namensnennung noch die Nennung
der Lizenz erfordert, wie beispielsweise CC0, oder der Autor der Seite selbst der
Urheber ist, soll auf die Quellenangabe verzichtet werden können, im Falle
einer Abbildung also nur der Name angezeigt werden und im Falle des
gestalterischen Elements der Eintrag in der Liste komplett weggelassen werden.

\subsubsection{Endgerätoptimierung}



\subsubsection{Datenhaltung}


%%% Local Variables:
%%% mode: latex
%%% TeX-master: "thesis"
%%% End:


\cleardoublepage
\section{Umsetzungsanalyse}
\label{sec:implementation-analysis}

Dieses Kapitel erläutert auf welche Art und Weise die im vorigen Kapitel geplanten Funktionen umgesetzt worden sind. Der softwaretechnische Unterbau der Entwicklungsumgebung setzt auf aktuelle Webtechnologien auf (siehe \ref{sec:req-web-application}~\nameref{sec:req-web-application} für die Diskussion der Begründung) und teilt sich in zwei distinkte Codebasen für Server und Client.

\begin{description}
\item[Server: Ruby mit Sinatra] \hfill\\
  Die Aufgaben des Servers sollen sich konzeptionell möglichst auf die Auslieferung und Speicherung von Daten beschränken. Die Interaktion findet dabei primär über eine REST-artige JSON Schnittstelle statt, serverseitig gerendert werden lediglich die Projekte der Schüler.
\item[Client: Typescript mit Angular 2] \hfill\\
  Aufgrund des hohen Grades an Interaktivität bietet sich eine rein clientseitige Visualisierung an, die weitestgehend auf Roundtrips zum Server verzichtet. Außer für den Zugriff auf serverseitige Resourcen (Datenbank, gespeicherte Ressourcen, gerenderte Seiten) werden alle Operationen im Browser ausgeführt.
\end{description}

Die grundsätzliche Struktur eines esqulino-Projektes wird in Diagramm \ref{uml:class-diagram-core-entities} ersichtlich. Diese Darstellung visualisert nicht die konkrete Implementierung des Servers oder des Clients, sondern illustriert die grundlegenden beteiligten Datenstrukturen. Jede dieser Entitäten, also sowohl Projekte als auch ihre Ressourcen, enthält eine eigene Versionsangabe. Dadurch kann auf jede Veränderung an dieser Struktur explizit eingegangen werden (\ref{sec:implementation-migration}~\nameref{sec:implementation-migration}), aktuell laden sowohl Server als auch Client nur Ressourcen deren Version exakt passt.

Jede Ressource (\lstinline{ProjectResource}) verfügt über eine interne ID sowie einen sprechenden Namen. In der aktuellen Version von esqulino handelt es sich bei dieser ID um eine GUID, sollte also weltweit einzigartig sein. Theoretisch wäre es dadurch denkbar diese Ressourcen auch zwischen Projekten zu kopieren bzw. zu Teilen. Intern werden Referenzen auf Ressourcen immer anhand der ID referenziert. Eine Umbenennung von Ressourcen durch den Benutzer hat daher keine Auswirkungen auf etwaige Referenzen an anderer Stelle.

\begin{diagram}[p]
  \begin{tikzpicture}
    \begin{interface}[text width=7cm]{ApiVersionable}{-4, 0}
      \attribute{apiVersion : string}
    \end{interface}
    
    \begin{class}[text width=7cm]{Project}{-8, -4}
      \implement{ApiVersionable}
      
      \attribute{id : string}
      \attribute{name : string}
      \attribute{description : string}
      \attribute{indexPageId : string}
    \end{class}

    \begin{abstractclass}[text width=7cm]{ProjectResource}{0, -4}
      \implement{ApiVersionable}
      
      \attribute{id : string}
      \attribute{name : string}
    \end{abstractclass}

    \begin{class}[text width=7cm]{Page}{0, -8}      
      \attribute{body: BodyNode}
      \attribute{referencedQueries: QueryReference[]}
      \attribute{parameters: PageParameter[]}
    \end{class}

    \begin{class}[text width=7cm]{Query}{0, -12}
      \attribute{select : Select}
      \attribute{delete : Insert}
      \attribute{insert : Insert}
      \attribute{update : Update}
      \attribute{from   : From}
      \attribute{where  : Where}
    \end{class}

    %\association{Query}{parentProject}{1}{Project}{0..*}{queries}

    \draw[] (Page.west)  -| (Project.south);
    \draw[] (Query.west) -| (Project.south);

    % Query and page implement ProjectResource
    \draw[->] (Query.east) -- ++ (1,0) -- ($(ProjectResource.east)+(1,0)$) -- (ProjectResource.east);
    \draw[-] (Page.east) -- ++ (1,0);

    \node[xshift=0.3cm, yshift=-0.5cm] at (Project.south) {1};
    \node[xshift=-0.5cm, yshift=0.3cm] at (Query.west) {0..n};
    \node[xshift=-2.5cm, yshift=0.3cm] at (Query.west) {Pages};
    
    \node[xshift=-0.5cm, yshift=0.3cm] at (Page.west) {0..n};
    \node[xshift=-2.5cm, yshift=0.3cm] at (Page.west) {Queries};
  \end{tikzpicture}

  \caption{Abstrakte Übersicht über die Ressourcen eines Projektes}
  \label{uml:class-diagram-core-entities}
\end{diagram}

\subsection{Serverseitige Persistenz}
\label{sec:implementation-persistence}

Im einfachsten Fall ist die serverseitig persistierte Version einer Resource identisch mit dem Übertragungsformat. Dabei müssen möglicherweise sensible Informationen gefiltert werden, in der Datenstruktur der jeweiligen Implementierung sollte diese Felder daher als optional betrachtet werden. Sofern jedoch keine Filterung notwendig ist, können diese Dokumente unverändert bereitgestellt werden. Damit wäre der rein lesende Zugriff auf esqulino-Projekte mit

\subsection{Datenbanksystem}
\label{sec:implementation-database-system}

Die Wahl des konkreten Datenbanksystems hat einen unmittelbaren Einfluss auf nahezu alle Bereiche von esqulino. Im einzelnen handelt es sich dabei um die exakte Variante der SQL Syntax, auf die Rahmenbedingungen für den Betrieb der Entwicklungsumgebung und auch auf die Fortführung der Projekte mit externen Programmen.

Die in der Praxis häufig dominierenden Entscheidungskriterien der Skalierbarkeit, die Unterstützung unterschiedlichster Zugriffsrechte und auch die allgemeine Performance spielen nur eine sehr untergeordnete Rolle. Die zu erwartenden Datenbeständen sollten normalerweise im Bereich nur einiger Megabyte liegen und die in der Praxis möglicherweise einzige Unterscheidung von Zugriffsrechten wäre zwischen lesendem und schreibenden Zugriff. Für die Wahl des Datenbanksystems werden stattdessen die folgenden Kriterien gewählt und hinsichtlich ihrer Relevanz sortiert:

\begin{description}  
\item[Kostenlose Verfügbarkeit] \hfill \\
  Der Betrieb des Datenbanksystems soll nicht mit Lizenzkosten für Schulen, Lehrkräft, Lernende oder auch freiwillige Entwickler verbunden sein.
\item[Einfacher Betrieb] \hfill \\
  Zwar ist für den Einsatz von esqulino aufgrund des Browsers als Client schon die Nutzung eines Servers nötig, das Datenbanksystem sollte den Betrieb dennoch nicht mehr als unbedingt notwendig weiter verkomplizieren. Eine wesentliche Rolle spielt dabei die Platformunabhängigkeit: Das Datenbanksystem sollte, wie auch der esqulino Server, auf jedem gängigen Betriebssystem lauffähig sein.
\item[Einfache Backups] \hfill \\
  Die gewünschte Exportfunktion für Projekte macht es nötig, den gesamten Datenbestand vergleichsweise einfach exportieren und importieren zu können. Darüber hinaus sollte es auch für Lehrkräfte möglichst einfach sein mit allen Projekten zu einem anderen esqulino-Server umzuziehen.
\item[Tools zur Modellierung] \hfill \\
  Da diese Arbeit sich nicht mit der Datenmodellierung befasst, muss das entsprechende Datenbankschema extern erzeugt werden. Von einer guten Unterstützung für Modellierungsvorhaben profitiert dementsprechend indirekt auch esqulino.
\item[Tools zur Entwicklung von SQL-Abfragen] \hfill \\
  Sobald ein Entwickler an die Grenzen des SQL-Editor von esqulino stößt, soll es so einfach wie möglich sein die Abfragen in einem externen Editor zu schreiben und danach in Textform wieder an esqulino zu übergeben.
\end{description}

Das Kriterium der ``kostenlosen Verfügbarkeit'' ist dankenswerterweise sehr einfach zu erfüllen: Es existiert eine Vielzahl von praktisch eingesetzten quelloffenen Datenbanksystemen. Die Kriterien ``einfacher Betrieb'' und ``einfache Backups'' teilen die denkbaren Systeme recht eindeutig in zwei Lager: Eingebette Datenbanken lassen sich sehr einfach betreiben und sichern. Das starten eines weiteren SQL-Server-Prozesses ist bei dieser Betriebsart nicht nötig, der Im- oder Export des gesamten Datenbestandes erfordert nur das kopieren einer einzigen Datei.

Um den Betrieb folglich so einfach wie möglich zu halten, wurden für esqulino zunächst eingebettete Datenbanksysteme betrachtet. Aus der Masse an verfügbaren Systemen sticht das SQLite-System jedoch sehr weit hinaus: Der Quelltext ist gemeinfrei, der Betrieb sogar auf exotischen Systemen möglich und es existiert eine Fülle von verschiedensten Programmen für alle Betriebssysteme.

\unsure[inline]{Theoretisch ist die Menge an denkbaren Systemen fast unüberschaubar groß, praktisch sticht SQLite aus der Masse an Optionen heraus. Wie ausführlich muss ich das begründen?}

\subsection{Hinweise zum Client}

Der verwendete Typescript Compiler hat zum Zeitpunkt der Anfertigung dieser Arbeit einen bekannten Bug in der Codegenerierung \cite{ts-compiler-class-order-bug} um den wiederholt herumgearbeitet werden musste. Konkret äussert sich dieser Fehler, wenn die Definition der Oberklasse einer sich davon ableitenden Klasse erst im Nachhinein erfolgt (Listing \ref{lst:ts:class-order-bug}). In diesem Fall kommt es zu keiner Warnung durch den Compiler, sondern zu einem Laufzeitfehler im kompilierten Javascript-Code.

\lstinputlisting[language=JavaScript,caption=Falsche Reihenfolge der Klassendefinition, label=lst:ts:class-order-bug]{snippets/class-inheritance-order-bug.ts}

\subsection{Administrationszugang}
\label{sec:implementation-administration}

Momentan aufgrund von Zeitmangel nur in Form der Kommandozeilenschnittstelle.

\subsubsection{Schema-Migrationen}
\label{sec:implementation-migration}

%%% Local Variables:
%%% mode: latex
%%% TeX-master: "thesis"
%%% End:


\cleardoublepage
\section{Fazit}
\label{sec:conclusion}

Ein Blick in den Anhang zeigt, dass der Prototyp dem eingangs erwähnten Ziel\footnote{"`Mit dem Blattwerkzeug lassen sich gestützt durch \textit{Drag \& Drop-Editoren} für beliebige \texttt{SQLite}-Datenbanken \textit{Abfragen formulieren} und \textit{Oberflächen entwickeln}"', siehe Kapitel \fullref{sec:introduction}} durchaus gerecht wird: Zu praktisch beliebigen SQLite-Datenbanken können Abfragen und Oberflächen entwickelt werden.

Etwas detallierter lässt sich der Grad des Erfolges einer prototypischen Implementierung vor allem am finalen Entwicklungsstand festmachen: Welcher prozentuale Anteil der angestrebten Funktionalität wurde erreicht? Daher muss sich der "`fertige Prototyp"' an den in Kapitel \fullref{sec:principles} formulierten Zielen messen lassen. Auch einige spätere Kapitel, insbesondere \fullref{sec:sql-subset}, lassen sich als ein recht detaillierter Anforderungskatalog verstehen.

Dieses vermeintlich objektive Kriterium berücksichtigt aber nur einen Teil der für \idename{} zu relevanten Ziele. Da auf dem prototypischen Stand weiter aufgebaut werden soll, ist eine weitere Frage von Bedeutung: Kann man auf diesem Stand die Weiterentwicklung fortführen?

\subsection{Erreichte Ziele}

Der aktuelle Stand der Implementierung ist vor allem als ein Durchstich zu verstehen: Auch wenn in fast jedem Teilbereich noch Funktionaliät fehlt, kann das Zusammenspiel dieser Systeme schon gut erprobt werden. Insbesondere bei der Verbindung der Datenbank mit der Oberfläche, sowohl lesend als auch schreibend, haben sich im Laufe der konkreten Implementierung noch einige unerwartete Stolpersteine aufgetan. Das zugrundeliegende Fundament, also die internen und textuellen Darstellungen von \texttt{SQL} und \texttt{HTML}, sind nun aber stabil und darüber hinaus auch mächtiger, als der mit dem Drag \& Drop-Editor editierbare Stand.

Das Grundprinzip "`\textbf{Semantik vor Syntax}"' wurde erreicht. Der Drag \& Drop-Editor schließt Syntaxfehler kategorisch aus, sofern diese im kompilierten Quelltext doch auftreten wäre das eindeutig ein Fehler in der Codegenerierung, nicht jedoch des Entwicklers. Bei Fehlermeldungen für erkannte Fehlersituationen gibt es aber noch Verbesserungspotenzial: Aktuell werden fehlerhafte Eingaben einfach zugelassen und dann erst im Nachhinein als Fehler markiert. An dieser Stelle wäre das im Prinzip angesprochene "`kontinuierliche Feedback"' vermutlich am besten mit einer eingebauten, kontextsensitiven Hilfe unterstützen. Diese könnte auf Fehler mit einer kurzen Erläuterung reagieren und dabei demonstrieren, wied die richtige Vorgehensweise wäre.

Ob die erstellbaren Seiten durch "`\textbf{praktisch vorzeigbare Ergebnisse motivieren}"', kann nur ein praktischer Test mit der Zielgruppe zeigen. Ein Fortschritt gegenüber der nicht sonderlich hohen Hürde "`besser sein als Texteingaben in einer \texttt{SQL}-\texttt{IDE}"' ist aber durchaus zu erwarten.

Die "`\textbf{einfache Inbetriebnahme}"' ist eine Frage der Perspektive. Aus Sicht eines Schülers ist der einfache Aufruf einer \texttt{URL} in der Tat einfach, bisher hat \idename{} auch gut unter allen ad-hoc probierten Kombinationen aus Betriebssystemen (Windows, Mac\-OS, Linux) und Browsern (Firefox, Chrome, Edge) gut funktioniert. Für Lehrkräfte wird aktuell eine virtuelle Maschine bereitgestellt, der Umgang mit dieser ist aufgrund des fehlenden Webinterfaces noch relativ unbequem.

Eine Notwendigkeit des Wechsels auf eine "`normale"' Desktopanwendung hat sich zu keinem Zeitpunkt ergeben. Das technische Fundament aus Ruby mit Sinatra und Typescript mit Angular 2 hat sich ebenso bewährt, wie die Entscheidung, den größten Teil der Logik im Client zu belassen. Der Betrieb von \idename{} fühlt sich nach der initialen Ladezeit sehr flüssig an. Die Möglichkeit der Entwicklung von Unit- und End-to-End-Tests fügt sich gut in den Entwicklungszyklus ein.

\subsection{Nicht erreichte Ziele}

Das Ziel einer "`\textbf{schrittweise komplexeren Benutzeroberfläche}"' wurde zunächst hinten angestellt, da es mit einem ausgearbeiteten Konzept einhergehen sollte. Kapitel \fullref{sec:sql-subset-ranks} untersucht zwar die möglichen Einschränkungen für \texttt{SQL}, geht aber nicht auf \texttt{HTML} oder die möglichen Auswirkungen auf das Zusammenspiel beider Systeme ein. Mit dem aktuell noch überschaubaren Funktionsumfang ist der akute Bedarf nach diesem spezifischen Grundprinzip allerdings auch noch nicht gegeben.

Auch das Ziel der "`\textbf{Fortführung der entwickelten Projekte}"' mit externen Programmen wurde für den Prototypen hinten angestellt. Und zumindest die Unterstützung der Quelltext-Editoren sollte noch implementiert werden, bevor der Bedarf für einen Export überhaupt abgeschätzt werden kann.

Der Umfang der tatsächlich implementierten \texttt{SQL}-Funktionen ist noch nicht abgeschlossen. Zum jetzigen Zeitpunkt fehlen neben der Unterstützung der \texttt{AS}-Direktive im Editor (der Syntaxbaum unterstützt die Benennung schon) vor allem Funktionen und Gruppierungen. Dieser Umstand schränkt die umsetzbaren Projekte empfindlich ein: Stünden diese Funktionen schon zur Verfügung könnten zum Beispiel auch Webseiten für Sportvereine, inklusive der dynamischen Erzeugung von Tabellen, mit \idename{} umgesetzt werden.

Der Einsatz des Prototypen im Unterricht ist aktuell vor allem aufgrund der rudimentären Benutzerverwaltung nur schwer vorstellbar. Noch ist noch keine Registrierung von Benutzern implementiert, Projekte können noch nicht über die Webseite kopiert werden. Die Implementierung einer Benutzerverwaltung ist allerdings eine vor allem handwerkliche Aufgabe und wurde daher im Rahmen dieser Thesis nicht vorangetrieben.

Aber auch der Betrieb für Lehrkräfte gestaltet sich deutlich komplizierter als notwendig: Die Notwendigkeit einer eigenen (Sub-)Domain ist keine triviale Hürde. Die Annahme "`man wird doch wohl mal auf dem Nameserver der Schuldomain einen Eintrag für \idename{} anlegen können"' hat sich zwar nicht als haltlos erwiesen, behindert die initiale Inbetriebnahme aber merklich.

\subsection{Weiterentwicklung}

Eine wesentliche Frage bei der Weiterentwicklung von \idename{} ist die Frage nach dem Zeitpunkt, zu dem man die eigentliche Zielgruppe (sowie deren Lehrer) in die Entwicklung mit einbezieht. Oder anders ausgesdrückt: Es stellt sich die Frage nach dem minimalen Satz an implementierten Funktionen, mit denen man sinnvoll Feedback bei Lehrern und Schülern einholen kann.

\begin{description}
\item[SQL: Unterstützung von Funktionen] \hfill\\

\item[SQL: Unterstützung von \texttt{GROUP BY}] \hfill\\

\item[SQL: Schemaeditor] \hfill\\
  
\end{description}



%%% Local Variables:
%%% mode: latex
%%% TeX-master: "thesis"
%%% End:


\appendix{}

\cleardoublepage
\section{Projektbeispiele}
\label{sec:project-examples}

Dieses Kapitel umfasst einige Ideen für Projekte, die sich gut mit \textbf{\idename} umsetzen lassen. Diese Auflistung ist natürlich nicht vollständig, sie umfasst vielmehr jene Projekte die gewissermaßen als ``Proof-of-Concept'' im Rahmen der Entwicklung entstanden sind.

\missing[inline]{Infobox zu jedem Beispiel}
\missing[inline]{Datenbankschema zu jedem Beispiel}

\subsection{Interaktive Geschichten}

Im englischsprachigen Raum hat sich für diese Art von Erzählung der Terminus ``Choose Your Own Adventure'' durchgesetzt. In Deutschland exisitiert kein feststehender Begriff, stattdessen werden häufig exemplarische Buchreihen wie ``Insel der tausend Gefahren'' stellvertretend für das Genre herangezogen. Für all jene, die auch mit diesen Begriffen nichts anfangen können, illustriert der folgende Abschnitt, wie eine solche Geschichte funktioniert:

\missing[inline]{Beispielgeschichte}

\subsection{Historische Personen und Ereignisse}

Welche historischen Ereignisse hat eigentlich Walt Disney erlebt, als er zwischen 20 und 30 Jahre alt war? Und welche historischen Persönlichkeiten waren am Leben, als der Buchdruck erfunden worden ist?

Dank Quellen wie WikiData ist es mittlerweile vergleichsweise einfach, große Datenmengen automatisiert zu extrahieren. Zu diesem Projekt gehört daher auch ein kleines Skript, um die Datenbank mit Werten zu füllen.

\subsection{Sehr einfacher Blog mit Kommentaren}

%%% Local Variables:
%%% mode: latex
%%% TeX-master: "thesis"
%%% End:


\cleardoublepage
\section{Technische Dokumentation der Oberfläche}
\label{app:documentation-ui}

%%% Local Variables:
%%% mode: latex
%%% TeX-master: "thesis"
%%% End:


\cleardoublepage
\section{Eidesstattliche Erklärung}

Ich erkläre hiermit an Eides statt, dass ich die vorliegende Arbeit
selbstständig und ohne Benutzung anderer als der angegebenen Hilfsmittel
angefertigt habe; die aus fremden Quellen direkt oder indirekt übernommenen
Gedanken sind als solche kenntlich gemacht. Die Arbeit wurde bisher in ähnlicher
Form keiner anderen Prüfungskommision vorgelegt und auch nicht veröffentlicht.

\bigskip
\bigskip
\bigskip
\bigskip
	
\begin{multicols}{2}
  \raggedright
  
  
  \raggedleft
  Ole Erik Werner Just
\end{multicols}
%%% Local Variables:
%%% mode: latex
%%% TeX-master: "thesis"
%%% End:


\listoftodos[Notes]

\printbibliography

\end{document}

%%% Local Variables:
%%% mode: latex
%%% TeX-master: t
%%% End:
