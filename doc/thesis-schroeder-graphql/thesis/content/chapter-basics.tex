%! Author = yannux
%! Date = 24.08.20

%************************************************
% Status Quo
%************************************************
\chapter{Grundlagen}
\label{sec:basics}
Das wesentliche Motiv bei der Nutzung des Internets ist die In\-for\-ma\-tions\-auf\-nah\-me~\cite{statista-1}\cite{ard-zdf}.
Informationen werden auf unzähligen Webapplikationen bereitgestellt, die jeder mit einem Internetzugang einsehen kann,
solange der Zugriff auf die Informationen nicht sonderlich geschützt wird.

Um persistente und sensible Daten gesichert und nicht für Jedermann zugreifbar lagern zu können, werden sie \emph{serverseitig} gehalten.
Möchte man diese Daten zusätzlich filtern, sortieren oder mehrere Datensätze miteinander verknüpfen, wird eine Datenbank benötigt.
Eine Datei, in der die Daten abgelegt werden, wäre auch eine Option, allerdings müsste man alle Methoden zum Filtern, Sortieren und Verknüpfen
selber implementieren.

Die in einer Datenbank gespeicherten Informationen sind also aus Nutzersicht nur über eine Anfrage an den Sever abrufbar.
Somit ist der Server das Bindeglied zwischen einem Client und der Datenbank und kümmert sich um Aufgaben wie Authentifizierung des Nutzers
und Überprüfung der Authorisierung bezüglich der angefragten Daten, aber auch um die Zusammensetzung und Ausführung von Datenbankabfragen.
Daraus geht hervor, dass ein Client nur begrenzten Zugriff bekommt, da die Ausführung von vordefinierten Funktionen, die Anfragen an die Datenbank beinhalten,
lediglich angefragt werden kann. Werden die vordefinierten Funktionen den Bedarf an Informationen nicht gerecht, müssen neue Funktionen entwickelt
oder mithilfe von mehreren Anfragen die Daten zusammengesammelt werden.

Dieser Entwicklungsaufwand könnte verringert werden, indem der Client mehr Flexibilität, Verantwortung und Effizienz besitzen würde,
z.B. durch eine direkte Anbindung an die Datenbank.
Er könnte exakt die benötigten Daten, mit nur einer Anfrage direkt und effizient aus der Datenbank auslesen.
Jedoch würde dieser Ansatz viele Gefahren mit sich bringen. Ein Client der direkten Datenbankzugriff erlangt,
könnte unerwünschte Transaktionen in der Datenbank ausführen, wodurch der erwartete Datenbestand geändert,
Einträge gar gelöscht oder sensible Daten anderer Nutzer abgefragt werden könnten. Also sollten Zugriffsbeschränkungen erteilt werden, die
auf der Datenbankschicht realisiert werden, da clientseitiger Code nach Belieben vom Nutzer eingesehen und verändert werden kann.
Hinzu kommen weitere Herausforderungen, wenn die Verbindung zur Datenbank veröffentlicht wird,
wie zum Beispiel das Schützen vor zu exzessiver Nutzung oder das Ausnutzen von bekannten Sicherheitslücken bei nicht aktuellsten Versionen~\cite{postgresql-security}.
Alles in allem ist das ein Verfahren, von dem dringend abgeraten wird, da es in den wenigsten Fällen nutzbringend und sicher gehandhabt werden kann~\cite{client-to-database}.

Im Folgenden werden diese Probleme anhand von grundlegenden Inhalten, die aktuell Gegenstand des von Marcus Riemers entwickelten Systems~\cite{riemer2016} sind,
wieder aufgegriffen und bei Erläuterungen bzw. Code-Beispielen als bekannt vorausgesetzt.
Dazu gehören die Unterkapitel \fullref{sec:basics:restapi}, \fullref{sec:basics:typescript}, \fullref{sec:basics:jsonschema} und \fullref{sec:basics:postgres}.
Diese müssen für die Schaffung und Umsetzung von Verbesserungen grundlegend verstanden werden.
Bei dem Kapitel \fullref{sec:basics:graphql} handelt es sich um eine Abfragesprache und Laufzeitumgebung,
für die im Laufe der Arbeit evaluiert wird, ob sie gewinnbringend in das bestehende System migriert werden und Teile ersetzen kann.

%Nicht spezifisch auf Blattwerzeuge werden!! Kleines Beispiel ist erlaubt (Begriff Blattwerzeug wird nicht erwähnt).
\section{REST API}
\label{sec:basics:restapi}
Der Begriff Representational State Transfer (abgekürzt REST, seltener auch ReST) wurde erstmalig in der Dissertation "Architectural Styles and the Design of
Network-based Software Architectures" von Roy Thomas Fielding im Jahr 2000 geprägt~\cite{fielding-dissertation}.
Er beschreibt REST als einen Architekturstil für verteilte Systeme, welcher in eine einheitliche Schnittstelle für Kommunikation mündet.
Dieser Architekturstil oder auch Programmierparadigma wird durch verschiedene Software-Engineering-Prinzipien und Beschränkungen definiert.
Im Folgenden werden die Prinzipien von REST näher erläutert.

\subsection{Client-Server-Modell}
Der Ausgangspunkt des Client-Server-Modells ist eine strikte Trennung der Benutzeroberfläche von der Datenhaltung/-verwertung.
Das bedeutet wiederum, dass kein HTML, CSS und Javascript vom Server an den Client geschickt wird, sondern ausschließlich Datensätze
meist in Form von XML oder JSON, die clientseitig in die Benutzeroberfläche eingebaut werden.
Dadurch verbessert sich die Portabilität der Benutzeroberfläche in Bezug auf die Anbindung an verschiedene Datenhaltungs/-verwertungs-Systeme,
also die Wiederverwendbarkeit und die Skalierbarkeit aufgrund der Vereinfachung der Serverkomponenten.

\subsection{Zustandslosigkeit}
Zustandslosigkeit ist eine Beschränkung in Bezug auf die Kommunikation zwischen Server und Client.
Anfragen vom Client müssen alle Informationen beinhalten um diese interpretieren zu können.
Insbesondere werden Anfragen ohne Bezug zu früheren Anfragen behandelt und keine Sitzungsinformationen -
wie Authentifizierungs- und Authorisierungsinformationen - ausgetauscht bzw. verwaltet.
Diese befinden sich ausschließlich auf dem Client und müssen bei Anforderung von geschützten Daten der Anfrage beigefügt werden.

Vorteile aus dieser Beschränkung sind, dass Anfragen unabhängig voneinander betrachtet werden können
und somit z.B. von mehreren Maschinen parallel ausgeführt werden können, da jede Anfrage für sich eine vollständige Anforderung an den Server beschreibt.
Zudem kann einfacher auf den Misserfolg einer Anfrage reagiert werden, als auf eine erfolglose Kette von zusammenhängenden Anfragen
und es ist nicht vonnöten Zwischenzustände bzw. Status zu speichern, welche die Ressourcenauslastung erhöhen würde.
Dies kann jedoch zu einer verringerten Netzwerkleistung führen aufgrund von Zusatzinformationen,
die sich bei mehreren verschiedenen Anfragen wiederholen und erneut mit gesendet werden müssen.

\subsection{Cache}
In Hinblick auf das Verbessern der Netzwerkleistung wurde ein Cache als Einschränkung hinzugefügt.
Diese Einschränkung setzt voraus, dass Daten aus einer Antwort vom Server implizit oder explizit als cachefähig oder nicht cachefähig gekennzeichnet werden.
Werden Daten in einer Antwort auf eine Anfrage als cachefähig gekennzeichnet, kann der Client diese Information speichern und erhält das Recht diese
bei einer späteren gleichwertigen Anfrage wiederzuverwenden.
Somit können Anfragen effizienter behandelt bzw. ganz durch eine direkt aus dem Cache geladene Antwort ersetzt werden.

\subsection{Einheitliche Schnittstelle}
\label{sec:basics:restapi:interface}
Ein zentrales Merkmal von REST ist die einheitliche und vom Dienst entkoppelte Schnittstelle.
Auf jede Ressource muss über einen einheitlichen Satz an URLs, hinter denen sich Transaktionen zum Erstellen, Lesen, Aktualisieren
und zum Löschen (CRUD) verbergen, zugegriffen werden können.

Durch eine einheitliche Komponentenschnittstelle wird die Sichtbarkeit der einzelnen Interaktionen erhöht.
Dies bedeutet, dass es für jede Ressource eine Menge fest definierter Interaktionen gibt, die sich in ihrer Struktur nur durch den Namen der
Ressource und ihre Fremdbeziehungen unterscheiden.

\begin{table}[h]
    \begin{tabular}{|p{0.14\textwidth}|p{0.08\textwidth}|p{0.12\textwidth}|p{0.2\textwidth}|p{0.32\textwidth}|}
        \hline
        \textbf{CRUD\newline  Operation} & \textbf{SQL} & \textbf{HTTP} & \textbf{URL} & \textbf{Bedeutung} \\ \hline
        Create & \inlinec{INSERT} & \inlinec{POST} & \inlinec{/projects} & Erstellen eines Projekts \\ \hline
        Read & \inlinec{SELECT} & \inlinec{GET} & \inlinec{/projects} & Abrufen aller Projekte \\ \hline
        Read & \inlinec{SELECT} & \inlinec{GET} & \inlinec{/projects/:id} & Abrufen eines Projekts \\ \hline
        Update & \inlinec{UPDATE} & \inlinec{PATCH/PUT} & \inlinec{/projects/:id} & Aktualisieren eines Projekts \\ \hline
        Delete & \inlinec{DELETE} &\inlinec{DELETE} & \inlinec{/projects/:id} & Löschen eines Projekts \\ \hline
        Read & \inlinec{SELECT} &\inlinec{GET} & \inlinec{/projects/:id/- user} & Abrufen des Nutzer eines Projekts \\ \hline
    \end{tabular}
    \vspace{5pt}
    \caption{Einheitliche Schnittstellen}
    \label{tbl:basics:crud}
\end{table}

Das hat zur Folge, dass anwendungsspezifische Daten in einer standardisierten Form übertragen werden müssen,
wodurch die Effizienz der Datenübertragung Mängel aufwerfen kann.
Insbesondere treten im Kontext der Arbeit zwei für REST bekannte Mängel auf, die hier näher erläutert werden.

\begin{description}
	\item[Overfetching\label{rest:overfetching}] \ \\
	Aufgrund der einheitliche Komponentenschnittstelle \ref{sec:basics:restapi:interface} gibt es für Projekte genau eine Route die alle Attribute zu einem bestimmten
	Projekt ungekürzt liefert. Wird nur ein Teil der Attribute benötigt sind weitere Angaben nutzlos und damit ineffizient.
	Dennoch werden sie von der API mitgeliefert.
	\item[Underfetching\label{rest:underfetching}] \ \\
	Ein weiteres Problem ist das mehrere Anfragen für Daten die in Beziehung zueinander stehen benötigt werden. Es wird oft vom N+1 Query Problem gesprochen, welches sich in diesem Kontext auf Anfragen an die Web API bezieht und nicht auf Datenbankabfragen.
	In \fullref{req:neg:rest-data-fetching} muss für jeden Nutzer eines Projektes, erfragt werden welche Freunde dieser hat. Vorausgesetzt wird, das eine entsprechende Route für die Abfrage nach Freunden eines Nutzers existiert.
	Sind einem Projekt N Nutzer zugeordnet muss für diese Information die dritte Abfrage N mal abgeschickt werden. Da zusätzlich noch die Liste von mit dem Projekt in Beziehung stehenden Nutzern erfragt wird spricht man von N+1 Abfragen.
\end{description}

\begin{figure}[h!]
	\centering
	\includegraphics[width=\linewidth]{snippets/rest-data-fetching.pdf}
	\caption{Abfragen von Projektdaten inklusive aller Nutzer die das Projekt einsehen können}
	\label{basics:rest:data-fetching}
\end{figure}

\subsection{Weitere Einschränkungen}
Zusätzlichgibt es noch die Einschränkung \emph{Layered System}, welches das Prinzip eines hierachisch in Schichten aufgebauten Systems
beschreibt und eine optionale Einschränkung \emph{Code-On-Demand}, welche die Client-Funktionalität durch Herunterladen und Ausführen von Code in
Form von Applets oder Skripten erweitert. Diese werden im Rahmen der Arbeit nicht genutzt und deshalb nur am Rande erwähnt.
%Als Entwickler einer Anwendung stellt Modularität ein von Beginn an bedachtes Kalkül dar.

\section{GraphQL}
\label{sec:basics:graphql}
%Nutzung einer statischen Schnittstelle zu einer sich ändernden Implementierung
Bei GraphQL handelt es sich um eine Abfragesprache für APIs und eine Laufzeitumgebnung,
zum Ausführen dieser Abfragen und Wiedergeben von Daten unter Verwendung eines von für die Daten definierten Typensystems.
Es ist an keinerlei Datenbanksysteme gebunden und lässt sich gut mit vorhandenen Code und Daten verbinden.

Ein GraphQL Service entsteht durch das Definieren eines Typschemas, vergleichbar mit Datenbanktabellen. Zu jedem Feld eines Typs lassen sich - genauso wie bei Datenbanken -
Datentypen und Restriktionen wie \emph{NOT NULL} definieren.

\begin{lstlisting}[language=Javascript,float=h!,caption={Project Typdefinition}, label={fig:basics:graphql:1}]
type Project {
    id: ID!
    public: Boolean
}
\end{lstlisting}

Das aufgeführte Codefragment~\fullref{fig:basics:graphql:1} definiert einen Typen \emph{Project} mit zwei Feldern, welcher ein Programmierprojekt darstellen soll.

\begin{itemize}
	\setlength\itemsep{-1em}
    \item \emph{id}: hat den von GraphQL vorgegebenen Typen ID, der als eindeutiger String gewertet wird und nicht dazu gedacht ist vom Menschen "lesbar" zu sein.
    Zusätzlich wurde mit "!" festgelegt, dass dieses Feld nicht Null sein darf.
    \item \emph{public}: besitzt den Typen Boolean, der den Wert Null annehmen kann.
    Es gibt an, ob das Projekt bereits veröffentlicht und für jeden zugreifbar gemacht wurde.
\end{itemize}

Im Gegensatz zu einem Datenbank-Schema ermöglicht das Typsystem von GraphQL zu jedem Typen Argumente zu definieren,
die wiederum einer Funktion (Resolver) übergeben werden können,
die aufgerufen wird, wenn das Feld im Kontext einer Query aufgelöst werden soll~\cite{graphql-resolver}.
Erweitern wir für ein Beispiel unseren Typen um ein Feld:

\begin{lstlisting}[language=Javascript,float=h!,caption={Erweiterung der Typdefinition von Project und Einführung eines Enums mit Ländercodes}, label={fig:basics:graphql:2}]
enum LanguageEnum {
    DE
    EN
    FR
}

type Project {
    id: ID!
    public: Boolean
    name(language: LanguageEnum = DE): String!
}
\end{lstlisting}

\begin{itemize}
	\setlength\itemsep{-1em}
    \item \emph{name}: besitzt den Typen String, der ebenfalls nicht den Wert Null annehmen kann. Zusätzlich wurde dem Feld ein Argument "language" zugeteilt,
    welches den selbst definierten Datentypen LanguageEnum besitzt und de Defaultwert DE setzt.
\end{itemize}

Zusätzlich wird eine Funktion benötigt, die das Argument \emph{language} verarbeiten kann und dementsprechend den Rückgabewert formuliert.
Solche Funktionen nennen sich in der Welt von GraphQL Resolver.
Aufgabe dieses Resolvers ist es durch Aufruf der in Zeile 2 \ref{fig:basics:graphql:3} als Pseudocode aufgeführten Funktion \emph{translate},
den Namen eines Projektes in die übergebene Sprache zu übersetzen.

\begin{lstlisting}[language=Javascript,float=h!,caption={Resolver des Feldes \emph{name}}, label={fig:basics:graphql:3}]
name(obj, args, context, info) {
    return translate(obj,args['language'])
}
\end{lstlisting}

Damit nun ein Datentyp abgefragt werden kann, müssen Queries zu den Datentypen definiert werden.

\begin{lstlisting}[language=Javascript,float=h!,caption={GraphQL Query Typdefinition}, label={fig:basics:graphql:4}]
type Query {
    projects: [Project!]!
}
\end{lstlisting}

Der Query Typ gehört zum Typsystem von GraphQL. Er beinhaltet alle für das Schema definierten Queries, wie in diesem Fall~\ref{fig:basics:graphql:4} angegeben, eine Query mit dem Bezeichner \emph{projects}, die
ein Array vom Typ \emph{Project} als Rückgabewert erwartet. Zusätzlich wurde angegeben, dass Projekte innerhalb des Arrays und das Array selber nicht Null sein dürfen.
Es wird also mindestens eine leeres Array erwartet, aber keinesfalls Null oder ein Array, das mit Null-Werten gefüllt ist~\cite{graphql}.

Um Queries im GraphQL Schema zugreifbar zu machen werden diese auf oberster Ebene des Schema definiert. Diese sind dann auch die Einstiegspunkte für alle Anfragen die der GraphQL Service behandeln soll.

\begin{lstlisting}[language=Javascript,float=h!,caption={GraphQL Schema}, label={fig:basics:graphql:7}]
schema {
  query: Query
  mutation: Mutation
}
\end{lstlisting}

Jetzt kann dem Client die Freiheit gewährt werden eigene Abfragen für genau den Datensatz der benötigt wird zu formulieren.
Zudem lässt sich anhand der gestellten Abfrage die Struktur der erhaltenen Antwort festlegen. Dies könnte wie folgt aussehen:

\begin{lstlisting}[language=Javascript,float=h!,caption={GraphQL Query mit dem Bezeichner Projects}, label={fig:basics:graphql:5}]
query Projects{
    projects {
        id
        name(language: EN)
    }
}
\end{lstlisting}

Der aufgeführte Code~\fullref{fig:basics:graphql:5} verkörpert eine GraphQL Query mit dem Bezeichner \emph{Projects},
die für alle vorhandenen Projekte die Felder \emph{id} und \emph{name} zurück gibt. 
Aus dieser Query lässt sich nicht der Rückgabewert von projects erschließen.
Diese Information erhält man als Entwickler lediglich aus den Query Definitionen.
Bei abschicken der Query wird neben der Query auch der 
Bezeichner als \emph{operationName} mit gegeben.
Nachdem eine GraphQL Query gegen das Typsystem validiert wurde, wird sie von dem GraphQL Server ausgeführt und
ein Ergebnis - typischerweise in Form von JSON - zurückgegeben, das die Form und Struktur der Anfrage spiegelt.

\begin{lstlisting}[language=Javascript,float=h!,caption={JSON Antwort auf die Projects Query}, label={fig:basics:graphql:6}]
{
  "projects": [
    {
      "id": "368b6ee9-2b1f-4661-a82f-ff7b62dc9251"
      "name": "Esqulino"
    },
    {
      "id": "b25c342e-f2b1-4a74-8124-a7a688911380"
      "name": "Trucklino"
    }
  ]
}
\end{lstlisting}




Dies könnte zum Beispiel der bereits definierte Query-Typ \emph{projects} sein.
Damit der GraphQL Server eine Anfrage an eine an den Server gebundene Datenbank schicken kann, wird die zum Query-Typen definierte
Resolver-Funktion ausgeführt~\cite{graphql-execution}. Innerhalb dieser Resolver-Funktion ist der Zugriff auf das Dateisystem festgelegt, sodass
neben Datenbanken sogar Dateien als Speichermedium genutzt werden könnten.

GraphQL bietet neben Queries zum Abfragen von Datensätzen auch Anfragen zum Speichern von Daten im gewählten Speichermedium. Solche Anfragen nennen sich Mutationen. Genau wie bei Resolvern eines Feldes wird bei einer Mutation Code hinzugefügt, der  für das Erstellen oder Ändern von Datensätzen zuständig ist. Technisch gesehen könnte jede Query auch so implementiert werden, dass diese das Speichern von Daten bewirkt. Es ist jedoch nützlich, eine Konvention festzulegen, dass alle Operationen, die Schreibvorgänge verursachen, explizit über eine Mutation gesendet werden sollten~\cite{graphql-mutations}.

\begin{lstlisting}[language=Javascript,float=h!,caption={GraphQL Mutation zum Erstellen eines Projektes}, label={fig:basics:graphql:7}]
type creationResponse {
  id: String
  errors: [String]
}

type Mutation {
  createProject(name: String!, public: Boolean):creationResponse
}

mutation CreateProject($name: String!, $public: Boolean) {
  createProject(name: $name, public: $public) {
    id
    errors
  }
}
\end{lstlisting}

\begin{itemize}
	\setlength\itemsep{-1em}
	\item \emph{Zeile 1-4}: Festlegung eines Typs mit den Feldern id und errors. Dieser Typ spiegelt den Rückgabewert einer Mutation zum Erstellen neuer Datensätze wieder. Wenn die Mutation erfolgreich war wird die id des neuen Datensatzen zurück gegeben, wenn Fehler aufgetreten sind, wird das errors Feld mit diesen gefüllt.
	\item \emph{Zeile 6-8}: Definiert eine Mutation wie bei den Queries~\ref{fig:basics:graphql:4}. Diese bekommt zwei Argumente \emph{name} und \emph{public} übergeben. Letzteres ist dabei optional.
	\item \emph{Zeile 10-15}: Festlegen einer Mutation mit dem Operations Namen \emph{CreateProject}, die das Argument \emph{name} als String zwingend erwartet und das optionale Argument \emph{public}.
	Diese Argumente werden dann der \emph{createProject} mutation übergeben.
	\emph{public} ist bei der Mutation zwar optional, wird aber bei Abfrage von Projektdaten zwingend erwartet~\ref{fig:basics:graphql:2}. Dementsprechend muss die Funktion zum Umsetzen der Mutation, bei nicht vorhandenem \emph{public} Argument, diesen Wert trotzdem setzen.
	Als Antwort auf die Mutation werden die Felder id und errors erwartet.
\end{itemize}










%Eine andere Möglichkeit, die bereits umrissen wurde, wären die in %Kapitel~\fullref{sec:basics:restapi} vorgestellten Inhalte einer REST API.
%Diese setzt voraus, dass der Zugriff auf persistent gespeicherte Daten %ausschließlich an den Server gekoppelt ist - oftmals in Form einer %Datenbank.
%Datenzugriffe, die vom Client angestoßen werden, werden über eine %festgelegte Menge an vordefinierten Transaktionen auf dem Server %bereitgestellt.
%Jede dieser unterschiedlichen Transaktionen kann über eine Anfrage an den %Server ausgelöst werden.

%Die in Tabelle~\fullref{tbl:basics:crud} aufgeführte Liste gibt zu jeder %Transaktion an, über welche Route sie aufgerufen werden kann.
%Je größer eine Anwendung wird, desto größer wird der Datenbestand und %damit die Menge an Transaktionen, die vordefiniert werden muss.
%Für jeden Datensatz, der an den Client ausgeliefert werden soll, muss also %solch ein Eintrag vorgenommen werden und die dazugehörige Transaktion %entwickelt werden.

%Nutzung einer statischen Schnittstelle zu einer sich ändernden Implementierung
%https://graphql.org/
%Serverseitigen Rendern arbeiten mit SQL
%Clientseitiges Rendern kein SQL möglich -> Begründung
%Graphql als Lösung für dieses Mismatch/Problem
%Alternative wäre REST API mit 1 Route pro Query

\section{Ausgewählte Details des Typescript Typsystems}
\label{sec:basics:typescript}
TypeScript ist ein typisiertes Superset von Javascript, das zu reinem Javascript kompiliert~\cite{typescript}.
Das heisst, es beinhaltet alle Funktionalitäten von Javascript und wurde darüber hinaus erweitert und verbessert~\cite{superset}.
Dazu gehört das in Typescript eingeführte Typsystem. Selbstverständlich besitzt Javascript ebenfalls Typen, doch kann eine Variable,
auf die ursprünglich eine \emph{number} zugewiesen wurde, auch als \emph{string} enden. Das kann schnell zu unbedachten Seiteneffekten führen.

Ein Typsystem ist eine Menge von Regeln, die jeder Variable, jedem Ausdruck, jeder Klasse,
jeder Funktion, jedem Objekt oder Modul im System einen Typ zuweist.
Diese Regeln werden zur  Kompilierungszeit (statische Typprüfung) oder zur Laufzeit (dynamische Typprüfung) geprüft,
um Fehler in einem Programm aufzudecken~\cite{typescript-typesystem-medium}.

Der Typescript-Compiler prüft zur Kompilierungszeit alle Variablen und Ausdrücke auf ihren Typen und entfernt anschließend alle Typinformationen
bei der Konvertierung zu Javascript Code~\cite{typescript-github-specification}.
Die im folgenden Beispiel deklarierte Funktion gibt die zweite Hälfte eines übergebenen \emph{strings} zurück.
Der erste Aufruf der Funktion führt zu einem Fehler beim Kompilieren. Es wird also direkt darauf hingewiesen, dass es sich um ein fehlerhaften Code handelt.

\begin{figure}[h]
    \lstinputlisting{snippets/basics/typescript-example.txt}
    \caption{Typescript Funktion mit typisiertem Parameter}
    \label{fig:basics:typescript:1}
\end{figure}

Nach der Kompilierung sind alle Typinformationen entfernt worden, wodurch erst durch einen fehlerhaften Aufruf
ein \emph{TypeError} auftritt.

\begin{figure}[h]
    \lstinputlisting{snippets/basics/typescript-example-compiled.txt}
    \caption{Zu Javascript kompilierte Funktion}
    \label{fig:basics:typescript:2}
\end{figure}

Nehmen wir an, wir möchten den in Kapitel GraphQL~\fullref{sec:basics:graphql} erstellten Typen \emph{Project} nutzen,
um eine Funktion zu schreiben, die einen neuen \emph{Project} Datensatz an den Server schickt. Um diesen Typen clientseitig nutzen zu können,
können wir ein äquivalentes Typescript Interface erstellen oder eines generieren lassen (dazu später mehr).

\begin{figure}[h]
    \lstinputlisting{snippets/basics/typescript-project-interface.txt}
    \caption{Typescript Project Interface}
    \label{fig:basics:typescript:3}
\end{figure}

Wollen wir jetzt einen neuen Datensatz an den Server schicken, können wir das Interface nutzen. Jedoch ist nur der Name des neuen Datensatzes bekannt,
die \emph{id} ist eine \emph{uuid} und wird serverseitig generiert. Also wollen wir die \emph{id} beim clientseitigen Erstellen außen vorlassen.
Dafür bietet Typescript unter einer Vielzahl von Werkzeugen, die allgemeine Typtransformationen ermöglichen~\cite{typescript-utility-types},
\emph{Omit<T,K>}, das alle Attribute von \emph{T} nimmt und anschließend \emph{K} aus den Attributen entfernt.

\begin{figure}[h]
    \lstinputlisting{snippets/basics/typescript-post-project.txt}
    \caption{Transformierter \emph{Project} Typ}
    \label{fig:basics:typescript:4}
\end{figure}

Der Typ \emph{PostProject} beinhaltet also alle Felder von \emph{Project}, allerdings ohne \emph{id}. Das Gegenstück zu \emph{Omit<T,K>} wäre \emph{Pick<T,K>},
welches aus dem Typ \emph{T} nur die Attribute \emph{K} nimmt. Mithilfe dieser Typen lässt sich eine typsichere Methode entwickeln,
um einen neuen Datensatz an den Server schicken zu können.

\begin{figure}[h]
    \lstinputlisting{snippets/basics/typescript-create-project-request.txt}
    \caption{Typen und Methode zum Abschicken eines \emph{Project}-Datensatzes}
    \label{fig:basics:typescript:5}
\end{figure}

Die Methode \inlinec{createProjectRecord} erwartet also ein \inlinec{Project} ohne \inlinec{id} als Parameter und gibt ein \inlinec{ProjectResponse} wieder.
Der Code im Methodenrumpf ist hierbei nur Pseudocode. Der Typ \inlinec{ProjectResponse} beinhaltet neben dem \inlinec{Project} auch ein \inlinec{error} Feld,
welches in dem Kontext angibt, ob ein neuer Datensatz auf dem Server erstellt werden konnte oder nicht.
Desweiteren gibt es noch \inlinec{Exclude<T,U>} wodurch sich von T diejenigen Typen ausschließen lassen,
die U zugeordnet werden können. Gäbe es mehrere \emph{"Response"}-Typen, ließe sich das Feld extrahieren,
über welches auf die Datensätze zugegriffen werden kann (siehe~\fullref{fig:basics:typescript:6})(dazu später mehr).

\begin{figure}[h]
    \lstinputlisting{snippets/basics/typescript-datakey.txt}
    \caption{Exclude zum Exkludieren von Schlüsseln}
    \label{fig:basics:typescript:6}
\end{figure}

% Definition eines Schemas mit Typescript Interfaces, Bsp. Project Tabelle
% pick operator / exclude operator

\section{JSON Schema}
\label{sec:basics:jsonschema}
JSON-Schema ist ein Vokabular, mit dem JSON-Dokumente  annotiert und validiert werden können~\cite{json-schema}.
Es wird zur Überprüfung genutzt, ob JSON Objekte die im JSON-Schema beschriebene Struktur einhalten.

Der Vorgänger von JSON-Schema war das XML-Schema.
XML-Schema erlaubte es, das allgemeine Format eines XML-Dokuments zu definieren,
d.h. welche Elemente erlaubt sind, die Anzahl und Reihenfolge ihres Auftretens, welchen Datentyp sie haben sollen usw.
Seit 2006 gibt es einen neuen Akteur auf dem Gebiet der Datenformate, JavaScript Object Notation (JSON).
Die JSON Daten sind viel kleiner als ihr XML-Gegenstück und ihre Instanzen sind gültige JavaScript-Objekte, was es interessant für Webentwickler macht, da sie beim Laden von
Informationen in asynchronen Webanwendungen über AJAX (Asynchronous JavaScript and XML) keinen separaten Konvertierungsschritt mehr benötigen~\cite{json-schema-xml}.

Nehmen wir an, wir möchten den in Kapitel GraphQL erstellten Typen \emph{Project}~\fullref{fig:basics:graphql:3} mit verschiedenen Attributen erweitern.
Eine JSON Instanz soll mindestens folgende Attribute beinhalten, wobei die Angabe, ob es sich bei dem Entwicklerum einen proudFather handelt, optional ist.

\begin{lstlisting}[language=Javascript,float=h!,caption={Ein Projekt als JSON Objekt}, label={fig:basics:jsonschema:1}]
{
  "id":"de0d91a7-61ae-49af-90d9-5a37dd883a01",
  "name":"Esqulino",
  "public": false,
  "createdAt":1452985200000,
  "developer": {
    "firstname":"Marcus",
    "proudFather":true
  }
}
\end{lstlisting}

Das passendes Schema dazu sieht folgendermaßen aus~\fullref{fig:basics:jsonschema:2}.

\begin{lstlisting}[language=Javascript,float=h!,caption={JSON Schema zu Projekt Objekt}, label={fig:basics:jsonschema:2}]
{
    "$schema": "http://json-schema.org/draft-07/schema#",
    "title": "project",
    "description": "A project from Esqulino",
    "type": "object",
    "properties": {
        "id": {
            "type": "integer"
        },
        "name": {
            "type": "string"
        },
        "public": {
            "type": "boolean"
        },
        "createdAt": {
            "description": "Date of creation in milliseconds",
            "type": "number"
        },
        "developer": {
            "description": "The developer of a project",
            "type": "object"
            "properties": {
                "firstname": {
                    "description": "The forename of the developer",
                    "type": "string"
                },
                "proudFather". {
                    "description": "Indicator if he is a father or not",
                    "type": "boolean"
                }
            },
            "required": [ "firstname" ]
        },
    },
    "required": [ "id", "name", "developer", "createdAt" ]
}
\end{lstlisting}

Neben den verwendeten Schlüsselwörtern gibt es noch eine Vielzahl weiterer, die es unter anderem erlauben Einschränkungen, Abhängigkeiten,
Muster in Form von Regulären Ausdrücken oder die maximale oder minimale Anzahl an zu einem Objekt gehörende Attribute festzulegen.
Die hier verwendeten Schlüsselwörter haben folgende Bedeutung:

\begin{itemize}
    \label{basics:jsonschema:items}
    \setlength\itemsep{-1em}
    \item \emph{2: \$schema}: besagt, dass dieses Schema nach einem bestimmten Entwurf des Standards geschrieben ist, in erster Linie zur Versionskontrolle.
    \item \emph{3: title/description}: haben nur beschreibenden Charakter.
    \item \emph{5: type}: Dieses Schlüsselwort für die Typüberprüfung definiert die erste Beschränkung für die JSON-Daten und in diesem Fall muss es sich um ein JSON-Objekt handeln.
    \item \emph{6: properties}: Beschreibt welche Attribute das Objekt haben darf.
    \item \emph{7-35}: Definierung der Attribute eines Projektes, wobei in Zeile 20-34 ein weiteres Objekt als Attribut definiert wird. Dieses besitzt die beiden Attribute firstname als String und proudFather als Boolean. Die Angabe von firstname wird bei einem developer Objekt zwingend erwartet, proudFather ist optional.
    \item \emph{36: required}: Da dieses Schlüsselwort ein Array von Strings beinhaltet, können bei Bedarf mehrere Attribute angeben werden, die erwartet werden.
\end{itemize}

Nehmen wir an ein Entwickler hat ein fehlerhaftes Projekt wie folgt erstellt~\fullref{fig:basics:jsonschema:2}.

\begin{lstlisting}[language=Javascript,float=h!,caption={Ein fehlerhaftes Projekt}, label={fig:basics:jsonschema:2}]
{
  "id":"de0d91a7-61ae-49af-90d9-5a37dd883a01",
  "public": false,
  "developer":{
    "firstname":"Michael",
    "professor": true
  },
  "createdAt":"1452985200000"
}
\end{lstlisting}

Es kommt bei der Validierung dieses Objektes zu folgenden Verstößen:

\begin{itemize}
    \setlength\itemsep{-1em}
    \item \emph{name}: ist im \emph{required} Array angegeben und muss somit vorhanden sein.
    \item \emph{createdAt}: Es wurde ein falscher Datentyp angegeben, \emph{string} statt \emph{number}.
    \item \emph{professor}: Dieses Attribut ist nicht im \emph{properties} Objekt angegeben und dadurch fehl am Platz.
\end{itemize}

Die händische Erstellung solcher JSON-Schema kann bei einer Vielzahl von Typen schnell lästig werden.
Um dem Problem entgegen zu wirken lassen sich aus Typescript Interfaces passende JSON-Schema Dateien generieren.
Der Vorteil daran ist, dass sich clientseitig definierte Datentypen durch die Generierung serverseitig validieren lassen;
denn für die meisten gängigen Programmiersprachen sind JSON-Schema Validatoren entwickelt worden.
Somit ist es unabhängig welche Programmiersprache der Server nutzt~\cite{json-schema-implementations}.

% kurz halten
% Äquivalenz: typescript Interface und JSON.schema

\section{Postgres jsonb und hstore Typen}
\label{sec:basics:postgres}
Das PostgreSQL Datenbanksystem kennt über den SQL-Standard hinaus die Datentypen \emph{hstore} und \emph{jsonb} zur Speicherung von JSON Strukturen oder assoziativen Arrays, die üblicherweise in NoSQL-Systemen gespeichert werden.
Diese beiden Typen werden im Kontext der Arbeit für Objekte genutzt, die sich nur mit sehr großem Aufwand und zukünftigen Migrationen in ein Datenbankschema gießen lassen. Einer dieser Typen ist ein Objekt, das ein multilingualen String darstellen soll~\ref{fig:basics:jsonschema:3}. 

\begin{lstlisting}[language=Javascript,float=h!,caption={Multilinguales Objekt}, label={fig:basics:jsonschema:3}]
{
  "DE": "Die Drei",
  "EN": "The three"
}
\end{lstlisting}

Zukünftig soll es möglich sein weitere Sprachen zu ermöglichen. Würde man dieses Objekt als Tabelle definieren, müsste bei zukünftigen Änderungen wie das Hinzufügen oder Entfernen einer Sprache eine Datenbankmigration durchgeführt werden. Um diese Objekte flexibel zu halten wurde sich für eine Speicherung als hstore entschieden.

Hstore differenziert sich von jsonb, indem es nur eine Ebene von Schlüssel-Werte-Paaren ohne weitere Verschachtelungen zulässt und diese als String abspeichert. Erst durch die Einführung von jsonb wurde aus Postgres auch eine dokumentenorientierte Datenbank.
Denn im Gegensatz zu hstore können jsonb Datensätze beliebig tief verschachtelt werden und
darüber hinaus werden sie in einem dekomprimierten Binärformat gespeichert, wodurch die Eingabe aufgrund des zusätzlichen Konvertierungs-Overheads etwas langsamer,
die Verarbeitung jedoch erheblich schneller ist, da kein Reparsen erforderlich ist~\cite{postgresql-json}.
Ansonsten haben beide Typen in vielen Dingen die gleichen Verhaltensweisen. Wie bei der Eingabe doppelter Schlüssel wird nur der letzte Wert beibehalten.
Zudem wurden für beide Datentypen eine beachtliche Menge an Operationen und Funktionen bereitgestellt (siehe Abbildung~\fullref{tbl:basics:hstore-operations}), die es möglich machen,
auf SQL Ebene einen hstore oder jsonb Datensatz fast wie ein Hash in Ruby oder ein JSON-Objekt in Javascript zu behandeln~\cite{postgresql-hstore}.

\begin{table}[h]
    \small
    \begin{tabular}{|p{0.21\textwidth}|p{0.31\textwidth}|p{0.33\textwidth}|p{0.1\textwidth}|}
        \hline
        \textbf{Operator} & \textbf{Beschreibung} & \textbf{Beispiel} & \textbf{Ergebnis}  \\ \hline
        \inlinec{hstore -> text} & liefert Wert für Schlüssel (NULL, wenn nicht vorhanden) & \inlinec{'a=>x, b=>y'::hstore -> 'a'} & \inlinec{x}  \\ \hline
        \inlinec{hstore -> text[]} & liefert Werte für Schlüssel (NULL, wenn nicht vorhanden) & \inlinec{'a=>x, b=>y, c=>z'::hstore -> ARRAY['c','a']} & \inlinec{{"z","x"}}  \\ \hline
        \inlinec{hstore || hstore} & konkateniert hstores & \inlinec{'a=>b, c=>d'::hstore || 'c=>x, d=>q'::hstore} & \inlinec{"a"=>"b", "c"=>"x", "d"=>"q"}  \\ \hline
        \inlinec{hstore ? text} & prüft ob hstore Schlüssel beinhaltet & \inlinec{'a=>1'::hstore ? 'a'} & \inlinec{t}  \\ \hline
        \inlinec{hstore - text} & entfernt Schlüssel des linken Operanden & \inlinec{'a=>1, b=>2, c=>3'::hstore - 'b'::text} & \inlinec{"a"=>"1", "c"=>"3"}  \\ \hline
        \inlinec{hstore - hstore } & entfernt passende Schlüssel/Wert-Paare vom linken Operanden & \inlinec{'a=>1, b=>2, c=>3'::hstore - 'a=>4, b=>2'::hstore} & \inlinec{"a"=>"1", "c"=>"3"}  \\ \hline
    \end{tabular}
    \vspace{5pt}
    \caption{Beispiele für bereitgestellte Hstore Operationen}
    \label{tbl:basics:hstore-operations}
\end{table}