\section{Anforderungsanalyse}
\label{sec:requirements}

In diesem Kapitel werden die Anforderungen beschrieben die an die
Containerisierung und die Bildverwaltung gestellt wurden.

Neben der Bildverwaltung stand auch die Verbesserung der Zug\"{a}nglichkeit der
Hosting und Entwicklungsumgebung. Docker ist eine verbreitete L\"{o}sung um
Anwendungen in sogenannten Containern zu isolieren, die weitgehend
unabh\"{a}ngig von der verwendeten Hardware ist und auch auf den meisten
verbreiteten Betriebssystemen eingesetzt werden kann, ohne dass die auf dem
Hostsystem installierte zus\"{a}tzlich installierte Software die Anwendung im
Container beeinflusst, sodass der Einfluss betriebssystemspeziefischer
Eigenheiten auf die Anwendung weitgehend vermieden werden kann.

Zur Bildverwaltung geh\"{o}hren ein API, die es erm\"{o}glicht, Bilder und
essentielle Metadaten anzulegen, auszugeben, zu bearbeiten und zu l\"{o}schen,
eine Bibliothek, die die durch die API bereitgestellten Funktionen Serverseitig
umsetzt und eine Frontendintegration, die dem Verwalter eines Projekts das
Anlegen, Betrachten, Modifizieren und L\"{o}schen der Bilder und Metadaten
erm\"{o}glicht sowie die Verwendung der Bilder in Projektvorschau, dem
Seiteneditor und den Datenbanktabellen.

\subsection{Containerisierung}



\subsubsection{Entwicklungsumgebung}



\subsubsection{Testumgebung}



\subsubsection{Produktivumgebung}



\subsection{Bildverwaltung}



\subsubsection{Lizenzen \& Metadaten}

Da Bilder nicht alle unterst\"{u}tzenswerten Bildformate Metadaten \"{u}ber
Urheber und Lizenz standatisiert haben, ist es nicht m\"{o}glich die aus
rechtlicher Sicht Notwendigen Metadaten zum Urheber und zur Lizenz der Bilder
in der Datei selbst zu speichern und aus rechlicher Sicht, reicht die blo\sze
Anwesenheit der Daten in der Datei nicht um Beispielsweise die Anforderung der
Urhebernennung zu erf\"{u}llen, weshalb die aus rechlicher Sicht notwendigen
Metadaten in einer zus\"{a}tzlichen Datenquelle gehalten werden m\"{u}ssen.

Die aus rechtlicher Sicht notwendigen Daten sind Name des Urhebers mit Link auf
eine Webpr\"{a}senz und der Name der Lizenz unter der das Bild verwendet wird
mit Link auf die Lizenz.

\subsubsection{Darstellungsformen}

F\"{u}r die Bilder sollen zwei Darreichungsformen existieren: Die Abbildung
(Figure) und das gestalterische Element (Graphic), die sich durch ihre
Einbettung und den Ort der Quellenangaben unterscheiden. Die Abbildung
eingerahmt, die Metadaten am Bild dargestellt, das Gestalterische Element bettet
das Bild in einen Link ein, der auf die Quellenangabe am Ende der Seite verweist.

\subsubsection{Darstellung von Quellenangaben}

Bei Abbildungen werden Name des Bildes, Urheber des Bildes und Name der Lizenz
direkt unter dem Bild innerhalb des eingerahmten Bereichs dargestellt, der Name
des Urhebers ist ein Link auf die Webpr\"{a}senz, der Name der Lizenz ein Link
auf den Volltext der Lizenz.

Die Quellenangaben der gestalterischen Elemente findet tabellarisch im Footer der
Webseite statt, Alle dargestellten gestalterischen Elemente werden dort in der
Reihenfolge ihres Erscheinens im Quelltext der Seite aufgelistet. Die Tabelle
besteht aus den Spalten Nummer des Quellverweises, Name des Bildes,
Name des Urhebers und Name der Lizenz, wobei die Nummer des Quellverweises auf
das gestalterische in der Seite verlinkt, der Name des Bildes auf eine neue
Seite bestehend aus einer Abbildung des Bildes gem\"{a}\sz der obig genannten
Darstellungsform von Abbildungen, der Name des Autoren ein Link auf dessen
Webpr\"{a}senz und der Name der Lizenz ein Link auf den Volltext der Lizenz.

\subsubsection{Notwendigkeit von Quellenangaben}

Sofern notwendig kann die obig genannte Darstellung der Quellenangaben nicht
weiter beeinflusst werden als durch die Auswahl eines der genannten Typen. Liegt
jedoch eine Grafik vor, deren Lizenz weder die Namensnennung noch die Nennung
der Lizenz erfordert, wie beispielsweise CC0, oder der Autor der Seite selbst der
Urheber ist, soll auf die Quellenangabe verzichtet werden k\"{o}nnen, im Falle
einer Abbildung also nur der Name angezeigt werden und im Falle des
gestalterischen Elements der Eintrag in der Liste komplett weggelassen werden.

\subsubsection{Endger\"{a}toptimierung}



\subsubsection{Caching}



\subsubsection{Datenhaltung}



\subsubsection{API}

Die API stellt verschiedene Routen bereit, die sich in das existierende
Strukturen einf\"{u}gen, daher sind alle Routen unterhalb von
\begin{description}
  \item ['/api/project/:project_id/image']
\end{description}
angesiedelt

\begin{description}
  \item [/api/project/:project_id/image]
    \begin{description}
    \item [GET]
    \item [POST]
    \end{description}
  \item [/api/project/:project_id/image/:image_id]
    \begin{description}
    \item [GET]
    \item[POST]
    \item[DELETE]
    \end{description}

\end{description}

\subsubsection{Library}



\subsubsection{Frontendintegration}



\subsubsection{Seiteneditorintegration}



\subsubsection{Datenbankintegration}



\subsubsection{Schemaeditorintegration}



\subsubsection{Queryeditorintegration}



%%% Local Variables:
%%% mode: latex
%%% TeX-master: "thesis"
%%% End:
