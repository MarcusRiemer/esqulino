\section{Einleitung}
\label{sec:introduction}

SQLino verfügt bereits über die Möglichkeit Webseiten aus Text und Tabellen aus
Datenbankabfragen zu erstellen. Nun lässt sich damit bereits eine ganze Menge an
Dingen tun, allerdings lassen sich viele Dinge ansprechender und leichter
verständlich darstellen, wenn Bilder verwendet werden können. Sei es ein Foto
des im Text beschriebenen beim Sturm auf ein Auto gefallenen entwurzelten
Baumes, der Schaltplan des vorgestellten Gadget Prototyps

\footnote{Zugunsten der leichteren Lesbarkeit wird
  in dieser Thesis auf die Ausdifferenzierung des Geschlechts von Personen
  verzichtet, da es für den Sachverhalt irrelevant ist und den Lesefluss stört.
  Wenn in dieser Thesis Personen genannt werden, steht das grammatikalische
  Geschlecht in keinem Zusammenhang zum tatsächliche Geschlecht der jeweilig
  genannten Personen und es sind immer alle Personen gemeint, auf die die
  restliche Beeschreibung zutrifft, unabhängig des Geschlechts.}

\todo[inline]{ein bis zwei absaetze warum man bilder haben will}

Die Verwendung von Bildern auf Webseiten stellt immer wieder Probleme dar. Neben
deutlich zu hohen Dateigrößen, die Ladezeiten in die Länge ziehen und bei
Mobilgeräten das in Deutschland im Vergleich zu anderen Ländern um
Größenordnungen geringere Datenvolumen unötig schnell aufbrauchen - von der
Unbenutzbarkeit solcher Webseiten in der ``Edge-Wüste'' der ländlichen
Mobilfunknetzte und Breitbandanschlüsse, die ihrem Namen nicht gerecht werden
mal ganz abgesehen - ist ein weiteres Problem die mangelnde auseinandersetzung
mit dem Urheberrecht und Nutzungslizenzen. Sogenannte Sozialze Netzwerke und
Diskussionsplatformen verleiten den Nutzer Bilder einzustellen und dem Anbieter der
Platform laut AGB diverse Rechte an diesen Bildern zu übertragen. Da der
durchschnittliche Nutzer weder die AGB gelesen hat, noch sich dessen bewusst
ist, dass die Befugnis die Rechte an den Bildern Dritter an den
Platformbetreiber weiterzugeben bei besagten Dritten liegt und nicht bei der
eigenen Person.

Ein weiteres Problem ist die Verwaltung dieser Bilder. Ein Bild wird bei
irgendeinem gerade angesagtem Imagehoster abgelegt, eingebunden und vergessen.
Andere Bilder werden anderswo abgelegt, eingebunden und vergessen. Einige der
Imagehoster liefern nach tausend Aufrufen im Monat das Bild nicht mehr aus ohne
sich dafür bezahlen zu lassen, andere ändern nach einiger Zeit ihr Konzept und
lassen einbetten ihrer Bilder auf fremden Seiten nicht mehr zu und zeigen die
Bilder nur noch auf ihrer eigenen Werbefinanzierten Webseite an und wiederum
andere haben in der zwischenzeit Insolvenz angemeldet und alle dort
gespeicherten Bilder sind nicht mehr verfügbar.

Diese Thesis beschreibt den Prototypen einer Bildverwaltung für SQLino, die auf
diese Probleme eingeht, indem verwendete Bilder an einem Ort gesammelt und
verwaltet werden, den Bildern Metadaten wie Urheber und Lizenz zugeordnet
werden und dem Betrachter der Webseite dargestellt werden und Bilder beim
ausliefern auf die für das Endgerät geeignete Größe herunterskaliert werden.

Ein weiteres bearbeites Problem ist das Verwalten von Programmabhängigkeiten,
die beim Hosten, Entwickeln und Testen der Software, insbesondere auf
Verschiedenen Betriebssystemen.

%%% Local Variables:
%%% mode: latex
%%% TeX-master: "thesis"
%%% End:
