\section{Implementierung}
\label{sec:implementation-analysis}

\subsection{"Kontext"}



\subsubsection{Kommunikation via Description Interface}



\subsubsection{Server: Ruby on Rails}



\subsubsection{Client: Angular 4}



\subsubsection{Caching}

\TODO caching von html requests


\subsubsection{API}

Die API stellt verschiedene Routen bereit, die sich in das existierende
Strukturen einfügen, daher sind alle Routen unterhalb von
\begin{description}
  \item ['/api/project/:project_id/image']
\end{description}
angesiedelt

\begin{description}
  \item [/api/project/:project_id/image]
    \begin{description}
    \item [GET]
    \item [POST]
    \end{description}
  \item [/api/project/:project_id/image/:image_id]
    \begin{description}
    \item [GET]
    \item[POST]
    \item[DELETE]
    \end{description}

\end{description}

\subsubsection{Library}



\subsubsection{Frontendintegration}



\subsubsection{Seiteneditorintegration}



\subsubsection{Datenbankintegration}



\subsubsection{Schemaeditorintegration}



\subsubsection{Queryeditorintegration}



%%% Local Variables:
%%% mode: latex
%%% TeX-master: "thesis"
%%% End:
