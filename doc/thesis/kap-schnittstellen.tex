\section{Schnittstellen und Formate}

Dieses Kapitel beschreibt das Projektformat und die Kommunikation zwischen Server und Client.

\subsection{Verwaltung von Projekten}

Um den Betrieb für Schüler und Lehrer zu vereinfachen, wird eine Instanz des Servers also in der Lage sein mehrere Schülerprojekte gleichzeitig bereitzustellen. Dafür ist es aber zunächst einmal notwendig zu definieren, wie ein solches Projekt überhaupt strukturiert ist. Grundsätzlich ist ein Projekt eine Sammlung von Dateien in einer festgelegten Ordnerstruktur.

\begin{dirstruct}
  \dirtree{%
    .1 empty-project/.
    .2 queries/.
    .2 pages/.
    .2 config.yaml.
    .2 db.sqlite.
  }
  \caption{Leeres Projekt}
\end{dirstruct}

%%% Local Variables:
%%% mode: latex
%%% TeX-master: "thesis"
%%% End:
