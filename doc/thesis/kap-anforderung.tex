\section{Anforderungsanalyse}

Dieses Kapitel beschreibt die generellen Anforderungen an die Entwicklungsumgebung und einiger Beispielprojekte. Dabei werden auch einige theoretische Hintergründe betrachtet, speziell didaktische Konzepte für die Vermittlung von SQL oder mögliche Vorgehensweisen um Schüler bei der Entwicklung von Oberflächen sinnvoll anzuleiten. Den Schwerpunkt bildet die Beschreibung der beiden wichtigsten Komponenten der Entwicklungsumgebung: Den spezialisierten Editoren für SQL und Oberflächen.

Die hier ``erträumte'' Konzeption wird möglicherweise aufgrund der Endlichkeit der im Rahmen einer Master-Thesis zur Verfügung stehenden Zeitspanne nicht vollständig umgesetzt werden können. Trotzdem ist eine möglichst vollständige Erfassung von Anforderungen unerläßlich, um im Falle einer Fortführung der Entwicklung nicht aufgrund von unberücksichtigten Anforderungen große Umbauarbeiten vornehmen zu müssen. Die Planung konkreter Umsetzungsstrategien, inklusive einer Erläuterung der notwendigen Schnittstellen, werden in Kapitel \ref{sec:implementation-analysis}~\nameref{sec:implementation-analysis} besprochen. Sofern sich während der Entwicklung Ideen für bisher nicht bedachte Funktionalität ergaben, so wurden diese in Anlehnung an den ``Minus 100 Points''-Artikel von Eric Gunnerson \cite{gunnerson-minus-100} geprüft.

\info[inline]{Der Verzicht auf die technischen Hintergründe der Implementierung in diesem Kapitel soll es auch für ``normale'' Informatiklehrer verständlich machen, ohne Sie gleich mit den ``unwichtigen'' Details der Realisierung zu erschlagen.}

\subsection{Zielgruppe}
\label{sec:target-audience}

\missing[inline]{Fehlt noch, halt Schüler ohne vertiefte Mathe- oder Englischkenntnisse.}

\subsection{Grundprinzipien}
\label{sec:principles}

Nach der Betrachtung der zu bedienenden Zielgruppe und der Beschäftigung mit bereits existierenden Alternativen, ist es zunächst sinnvoll ein paar allgemeine Grundprinzipien zu formulieren. Diese Prinzipien bilden die Philosophie hinter der Schülerentwicklungsumgebung ab und dienen als Leitfaden für Designentscheidungen.

Praktisch erlaubt das vor allem eine relativ akkurate Abschätzung, ob sich die Implementierung einer bestimmten Idee lohnt und wie sie zu priorisieren ist. Diese Sortierung dieser Prinzipien ist entsprechend ihrer Bedeutung absteigend sortiert, das wichtigste Prinzip wird also zuerst genannt..

\begin{description}
\item[Semantik vor Syntax] \hfill\\
  Den Lernenden sollen kontextsensitiv sinnvolle Operationen angeboten werden, optimalerweise mit einer kurzen Erläuterung, warum gerade nur diese Teilmenge an Operationen möglich ist. Die eigentliche Programmierung erfolgt dann durch die Kombination von Bausteinen, ähnlich wie bei der Lernsoftware ``Scratch''. Durch kontinuirliches Feedback der Entwicklungsumgebung sollen die Lernenden in die Lage versetzt werden, auch ohne ständige Rückversicherung bei der Lehrkraft eigene Ansätze zu erproben.
\item[Motivation durch praktisch vorzeigbare Ergebnissen] \hfill\\
  Der Einstieg in die Programmierung ist oftmals von relativ langweiligen Programmen geprägt, häufig textbasierten Konsolenanwendungen, welche sich nicht gut im Freundes- oder Bekanntenkreis präsentieren lassen. Im Sonderfall der Vermittlung von SQL Kenntnissen ist das Ergebnis der Arbeit sogar überhaupt nicht sinnvoll zu demonstrieren, weil die erstellten Abfragen isoliert für sich stehen und häufig auch nur in der Entwicklungsumgebung der jeweiligen Datenbank ausführbar sind. Mit der im Rahmen dieser Arbeit zu erstellenden Software sollen sich hingen praktisch relevante, allerdings sehr datenorientierte Programme umsetzen lassen. Diese verfügen über von den Lernenden zusammengestellte Eingabemasken um Daten einzufügen oder zu manipulieren und verschiedene Ausgabesichten um den Datenbestand sinnvoll zu präsentieren.
\item[Schrittweise komplexere Benutzeroberfläche der Entwicklungsumgebung] \hfill \\
  Konventionelle Entwicklungsumgebungen sind Programme von Profis für Profis und bieten einen dementsprechend ausgerichteten Funktionsumfang. Gerade wenn man aber dabei ist etwas Neues zu lernen kann es sinnvoll sein, die Menge der möglichen Optionen zu beschränken. In diesem Sinne sollte die Lehrkraft die Möglichkeit haben, den Funktionsumfang der Entwicklungsumgebung für Schüler gezielt zu reduzieren. Es sollten sich also Funktionen der Entwicklungsumgebung ausblenden lassen, wenn dies aus didaktischen Gründen sinnvoll erscheint. Zum Beispiel könnten bestimmte Bedienelemente oder Bestandteile von SQL ausgeblendet werden, wenn diese noch nicht behandelt worden sind.
\item[Einfache Inbetriebnahme] \hfill \\
  Eine initiale Hürde jeder (Lern-)Software ist deren Installation, insbesondere bei Programmen aus dem Datenbankumfeld. Die Inbetriebnahme der für Server konzipierten Programme auf privaten, ``normalen'' Rechnern führt immer wieder zu Problemen aufgrund von nicht aufgelösten Abhängigkeiten, fehlenden Rechten beim Starten von Systemdiensten oder bei Dateizugriffen. Die zunehmende Heterogenität an Betriebssystemen, insbesondere die zunehmende Verwendung MacOS \cite{statista-os-verbreitung}, tut ein Übriges um die Verteilung von Software zu erschweren. Damit der eigentliche Lernprozess nicht schon vor dem Start der Entwicklungsumgebung behindert wird, ist eine möglichst einfache Inbetriebnahme von entsprechender Bedeutung. Die Informatiklehrkräfte stehen bei dem Betrieb der jeweiligen Programme in der Schule vor ähnlichen Problemen wie ihre Schüler, nur dass Sie auf die Konfiguration des Rechnerpools ihrer Schule oftmals nur einen eingeschränkten Einfluss haben. Die im Vergleich zu privaten Rechnern wesentlich restriktiver gehandhabte Rechte eines Schul-PCs verkomplizieren dieses Szenario zusätzlich.
\item[Fortführung der entwickelten Projekte] \hfill \\
  Lernumgebungen wie Scratch oder der AppInventor sind in sich geschlossene Systeme, deren Arbeitsergebnisse nur schwer in anderen Programmen oder Kontexten von Nutzen sind. Sobald der Lernende dann die Grenzen der verwendeten Lernsoftware erreicht hat, steckt er in einem Dilemma: Es ist für ihn notwendig auf eine andere Software auszuweichen, in diese kann er sein bestehendes Projekt aber nicht einfach mitnehmen. Die Arbeitsergebnisse dieser zu entwickelnden Software sollen daher zumindest einfach einsehbar sein, optimalerweise nach einem Export sogar mit gängigen Entwicklungsumgebungen oder Texteditoren erweiterbar.
  
  
\end{description}

\subsection{``Out-of-Scope''\todo{Deutsch?}}
\label{sec:out-of-scope}

Umgekehrt ist es auch wichtig, den Umfang des zu entwickelnden Programms auf einem im Rahmen der Thesis machbaren Umfang zu begrenzen. Die folgenden Ideen wären mehr oder minder naheliegende Ergänzungen, welche aufgrund der zur Verfügung stehenden Zeit aber nicht implementiert werden sollen.

\begin{description}
\item[Datenmodellierung] \hfill \\
  Der Schwerpunkt dieser Arbeit liegt zunächst auf der Vermittlung von Kenntnissen zur Abfrage und Manipulation von Daten in einem bestehenden Schema. Änderungen an diesem Schema sind nicht vorgesehen, demzufolge ist auch der Neuentwurf eines Schemas mit externen Mitteln zu bewerkstelligen. Völlig außerhalb des Umfangs dieser Arbeit ist die Überführung von konzeptionellen Modellen (z.B. ER-Schemata) in physikalische Modelle.
\item[Aufwändiges Design von Benutzerschnittstellen] \hfill \\
  Auch wenn die Konzeption der Benutzerschnittstelle für die verschiedenen Masken in den eben aufgezählten Prinzipien auftaucht, ist es wichtig den engen Rahmen dieses Aspektes zu verstehen. Es geht um die Schaffung von einfachen, datenzentrierten Eingabemöglichkeiten, nicht um die Umsetzung besonders kreativer Bedienkonzepte. Dementsprechend ist z.B. die Erweiterung der zur Verfügung stehenden Eingabeelemente durch die Lernenden außerhalb des Rahmens dieser Arbeit.
\end{description}

\subsection{Allgemeines Konzept der Entwicklungsumgebung}
\label{sec:design-general-concept}

Dieses Kapitel beschreibt sehr allgemeinen Anforderungen an die Entwicklungsumgebung, die Details dazu werden in den spezifischeren Kapiteln besprochen, welche nach Möglichkeit verlinkt sind.

Um die einfachste Verwendung der Software für Lernende zu gewährleisten, wird die esqulino als Webanwendung entwickelt. Für die ersten Schritte der Softwareentwicklung mit SQL reicht dann ein beliebiger aktueller Browser. Wie aus Kapitel \ref{sec:related-work} \nameref{sec:related-work} ersichtlich, befindet sich esqulino mit diesem Ansatz in guter Gesellschaft: Das ``große Vorbild'' Scratch ist über den Browser erreichbar, genau so wie auch die anderen explizit als Lernsoftware beschrieben Anwendungen.

Selbstverständlich muss eine Entscheidung mit einer solchen Tragweite natürlich dennoch genauer hinterfragt werden. Das vergleichbare Arbeiten sich ebenfalls für diesen Ansatz entschieden haben ist zwar ein starkes Indiz, isoliert betrachtet aber natürlich keine ausreichende Begründung. Immerhin existiert von Scratch auch eine lokal installierbare Fassung, welche der Entwicklung der Webversion allerdings leicht hinterherhinkt\footnote{https://scratch.mit.edu/scratch2download/ "The backpack is not yet available."}.

Praktisch bedeutet dieser Ansatz vor allem eine Verschiebung der Probleme mit der Inbetriebnahme auf z.B. eine Lehrperson. Bei der Konzeption von esqulino wurde davon ausgegangen, dass die Bereitstellung eines Serverdienstes in Zeiten von Virtualisierungs- und Containerumgebungen für Informatik-affine Lehrkräfte keine nennenswerte Hürde mehr darstellt.

Ebenfalls aus Gründen der einfacheren Zugänglichkeit sollte eine einzelne Serverinstanz in der Lage sein, mehrere Projekte simultan zu bedienen. Die Lehrperson kann also mit einem einzigen Serverprozess eine ganze Klasse versorgen. Eine Begleiterscheinung dieses zentralen Angebots ist allerdings die notwendige Implementierung von zumindest rudimentären Zugriffsbeschränkungen: im Normalfall sollen Schüler die Projekte ihrer Klassenkameraden zwar jederzeit begutachten, aber nicht modifizieren können. Würde die Entwicklungsumgebung von den Schülern lokal betrieben, entfiele die unmittelbare Notwendigkeit einer Zugriffsbeschränkung.

Durch den Betrieb eines zentralen Servers für die Projekte der Schüler entfällt auch eine weitere typische Problemquelle im Unterichtsalltag: Sofern der Server tatsächlich durchgehend betrieben wird, ist es sehr einfach auch von zuhause an den Projekten weiter zu arbeiten. Die Notwendigkeit manuell Dateien über USB-Sticks, Netzlaufwerke oder sonstige Dienste zu synchroniseren entfällt.

Ein weiterer Vorteil eines zentralen Servers ergibt sich in der praktischen Vorzeigbarkeit der eigenen Ergebnisse: Durch einfache Weitergabe der URL für Endbenutzer können die Lernenden ihre entwickelten Applikationen mit Eltern, Freunden und anderen Personen teilen. 

Nicht verschwiegen werden sollen allerdings auch die negativen Nebeneffekt dieser grundsätzlichen Entscheidung: Zunächst einmal wird dadurch ein eigenständiger Betrieb der Entwicklungsumgebung durch Schüler deutlich erschwert wird. Diese müssten auf ihrem eigenen Rechner plötzlich doch eine Server-Instanz betreiben. Faktisch wird die Bereitstellung einer dezidierten Umgebung für den Server also zur Pflichtaufgabe der jeweiligen Lehrperson. Vom Betrieb eines dedizierten Servers in der Schule, über die Verwendung eines ``privat greifbaren'' Servers der Lehrperson bis hin zur Verwendung einer Cloud-Computing-Instanz im Internet sind unterschiedlichste Szenarien denkbar. Der erfolgreiche Einsatz von esqulino hängt in der Praxis dann aber auch maßgeblich von der Verfügbarkeit dieses Serverdienstes ab.

\subsubsection{Projektbasiert}

Nachdem der Begriff des ``Projekts'' nun schon einige Male gefallen ist, wird es Zeit diesen eindeutig zu definieren. Abstrakt betrachtet handelt es sich bei einem Projekt um eine Datenbasis in Form einer Datenbank. Zu dieser Datenbasis kann man als Entwickler SQL-Abfragen entwickeln (\ref{sec:design-sql-editor} \nameref{sec:design-sql-editor}) und einige dieser Abfragen mit einer Oberfläche für Endbenutzer versehen (\ref{sec:design-ui-editor} \nameref{sec:design-ui-editor}).

Als unmittelbare Ausgangspunkt eines Projektes sollen einigermaßen einfache, aber grundsätzlich beliebige SQLite-Datenbanken dienen\footnote{Kapitel \ref{sec:implementation-database-system} \nameref{sec:implementation-database-system} erläutert, warum ausgerechnet SQLite zum Einsatz kommt und nicht ein anderes DBMS.}. Diese Schemata werden vorher mit externen Programmen erzeugt, was sowohl im Dialog mit den Schülern als auch ``einfach so'' erfolgt sein kann.

Schon um Vandalismus zu vermeiden, sind leider einfache Zugriffskontrollen auf Projektebene nötig. Um etwaige Missverständnisse von vornerein zu vermeiden: Ziel dieser Kontrollen ist es zu verhindern, dass z.B. die Quellen für die Abfragen oder Benutzeroberfläche eines Projektes entstellt oder gelöscht werden. \textbf{Nicht geschützt} werden können die eigentlichen Daten, also die Zeilen, in der Datenbank. Sobald in der Benutzeroberfläche für Endanwender Funktionalität zum Löschen oder Editieren von Datensätzen vorgesehen ist, gibt es keine einfache Möglichkeit legitime Veränderungen von illegalen Veränderungen zu unterscheiden.

Um zumindest einen grundsätzlichen Schutz zu ermöglichen, können für ein Projekt aber zwischen drei schreibenden Zugriffsarten für angelegte Benutzer gewählt werden.

\begin{description}
  \item[Entwickler] \hfill \\
    Mit diesem Zugangslevel hat ein legitimierter Benutzer das Recht beliebige Änderungen an dem Projekt vorzunehmen.
  \item[Benutzer] \hfill \\
    Diese Personengruppe hat keine Möglichkeit den Quelltext eines Projektes zu editieren, kann aber uneingeschränkt auf die Funktionalität der Oberfläche für Endbenutzer zugreifen. Sofern z.B. das Löschen von bestimmten Datensätzen über die Oberfläche für Endbenutzer nicht möglich ist, können diese Daten von dieser Personengruppe auch nicht gelöscht werden.
  \item[Gast] \hfill \\
    Gäste können keinerlei Schreibzugriffe vornehmen, weder über die Entwicklungsumgebung, noch über die Oberfläche für Endbenutzer.
\end{description}

\warning{Nur zur Sicherheit sei an dieser Stelle nochmals deutlich erwähnt, dass dieser Schutz nur für schreibende Zugriffe greift. Der lesende Zugriff auf den gesamten Datenbestand ist über die Entwicklungsumgebung auch für Gäste stets uneingeschränkt möglich.}

\missingfigure{UI-Konzept mit traditioneller ``Seitenleiste'' und ``Editorbereich''}

\subsubsection{Einfaches anlegen und kopieren von Projekten}

Im Sinne einer möglichst niedrigen Einstiegshürde soll es den Anwendern leicht gemacht werden, bestehende Projekte zu übernehmen. Im Informatikunterricht müssen häufiger zu Beginn bestimmte Schritte ausgeführt werden, ``weil das nun mal so ist'' (also eine ausführliche Erklärung zu diesem Zeitpunkt zu weit gehen würde). Im Falle des Sprachumfangs von SQL ist das zwar im Vergleich zu z.B. Java nicht ganz so dramatisch\footnote{``Herr Lehrer, was macht eigentlich dieses \texttt{public static void}?''}, aber nach Möglichkeit zu vermeiden.

Dementsprechend sollte normalerweise jeder Schritt beim Anlegen eines neuen Projektes für die Schüler nachvollziehbar sein. Wenn dann aber doch einmal eine Serie von ``mechanisch auszuführenden'' Anweisungen erforderlich sein sollte, wäre es aber dennoch praktisch, diesen Vorgang durch eine Kopie eines schon bestehenden Projektes abzukürzen. Die Lehrkraft würde in diesem Fall ein Projekt für ihre Schüler vorbereiten, welches diese dann mit einem Klick in ein anders benanntes Projekt kopieren können. Es werden dann alle Bestandteile, möglicherweise mit einer einfachen Anpassung der Zugriffsrechte, unter dem neuen Namen verfügbar gemacht.

\subsubsection{Zwei Modi: Entwickeln und Anschauen}

Grundsätzlich unterschieden werden muss bei der Schülerentwicklungsumgebung, ähnlich wie bei Scratch, zwischen zwei Betriebsmodi für ein Projekt: Zunächst wird ein Projekt im Entwicklermodus betrieben, in diesem Fall stehen dem Benutzer alle Entwicklungstools zur Verfügung. Wenn es dann später einmal fertig ist und an Endanwender, z.B. Bekannte oder Freunde, weitergegeben wird, erwarte diese natürlich eine normale Benutzeroberfläche zu sehen. Der Wechsel zwischen diesen beiden Modi sollte dabei, ebenfalls analog zu Scratch, zu jedem Zeitpunkt möglich sein. Das ``Verstecken'' von Vorgehensweisen in Projekten ist nicht vorgesehen, jedes fremde Projekt soll auch als Inspiration für die Umsetzung eigener Ideen dienen können.


\subsection{Didaktisch sinvolle Teilmengen von SQL}
\label{sec:sql-subset}

\warning{Nicht alle hier beschriebenen Optionen sind notwendigerweise in der Entwicklungsumbegung implementiert! Dieses Kapitel ist eine \textit{allgemeine} Betrachtung von didaktisch sinnvollen Teilmengen von SQL. Um zu sehen, welche Funktionalität tatsächlich implemtiert worden ist lohnt sich ein Blick in Anhang \ref{app:implemented-features}:~\nameref{app:implemented-features}.}

Die uneingeschränkte Umsetzung des SQL-Standards würden sowohl für den Verfasser dieser Arbeit als auch für die Anwender der Entwicklungsumgebung viele unpraktische Probleme aufwerfen. Der SQL-Sprachumfang ist mittlerweile gewaltig und die Anzahl der verschiedenen Dialekte unüberschaubar groß. ``Normale'' Entwicklungsumgebungen für SQL werden von großen Entwicklerteams konzipiert und umgesetzt, der Autor dieser Arbeit bildet hingegen ein (immerhin motiviertes) Ein-Mann-Team.

Glücklicherweise wäre aber auch der Zielgruppe dieser Entwicklungsumgebung mit einer vollständigen Umsetzung des SQL-Standards überhaupt nicht gedient. Im Gegensatz zu allen anderen Entwicklungsumgebungen da draußen soll die Stärke dieser Entwicklungsumgebung in einer sinnvollen Reduktion liegen. Dieses Kapitel beschreibt daher, wie man zum ``normalen'' SQL kompatible Untermengen definieren kann, um einige spezifische Aufgaben zu erfüllen. Oder in Anlehnung an die 80/20 Regel ausgedrückt: Es gilt jene 20\% des Sprachumfangs abzubilden, mit denen sich 80\% der typischen Aufgaben gut lösen lassen.

Passenderweise erlaubt die Struktur von SQL die Angabe von sehr lokalen Einschränkungen, deren Auswirkungen sich sehr gut abschätzen lassen. Einzig der Umgang mit Ausdrücken überspannt fast alle Bereiche von SQL: Diese können an sehr vielen Stellen auftreten und müssen dementsprechend sorgfältig analysiert werden. Diese Arbeit wird dabei demonstrieren, dass es aus didaktischer Sicht sinnvoll ist, die Mächtigkeit der Ausdrücke initial zu beschränken. An deren Stelle tritt dann eine übersichtliche Anzahl von ``einfachen'' Ausdrücken, mit denen sich jedoch der überwiegende Teil aller auftretenden Fragestellungen lösen lässt.

\unsure[inline]{Erwähnen, inwiefern das herauslösen von Sprachfeatures bei einer ``normalen'' Programmiersprache komplizierter wäre? Es gibt bei SQL halt vergleichsweise wenige Abhängigkeiten und daher viele mögliche Reihenfolgen in denen man die Features aktivieren kann. Bei einer Sprache wie z.B. Java bauen die einzelnen Sprachfeatures sehr viel stärker aufeinander auf, es ergeben sich deutlich weniger Reihenfolgen in denen man diese sinnvoll aktivieren könnte.}

Wir klären in diesem Kapitel daher zunächst, welche Features von SQL überhaupt für die Entwicklungsumgebung von Belang sind und wie diese voneinander abhängen. Die Abgrenzung von Teilsprachen baut auf diesen hier herausgearbeiteten Optionen auf und ist Gegenstand von Kapitel \ref{sec:sql-subset-ranks} \nameref{sec:sql-subset-ranks}.

\unsure[inline]{Technisch gesehen soll es für Lehrkräfte möglich sein, fast beliebige Teilmengen von SQL für ihre Schüler vorzugeben, nicht nur die im Rahmen dieser Arbeit als sinnvoll erachteten Kombinationen. Das grundsätzliche Vorgehen sollte dabei normalerweise einer Whitelist entsprechen: Jede explizit erlaubte Funktionalität steht im SQL-Editor (siehe Kapitel \ref{sec:design-sql-editor} \nameref{sec:design-sql-editor}) zur Verfügung. Und sofern eine Funktionalität nicht erlaubt sein sollte, ist sie auch nicht verfügbar. Der Begriff ``Einschränkung'' ist aus der Sicht einer Whitelist also zwar nicht ganz optimal gewählt, beschreibt die stattfindende Reduktion des Umfangs von SQL aber akkurat.\newline Alternativ sollen sich auch wortwörtliche Einschränkungen im Sinne einer Blacklist angeben lassen. Für die in der Regel sehr übersichtlichen Teilsprachen die in dieser Arbeit konzipiert werden, ist diese Variante allerdings bei weitem nicht so gut lesbar: Man muss den größtmöglichen Umfang des hier beschriebenen Dialekts immer im Hinterkopf haben. Außerdem kommen im Falle einer Erweiterung des Sprachumfangs, z.B. durch ein späteres Update der Entwicklungsumgebung, möglicherweise unerwünschte Möglichkeiten hinzu.}

\subsubsection{Mögliche Einschränkungen der Komponenten}
\label{sec:sql-subset-local}

Zunächst definieren wir einige mögliche Einschränkungen anhand der grundlegenden Struktur jeder SQL-Abfrage. Da die Reihenfolge der Komponenten nicht beliebig ist, ergibt sich eine sehr klare Struktur in der einige Komponenten auch ohne Probleme übersprungen werden können. Abbildung \ref{fig:sql-steps} wurde aus den Vorlesungsunterlagen von Prof. Dr. Hoffmann entnommen und demonstriert den logischen Ablauf bei der Verarbeitung einer SQL-Abfrage inklusive aller Zwischenschritte und optionaler Komponenten. Daraus ist folglich gut ersichtlich, dass alle Komponenten bis auf die logisch Erste (\texttt{FROM}) und Letzte (\texttt{SELECT}) optional sind.

\begin{figure}
  \centering \includegraphics{images/sql-steps.png}
  \caption{Komponenten und sich ergebende Zwischenschritte bei der Verarbeitung einer SQL-Abfrage von Prof. Dr. Ulrich Hoffmann.}
  \label{fig:sql-steps}
\end{figure}

Einige der im folgenden beschriebenen Optionen sind redundant. Das liegt zum einen an SQL selbst, die Sprache sieht schlichtweg mehrere syntaktische Varianten für semantisch äquivalente Sachverhalte vor. Zum anderen kann es aber auch aus didaktischen Gründen sinnvoll sein, die Wissensvermittlung mit einer Reihe von Spezialfällen zu beginnen\todo{Beleg}.

\begin{enumerate}
\item \textbf{Projektionen mit \texttt{SELECT}} \\
  Diese Komponente als solche kann natürlich nicht ausgeschlossen werden. Außerdem muss mindestens \ref{feat:select-all} $\lor$ \ref{feat:select-column} erlaubt sein, sonst können keine gültigen SQL-Abfragen erstellt werden.
  \begin{enumerate}
  \item \label{feat:select-all} Auswahl aller Spalten (``Sternchen-Operator'' \texttt{SELECT *})
  \item \label{feat:select-column} Auswahl von Spalten
  \item \label{feat:select-single-function} Ausdrücke aus einfachen Funktionsaufrufen zulassen
  \item \label{feat:select-simple-expression} Einfache Ausdrücke zulassen.
  \item \label{feat:select-expression} Ausdrücke zulassen.
  \item \label{feat:select-limit} Beschränkung der Datenmenge mit \texttt{LIMIT}
  \end{enumerate}
\item \textbf{Angabe von Datenquellen mit \texttt{FROM}} \\
  Auch die Auslassung dieser Komponente ist nicht zulässig, obwohl einige SQL Dialekte dies durchaus erlauben. Der Mehrwert dieser Auslassung wäre aus didaktischer Sicht aber nicht gegeben: Die Schüler sollen SQL Abfragen grundsätzlich auf Tabellen beziehen und keine Daten ``aus der Luft greifen''.
  \begin{enumerate}
  \item \label{feat:from-cross-join} Kreuzprodukt (\texttt{JOIN}) zulassen
  \item \label{feat:from-cross-comma} Kreuzprodukt (komma-separierte Schreibweise) zulassen
  \item \label{feat:from-natural-join} Automatische innere Verknüpfung zulassen (\texttt{NATURAL JOIN})
  \item \label{feat:from-inner-join} Innere Verknüpfung zulassen (\texttt{INNER JOIN}), erfordert \ref{feat:from-using} $\lor$ \ref{feat:from-on-simple}
  \item \label{feat:from-left-join} Linke äußere Verknüpfung zulassen (\texttt{LEFT OUTER JOIN}), erfordert \ref{feat:from-using} $\lor$ \ref{feat:from-on-simple}
  \item \label{feat:from-right-join} Rechte äußere Verknüpfung zulassen (\texttt{RIGHT OUTER JOIN}), erfordert \ref{feat:from-using} $\lor$ \ref{feat:from-on-simple}
  \item \label{feat:from-full-join} Volle äußere Verknüpfung zulassen (\texttt{FULL OUTER JOIN}), erfordert \ref{feat:from-using} $\lor$ \ref{feat:from-on-simple}
  \item \label{feat:from-using} \texttt{USING}-Bedingung zulassen
  \item \label{feat:from-on-simple} \texttt{ON}-Bedingung mit einfachen Ausdrücken zulassen
  \item \label{feat:from-recursive} Rekursive Verknüpfung zulassen 
  \item \label{feat:from-sub} Unterabfragen zulassen
  \end{enumerate}
\item \textbf{Filterung mit \texttt{WHERE} und \texttt{HAVING}} \\
  Da an dieser Stelle Ausdrücke zum Einsatz kommen, sind prinzipiell zwei unterschiedliche Ansätze denkbar. Zum einen die Verwendung einer vordefinierten Menge an Vergleichen mit einer einfachen Struktur, diese werden in \ref{sec:sql-subset-expression} beschrieben. Dem gegenüber steht der Ansatz fast beliebige Ausdrücke zu erlauben, was auf der einen Seite einen Ausbruch aus einem zu simplen Schema gleichkommen kann, auf der anderen Seite möglicher aber auch den Lernenden mit einem zu großen Freiheitsgrad überfordert.
  \begin{enumerate}
  \item \label{feat:where-allow} \texttt{WHERE} zulassen
  \item \label{feat:having-allow} \texttt{HAVING} zulassen
  \item \label{feat:where-and} Verknüpfungen mit \texttt{AND}
  \item \label{feat:where-or} Verknüpfungen mit \texttt{OR}
  \item \label{feat:where-simple-expression} Einfachen Ausdrücke zulassen
  \item \label{feat:where-expression} Beliebige Ausdrücke zulassen
  \end{enumerate}
\item \textbf{Bildung von Gruppen mit \texttt{GROUP BY}}
  \begin{enumerate}
  \item \label{feat:group-allow} \texttt{GROUP BY} zulassen
  \item \label{feat:group-multiple} Grupperierung mehrerer Spalten zulassen
  \end{enumerate}
\item \textbf{Sortierung mit \texttt{ORDER BY}} \\
  Bei dieser Komponente fehlt bewusst die Möglichkeit anhand eines Spaltenindexes aus dem \texttt{SELECT} zu sortieren. Da die gute Lesbarkeit der Abfragen ein erklärtes Ziel der Entwicklungsumgebung ist, wurde auf diese redundante Möglichkeit der Spaltenauswahl verzichtet.
  \begin{enumerate}
  \item \label{feat:order-allow} \texttt{ORDER BY} zulassen, die Auswahl der Richtung (\texttt{ASC} und \text{DESC}) ist dann grundsätzlich möglich
  \item \label{feat:order-select} Sortierung nach im \texttt{SELECT} erwähnten Spalten oder Ausdrücken zulassen
  \item \label{feat:order-any-column} Sortierung nach beliebigen Spalten zulassen
  \item \label{feat:order-expression} Sortierung anhand eines beliebigen Ausdrucks zulassen
  \end{enumerate}
\end{enumerate}

\subsubsection{Mögliche Einschränkungen der Ausdrücke}
\label{sec:sql-subset-expression}

Nach dieser Betrachtung der mit der unmittelbaren SQL-Struktur zusammenhängenden Funktionalität ist es nun an der Zeit, sich den Ausdrücken ausführlicher zu widmen. Um eine graduelle Lernkurve sowie eine sinnvolle Unterstützung durch den Editor zu gewährleisten, werden diese in drei Klassen eingeteilt:

\begin{description}
\item[Einfache Ausdrücke] \hfill\\
  Einfache Ausdrücke unterliegen einer ganz festen Struktur, die nur sehr begrenzt die Schachtelung oder Verkettung von zusammengesetzten Ausdrücken zulässt. Vereinfacht gesagt soll die Auswertungsreihenfolge dieser Ausdrücke auch ohne die Verwendung von Klammern ``intuitiv'' eindeutig sein.
  
  Die Bearbeitung dieser Ausdrücke ist nicht völlig frei, sondern kann auch hinsichtlich der möglichen Operatoren oder der Struktur eingeschränkt werden. In diesem Fall kann die Lehrkraft einige Teilausdrücke für die Bearbeitung sperren.
\item[Komplexe Ausdrücke] \hfill\\
  Komplexe Ausdrücke lassen sich schachteln bzw. verketten und könnten mit dem Scratch-ähnlichen SQL-Editor frei bearbeitet werden.
\item[Beliebige Ausdrücke] \hfill\\
  Textuell notierte Ausdrücke, die vom SQL-Editor nicht weiter geprüft werden sondern ungefiltert in die Abfrage eingesetzt werden. Sofern diese Art von Ausdrücken zugelassen wird erfolgt also auch keinerlei inhaltliche Einschränkung der zur Verfügung stehenden Möglichkeiten oder eine Syntaxüberprüfung.
\end{description}

Mit dieser Unterscheidung im Hinterkopf, können wir nun eine Menge an didaktisch sinvollen Einschränkungen definieren. Prinzipiell wäre es dabei auch denkbar, in unterschiedlichen Komponenten verschiedene Teilmengen der Ausdrücke zuzulassen, inhaltlich erscheint das aber wenig sinnvoll\todo{Treffen: Meinung geändert}. Die Schüler sollen durchaus verstehen, dass es sich bei Ausdrücken in SQL um ein allgemeines Konzept handelt, welches Prinzipiell in der \texttt{SELECT}-Komponente genau so angewandt werden kann wie beim \texttt{ORDER BY}. Oder plastisch ausgedrückt: Wenn sich die Schüler aus didaktischen Gründen erstmal nur mit Ausdrücke im Rahmen der \texttt{SELECT}-Komponente beschäftigen sollen ist das eine sinnvolle, und daher mögliche Einschränkung. Nicht sinnvoll hingegen wäre es die Anwendung der \texttt{LENGTH(X)} Funktion nur in einer der beiden Komponenten zuzulassen. Letzteres wird daher auch nicht unterstützt.

\begin{enumerate}
\item \textbf{Allgemeine Optionen} \\
  Diese Optionen betreffen prinzipiell alle Stellen, an denen die entsprechenden Arten von Ausdrücken zugelassen werden.
  \begin{enumerate}
  \item \label{feat:expr-allow-any-text} Ausdrücke als beliebigen Text zulassen, diese Option erlaubt die Umgehung aller Einschränkungen für einfache Ausdrücke!
  \item \label{feat:expr-allow-placeholder} An Stelle von Konstanten auch die Verwendung von Platzhaltern zulassen
  \end{enumerate}

\item \label{feat:expr-compute}  \textbf{Einfache Berechnende Ausdrücke} \\
  Diese Ausdrücke tauchen normalerweise in \texttt{SELECT}- oder \texttt{ORDER BY}-Komponenten auf, möglicherweise aber auch als Bestandteil eines einfachen Vergleichs.
  \begin{enumerate}
      \item \label{feat:expr-compute-simple-func} Einfacher Funktionsaufruf mit Spalten oder Konstanten als Argument
      \item \label{feat:expr-compute-simple-arithmetic} Einfache Rechenausdrücke mit einer mathematischen Operation und einer Konstanten
      \item \label{feat:expr-compute-simple-concat} Konkatenation von Strings
  \end{enumerate}

\item \textbf{Einfache Vergleichende Ausdrücke} \\
  Diese ``typischen'' Vergleichsausdrücke sollten ausreichen um einen Großteil der nötigen Vergleiche in \texttt{WHERE}, \texttt{HAVING} und \texttt{ON} Komponenten abzudecken. Die linke Seite dieser Ausdrücke ist stets eine zur Verfügung stehende Spalte, der Vergleichsoperator und die rechte Seite eine der hier aufgezählten Möglichkeiten.

  \missing[inline]{Testen wie weit man mit diesen einfachen Ausdrücken bei den Aufgaben von SqlZoo oder der Übung Datenbanken kommt.}
  \begin{enumerate}
  \item \label{feat:expr-compare-const} Vergleich einer Spalte mit einem konstanten Wert
  \item \label{feat:expr-compare-null} Vergleich einer Spalte gegen \texttt{NULL}
  \item \label{feat:expr-compare-other-col} Vergleich einer Spalte mit einer anderen Spalte
  \item \label{feat:expr-compare-simple-func} Vergleich einer Spalte mit einem einfachem berechnenden Ausdruck (siehe \ref{feat:expr-compute})
  \item \label{feat:expr-compare-between} Test einer Spalte mit dem \texttt{BETWEEN}-Operator und zwei Konstanten
  \item \label{feat:expr-compare-like} Nutzung des \texttt{LIKE}-Operators mit einer Konstanten
  \item \label{feat:expr-compare-in} Nutzung des \texttt{IN}-Operators mit einer Liste von Konstanten
  \item \label{feat:expr-compare-negate} Negation aller einfachen Ausdrücke
  \end{enumerate}

\item \textbf{Allgemeine Funktionen (in alphabetischer Reihenfolge)} \\
  In einem Umfeld für Anfänger ist zu erwarten, dass fast jede Funktion (bzw. jede logische Gruppe an Funktionen) einer eigenen Einführung bedarf. Die Lehrkräfte sollen diese daher isoliert zur Verfügung stellen können, ohne Ablenkung durch nicht benötigte Funktionen.

  Beim Unterschiedlich überladene Varianten der gleichen Funktion werden vermieden, um Irritationen zu vermeiden.
  \begin{enumerate}
  \item \label{feat:fun-abs} \texttt{ABS(X)}
  \item \label{feat:fun-coalesce} \texttt{COALESCE(X,Y,...)}
  \item \label{feat:fun-length} \texttt{LENGTH(X)}
  \item \label{feat:fun-casing} \texttt{LOWER(X)} und \texttt{UPPER(X)}
  \item \label{feat:fun-random} \texttt{RANDOM()}
  \item \label{feat:fun-round} \texttt{ROUND(X,Y)}
  \item \label{feat:fun-trim} \texttt{TRIM(X)} sowie \texttt{LTRIM(X)} und \texttt{RTRIM(X)}
  \end{enumerate}

\item \textbf{Aggregierende Funktionen (in alphabetischer Reihenfolge)} \\
  Die Verfügbarkeit dieser Funktionen hängt von der Verfügbarkeit der \texttt{GROUP BY}-Komponente ab (\ref{feat:group-allow}).
  \begin{enumerate}
  \item \label{feat:fun-avg} \texttt{AVG(X)}
  \item \label{feat:fun-count} \texttt{COUNT(X)}
  \item \label{feat:fun-group-concat} \texttt{GROUP\_CONCAT(X,Y)}
  \item \label{feat:fun-max-min} \texttt{MAX(X)} und \texttt{MIN(X)}
  \item \label{feat:fun-sum} \texttt{SUM(X)}, gibt im Falle von komplett fehlenden Werten \texttt{NULL} zurück
  \item \label{feat:fun-total} \texttt{TOTAL(X)}, gibt im Falle von komplett fehlenden Werten \texttt{0.0} zurück
  \end{enumerate}
\end{enumerate}

\subsubsection{Sprachstufen}
\label{sec:sql-subset-ranks}

Angelehnt an die bei Racket genutzte Idee der Feature-Levels soll dieses Kapitel einige sinnvolle Untermengen von SQL definieren, auf denen die mitgelieferten Beispielprojekte aufbauen. Es handelt sich dabei um Vorschläge, nicht um die einzig möglichen Ausprägungen von sinnvoll reduzierten SQL-Varianten.

\missing[inline]{Ich muss erst noch ein paar mehr Beispiele an der Hand haben, bis ich mich diesem Kapitel erneut widme.}


\subsection{Editor für SQL}
\label{sec:design-sql-editor}

Der grafische Editor soll grundsätzlich ähnlich zu den aus Scratch bekannten Bedienkonzepten funktionieren. Es kommen also distinkte Bedienelemente für die verschiedenen Komponenten einer SQL Abfrage zum Einsatz, kein reiner Texteditor. Der komponentenorientierte Editor soll dabei nicht unbedingt die Konzeption von beliebigen Abfragen ermöglichen, wohl aber die in Kapitel \ref{sec:example-queries} \nameref{sec:example-queries} beschriebenen exemplarischen Arbeitsabläufe unterstützen. 

Wenn ein Anwender der Entwicklungsumgebung an die Grenzen des unterstützenden Editors stößt, sollte er die Möglichkeit haben einmalig und für diese konkrete Abfrage eine Umwandlung in eine textbasierte Darstellung vornehmen zu können. Der umgekehrte Weg, also der Import von beliebigen SQL-Abfragen in den grafischen Editor, ist dann jedoch nicht mehr möglich. Die Implementierung eines Parsers für die Umwandlung der textuellen SQL-Repräsentation in die interne Darstellung von esqulino würde im Verhältnis zum marginalen Nutzen einen unverhältnismäßigen Aufwand bedeuten.

Grundsätzlich bedürfen einige Komponenten der Abfrage besonderer Aufmerksamkeit, weil sie große Auswirkungen auf das Verhalten der anderer Komponenten haben. Vorrangig ist hier die \texttt{GROUP-BY} Komponente zu nennen. Sobald die Query mit einer \texttt{GROUP BY} Komponente ausgestattet wird ist der Zugriff auf die konkreten Spalten einzelner Zeilen z.B. im allgemeinen Fall nicht mehr möglich. Das hinzufügen (oder entfernen) dieser Komponente hat also große Auswirkungen auf die Korrektheit der gesamten Abfrage. Ähnliches gilt für den Umgang mit \texttt{JOIN}s: Aus diesen leitet sich ab welche Daten von der Entwicklungsumgebung überhaupt angeboten werden sollten.

\subsubsection{Visuelle Gestaltung}

Für technische Details irrelevant, aber unbedingt ebenfalls im Voraus zu klären ist die Frage, ob es sinnvoll wäre, das sehr bunte, blockige Design von Scratch zu imitieren. Abbildung \ref{fig:compare-colourful} zeigt einen Vergleich eines frühen Prototypen in einem an Syntax-Highlighting angelehnten Design und in einem wesentlich bunteren, blockigen Design.

\todo[inline]{\textbf{Todo}: Quelle mit Beleg für ``Kinderfreundliches'' buntes Design suchen. Das wird man bei Scratch schon aus gutem Grund gemacht haben. Relevante Frage: Zielgruppe (in Jahren) von Scratch vs. Zielgruppe dieses Projektes?}

Aber auch wenn man den sehr bunten Stil gegen einen etwas nüchtereren, aber immer noch ``blockartigen'' Stil austauschen würde, ergeben sich dadurch nur vergleichsweise wenige Vorteile. In Scratch zeigen z.B. die Konnektoren des Blocks für eine Endlosschleife sehr deutlich, wie sich der Kontrollfluss durch dieses Element verändern wird. Für eine vollständige Programmiersprache ist das mit Sicherheit eine gute Wahl, aber für den sehr linearen Ablauf einer SQL-Abfrage ist dieser Umstand nicht von Bedeutung.

Ein weiterer Vorteil eines etwas nüchterneren, IDE-ähnlichen Designs wäre zudem die größere Nähe zu ``normalen'' Entwicklerprogrammen. Diese wirken möglicherweise weniger einschüchternd, wenn man sich schon an den Anblick von recht viel Text mit Syntax-Highlighting gewöhnt hat.

Letztendlich überwiegt aber ein sehr viel profanerer Fakt: Der Autor dieser Arbeit ist kein Grafikdesigner und würde vermutlich kein ansprechendes und zugleich buntes Farbkonzept auf die Beine stellen können.

\begin{figure}[h]
  \begin{subfigure}[b]{0.45\textwidth}
    \includegraphics[width=\textwidth]{images/sql-sketch-early-colourful}
    \caption{Starke Farbakzente, ähnlich zu Scratch}
    \label{fig:screen-sql-editor-early-colourful}
  \end{subfigure}\hfill
  \begin{subfigure}[b]{0.45\textwidth}
    \includegraphics[width=\textwidth]{images/sql-sketch-early-syntax-highlight}
    \caption{Syntax-Highlighting, ähnlich einer IDE}
    \label{fig:screen-sql-editor-early-syntax-highlighting}
  \end{subfigure}
  \caption{Vergleich unterschiedlicher Gestaltungsansätze}
  \label{fig:compare-colourful}
\end{figure}

\subsubsection{Grundsätzlicher Aufbau des Editors}

Ein grundsätzlicher Nachteil dieses Komponentenorientierten Bedienkonzeptes ist der nötige Platz für die Unterbringung aller verwendbaren Komponenten. Es muss sorgfältig geplant werden, wie diese anzuordnen sind und welche Optionen in welchem Kontext gerade sichtbar sein müssen. Anders als bei Scratch muss dabei nicht unbedingt Drag \& Drop das vorherrschende Bedienparadigma sein. Da der Benutzer immer nur eine Abfrage zur Zeit bearbeiten können soll, ergibt sich der einzig mögliche Platz für viele Blöcke automatisch. Darüber hinaus ist die Reihenfolge eines Großteils der Komponenten einer Abfrage sehr strikt festgelegt, so kann eine \texttt{GROUP BY} Komponente nicht an beliebigen Stellen verwendet werden, sondern nur nach der \texttt{FROM} oder der \texttt{WHERE} Anweisung. Andere Komponenten wie z.B. \texttt{HAVING} oder logische Verknüpfungen mit \texttt{AND} oder \texttt{OR} sind nicht nur von der Reihenfolge, sondern auch von der Existenz anderer Bestandteile abhängig. Die Verwendung eines separaten Bereichs mit allen Bestandteilen einer SQL-Abfrage (bei Scratch ``Toolbox'' genannt) wäre also zwar denkbar, sollte aber zumindest keinesfalls Drag \& Drop erzwingen, sofern das hinzufügen der Komponente nur an einer einzigen Stelle möglich ist.

\begin{wrapfigure}{r}{0.46\textwidth}
  \includegraphics[width=0.45\textwidth]{images/sql-sketch-all-editing}
  \caption{So nicht! Simultane Anzeige (fast) aller Möglichkeiten}
  \label{fig:screen-sql-editor-all-editing}
\end{wrapfigure}

Die Alternative dazu wäre eine Anzeige von ``Platzhaltern'' für die entsprechenden Komponenten unterhalb der eigenen Abfrage. Ein Klick auf den Platzhalter wandelt diesen dann in einen konkreten Block um und fordert ggfs. zur Angabe der benötigten Parameter auf. Im Rahmen der entwickelten Prototypen hat sich herausgestellt, dass eine permanente Anzeige aller Editierungsmöglichkeiten mit einem sehr überladen wirkenden Benutzerinterface einher geht, insbesondere was die Einblendung von Platzhaltern angeht. Abbildung \ref{fig:screen-sql-editor-all-editing} zeigt einen Screenshot des Prototypen, bei dem nahezu alle denkbaren Editierungsoptionen gleichzeitig verfügbar sind.

Dementsprechend wird ein anderer Ansatz verfolgt werden müssen: Es lässt sich immer ein Block zur Zeit in einen ``editieren'' Modus versetzen. Nur innerhalb dieses Blocks erlauben die Bedienelemente dann eine Änderung der Komponente. Für einige Komponenten, speziell \texttt{FROM}, \texttt{WHERE} und \texttt{HAVING} lassen sich neu anzuhängende Komponenten auch sehr gut im Rahmen dieses Modus anbieten. Es stellt sich dann natürlich die Frage, wie man Blöcke ohne einen eindeutig passenden Kontext, z.B. ein \texttt{ORDER BY}, der Abfrage hinzufügen kann. Eine Antwort darauf wird sich vermutlich erst durch reale Tests mit der Oberfläche finden lassen, die folgenden Ansätze sind aber denkbar:

\begin{itemize}
\item Die \texttt{SELECT}-Komponente hat einen losen Bezug zu allen anderen Komponenten. Schwierig einzuordnende Komponenten wie \texttt{WHERE}, \texttt{GROUP BY}, \texttt{ORDER BY} \texttt{LIMIT} oder auch \texttt{UNION} werden immer dann angeboten, wenn die \texttt{SELECT}-Komponente gerade editiert wird.
\item Eine Leiste mit möglichen Komponenten ohne eindeutigen Kontext erlaubt deren Ergänzung zu jedem Zeitpunkt.
\end{itemize}

\info[inline]{Im Editor immer zu sehen, aber aktuell nicht beschrieben: Eine Tabelle mit den Daten, welche die Abfrage im aktuellen Zustand liefern würde.}

\subsubsection{Unterstützung bei Ausdrücken und Vergleichen}
\label{sec:sql-compare-and-expression}

Die Abbildungen des Prototypen haben einen Umstand schon vorweggenommen: Die Vergleiche in den \texttt{ON} und \texttt{WHERE} Ausdrücken sollten nach Möglichkeit keine einfachen Textfelder sein, sondern dem Benutzer sinnvolle, kontextabhängige Vorschläge machen.

Der überwiegende Teil an für Schüler sinnvollen Verknüpfungen oder Ausdrücken lässt sich gut kontextgestützt implementieren. Dazu werden im Editor einige typische Schablonen hinterlegt, zu deren ``Löchern'' dann Vorschläge von der Entwicklungsumgebung gemacht werden. Dieser Unterstützungsprozess basiert dabei auf Datentypen und geht immer von der Auswahl einer Spalte der Abfrage aus. Wenn ein Schüler also einen neuen Vergleich hinzufügen möchte, wählt er zunächst eine Spalte und erhält dann Vorschläge für mögliche Operatoren und die rechte Seite des Vergleichs. Für Ausdrücke im \texttt{SELECT} ist dieses Vorgehen ähnlich, das Ergebnis des Ausdrucks muss dann nur nicht zwingend ein Boolescher Wert sein. Im Rahmen dieser Arbeit werden diese strukturell vordefinierten Ausdrücke als \textbf{einfache Ausdrücke} bezeichnet und könnten z.B. die folgenden Szenarien unterstützen:

\begin{itemize}
\item Für \texttt{TEXT} Spalten können in Ausdrücken einige einfache Stringfunktionen wie \texttt{LOWER} oder \texttt{UPPER}) angeboten werden. Ebenfalls interessant wäre eine Implementierung der \texttt{LENGTH}-Funktion, da diese einen vom Parametertyp abweichenden Rückgabetyp aufweist.
\item Mathematische Ausdrücke für \texttt{NUMBER} Spalten sind ebenfalls denkbar, der nötige Implementierungsaufwand ist im Vergleich zum erwartbaren Nutzen allerdings vergleichsweise groß.
\item Im Falle des Vergleichs von \texttt{NUMBER} Spalten werden konstante Werte oder andere \texttt{NUMBER} Spalten angeboten.
\item Für Vergleiche von \texttt{TEXT} Spalten bietet sich der unscharfe Vergleich mit dem \texttt{LIKE} Ausdruck an.
\item Perspektivisch interessant, aber vermutlich schon außerhalb des Umfangs dieser Arbeit, wäre der Umgang mit Datums und Zeitfunktionen.
\end{itemize}

Da es an allen Stellen, bei denen in SQL Werte verglichen oder ausgegeben werden, möglich ist, diese durch Funktionen zu verändern oder durch Ausdrücke zu berechnen, wird allerdings ein Rückgriff auf beliebige, textuelle Ausdrücke vorgesehen werden müssen. Es wird kaum möglich sein, einen gestützten Editor für den vollen Umfang aller denkbaren Ausdrücke bereitzustellen: spätestens aufgrund der möglichen Schachtelung und Verkettung können diese sehr schnell enorm kompliziert werden. Eine inhaltliche Prüfung solcher \textbf{beliebiger Ausdrücke} wäre zumindest im Rahmen dieser Arbeit nicht zu leisten.

Zwischen diesen beiden Extremen läge die Möglichkeit einen allgemeinen Editor für Ausdrücke zu implementieren. Dieser würde, ähnlich wie Ausdrücke in Scratch, die Schachtelung und Verkettung von mehr oder minder beliebigen, aber stets syntaktisch korrekten, Teil-Ausdrücken unterstützen. Abfragen mit dieser strukturellen Freiheit werden als \textbf{komplexe Ausdrücke} bezeichnet.

\unsure[inline]{Ich bin mir noch nicht sicher, wie ich das gerne handhaben möchte. Vermutlich ein mehr oder minder separater Editor für Ausdrücke? Auf jeden Fall muss ich hier den Umfang sehr deutlich vorher abstecken. Gerade an dieser Stelle wäre ein Scratch-ähnlicher Editor dann doch wieder ein echter Gewinn. Für den Anfang wird es aber auch ein einfaches Textfeld tun müssen, von da aus kann ich dann schauen wie weit ich komme.}

\subsubsection{\texttt{GROUP BY}}

Sobald in einer Abfrage eine \texttt{GROUP BY} Komponente auftaucht, hat dies Auswirkungen auf die Möglichkeiten innerhalb der \texttt{SELECT} Anweisung. Da keine Auswahl von ungruppierten und nicht-aggregierten Spalten möglich sein darf, müssen diese entfernt werden. Insbesondere wenn der Benutzer vorher schon mit komplizierten Ausdrücken im \texttt{SELECT} gearbeitet haben sollte, ist also zumindest eine Warnung nötig.

Danach verhält es sich mit den Aggregatsfunktionen nicht anders als mit den im vorigen Abschnitt beschriebenen Vergleichen und Ausdrücken, nur dass halt in diesem Kontext andere Funktionen zur Verfügung stehen.

\subsubsection{Verknüpfungen mit \texttt{JOIN}}

Sofern ein Editor für Ausdrücke (siehe \ref{sec:sql-compare-and-expression}) vorausgesetzt werden kann, lassen sich Verknüpfungen ohne größeren Mehraufwand fast vollumfänglich umsetzen. Im Falle von \texttt{JOIN}-Ausdrücken mit \texttt{USING} wird sich im Regelfall sogar exakt eine Spalte für die Verknüpfung anbieten. Bei Ausdrücken mit \texttt{ON} muss dann der Rückgriff auf den Editor für Ausdrücke erfolgen.

\subsubsection{Deaktivierbare Komponenten}

Unmittelbar nach dem Hinzufügen einer neuen Komponente sind möglicherweise noch nicht alle nötigen Informationen verfügbar. Daher werden Komponenten immer in einem deaktivierten Zustand hinzugefügt, sie lassen sich erst aktivieren wenn alle relevanten Informationen eingetragen worden sind. Genau so werden Komponenten mit inhaltlichen Fehlern deaktiviert, bis der Fehler behoben worden ist. Und natürlich kann es auch Gründe geben eine Komponente freiwillig zu deaktivieren, auch das soll selbstverständlich möglich sein.

Speziell für die \texttt{WHERE} und \texttt{HAVING} Komponenten ist zu klären, welche Auswirkung eine Deaktivierung haben soll. Sollen alle Kinder ebenfalls deaktiviert werden oder wird die nächste folgende Komponente zum \texttt{WHERE} bzw. \texttt{HAVING}? Auch hier sind beide Möglichkeiten denkbar, die bessere Alternative muss sich durch Tests mit realen Benutzern zeigen.

\subsubsection{Änderung der Reihenfolge}

Wiederum für die \texttt{WHERE} und \texttt{HAVING}, aber auch für die \texttt{JOIN} Komponenten wäre es sehr hilfreich, die Reihenfolge durch eine Schaltfläche oder auch Drag \& Drop variieren zu können.

\subsubsection{Unterabfragen}

Noch komplizierter wird der Umgang mit Ausdrücken, wenn man in diesen auch Unterabfragen zulassen möchte.

\unsure[inline]{Hier das gleiche Problem wie mit Ausdrücke, ich tendiere momentan zu ``mach ich nicht, sehe ich wenn überhaupt konzeptuell vor''.}

\subsubsection{Abfragen mit Parametern}
\label{sec:design-query-params}

Um eine Interaktion mit Endbenutzern zu ermöglichen können in Abfragen auch benannte Parameter anstelle von konstanten Werten verwendet werden. Diese müssen dann zur Laufzeit vom Benutzer angegeben werden. Aus Sicht der SQL-Abfrage ist diese Ergänzung relativ trivial, die eigentliche Problematik liegt im Binden dieser Parameter zur Laufzeit. Das Kapitel \ref{sec:design-ui-editor} beschreibt den Editor für Benutzeroberflächen, durch welchen die Verknüpfung mit den Bedienelementen der Oberfläche vorgenommen wird.

\subsubsection{Manipulation von Daten}

Neben den bisher ausführlich besprochenen \texttt{SELECT}-Anweisungen müssen sich natürlich \texttt{INSERT}-, \texttt{UPDATE}- und \texttt{DELETE}-Anweisungen umsetzen lassen. Da diese ihrer Natur nach nicht idempotent sind verbietet sich natürlich eine wiederholte, automatische Ausführung während der Entwicklungszeit. Es ist zu erwarten, dass die meisten von den Schülern verfassten Abfragen dieser Art ausführlichen Gebrauch von den in Kapitel \ref{sec:design-query-params} beschriebenen Parametern machen.

\subsection{Konzept für Oberflächen}
\label{sec:design-ui-concept}

\warning{Diese Oberflächen werden nicht in reinem HTML verfasst, es kommen noch einige Erweiterungen in Form einer Templatingengine zum Einsatz. Diese technischen Details bleiben den Entwicklern allerdings verborgen, im Normalfall sehen sie nicht einmal den ``normalen'' HTML-Code. Ähnlich wie für die SQL-Abfragen wird also auch für die Oberflächen ein dezidierter Editor (Kapitel \ref{sec:design-ui-editor}) implementiert. Die Schüler sollten zu keinem Zeitpunkt selber Quelltexte tippen müssen, optional aber durchaus die Möglichkeit dazu haben.}

Damit sich mit der Schülerentwicklungsumgebung erstellte Projekte auch von normalen Endanwendern bedienen lassen können Bedarf es natürlich einer entsprechenden Benutzeroberfläche. Hier bietet sich aus Gründen der einfachen Weitergabe eine webbasierte Oberfläche an: So entfällt bei den Endanwendern jegliche Installation und gerade für datenzentrierte, verteilte Anwendungen ist eine einheitliche Sicht auf den Datenbestand selbstverständlich essentiell. Beides Umstände, die sich gut mit einem zentralen Datenbestand auf einem Server erreichen lassen.

Grundsätzlich unterschieden werden alle Bedienelemente anhand der Anzahl der zu erwartenden Zeilen und den nötigen Eingabeoptionen. Die Kapitel \ref{sec:design-ui-bind-output} \nameref{sec:design-ui-bind-output} und \ref{sec:design-ui-bind-input} \nameref{sec:design-ui-bind-input} erläutern diese Unterscheidung noch einmal ausführlich. Abbildung \ref{fig:ui-element-concept} verdeutlicht diese Kategorisierung aber schon einmal Anhand einiger Beispiele.

\begin{figure}[p]
  \centering \begin{tikzpicture}
  \tikzset{square matrix/.style={
      column sep=-\pgflinewidth, row sep=-\pgflinewidth,
      nodes={
        minimum height=#1,
        anchor=center,
        text width=#1,
        align=center,
        inner sep=6pt
      },
    },
    square matrix/.default=3.50cm
  }


  \matrix[square matrix] (my matrix) at (0,0)
  {
    \node (single)   {Unmittelbare Ausgabe, z.B. in einem Text}; &
    \node (passive)  {Einfaches Eingabelement, z.B. eine Textbox}; \\
    \node (multiple) {Wiederholte Ausgabe, z.B. eine Tabelle}; &
    \node (input)    {Eingabelement mit Mehrfachauswahl}; \\
  };
  \draw [thick,-] (my matrix.east)  |- (my matrix.west);
  \draw [thick,-] (my matrix.south) |- (my matrix.north);

  \node [left=of single, anchor=north, rotate=90] {\textsc{Eine Zeile}};
  \node [left=of multiple, anchor=north, rotate=90] {\textsc{Beliebig}};
  \node [above=of single, anchor=north] {\textsc{Keine Eingabe}};
  \node [above=of passive, anchor=north] {\textsc{Eingabe}};
\end{tikzpicture}



%%% Local Variables:
%%% mode: latex
%%% TeX-master: "../thesis"
%%% End:

  \caption{Einordnung von Bedienelementen}
  \label{fig:ui-element-concept}
\end{figure}

\subsubsection{Layout}

Eine immer wiederkehrende Frage bei der Entwicklung von Webseiten ist die nach dem Layout der gesamten Seite. Technisch gesehen erfolgt die Umsetzung des Layouts heutzutage über eine Vielzahl unterschiedlicher CSS-Direktiven, deren Erstellung keinesfalls ein Kernbereich der Entwicklungsumgebung ist. Allerdings lassen sich auf den meisten Seiten immer wiederkehrende Strukturen feststellen: Die Unterteilung der Seite in ein Raster aus Zeilen und Spalten. Abbildung \ref{fig:grid-example} zeigt ein solches Layout. Dabei werden die beiden obersten Zeilen für den Header und die Pfadangabe nicht weiter in Spalten unterteilt. Die dritte Zeile enthält drei Spalten (Navigation, Inhalt und eine Leiste für Module), wobei diese letzte Modulspalte wiederum mehrere Zeilen enthält.

\begin{figure}[p]
  \includegraphics[width=\textwidth]{sketches/grid-example}
  \caption{Seite mit einem Raster aus Zeilen und Spalten}
  \label{fig:grid-example}
\end{figure}

Diese Layoutstruktur wird mittlerweile von einer Vielzahl von vorhandenen CSS-Bibliotheken unterstützt und soll auch das grundsätzliche Gedankenmodell für die hier beschriebenen Oberflächen sein. Es ergeben sich daraus die folgenden Layoutelemente:

\begin{description}
\item[Zeilen] \hfill \\
  Zeilen enthalten eine oder mehrere Spalten, wobei jeder Inhalt zwingend in einer Spalte platziert werden muss. Die Höhe einer Zeile entspricht der Höhe der höchsten Spalte.
\item[Spalten] \hfill \\
  Die Breite einer Spalte wird in einem relativen Maß mit Bezug zu einer bestimmten Bildschirmgröße angegeben. So können für breite Bildschirme z.B. drei Spalten vorgesehen werden, für sehr schmale Geräte wie Smartphones hingegen nur eine Spalte. Sollte die Summe dieser Breitenangaben der Spalten die maximale Breite einer Zeile überschreiten, werden diese auf mehrere Zeilen umgebrochen.
\end{description}

\missing[inline]{Visuelles Beispiel für den Umbruch}

\subsubsection{Allgemeine Bedienelemente}

Zwar sollen im Regelfall keine komplett statischen Seiten angezeigt werden, trotzdem muss es natürlich eine Möglichkeit geben statische Texte ohne Bindung an irgendwelche Abfragen zur Anzeige zu bringen. Die hier vorgestellten Bedienelemente funktionieren auch ohne jede Datenquelle, die folgenden Kapitel stellen aber noch Erweiterungen zur Datenanbindung vor. Ferner ersetzen diese kurzen Beschreibungen keinesfalls die technische Dokumentation, diese findet sich im Anhang.

\begin{description}
\item[Überschriften] \hfill \\
  HTML sieht die Verwendung von Überschriften in sechs Hierarchiebenen vor und stellt dafür distinkte Elemente zur Verfügung (\texttt{<h1>} bis \texttt{<h6>}). Dieser Umstand soll nicht unmittelbar abgebildet werden. Stattdessen gibt es ein allgemeines Bedienelement ``Überschrift'', zu dem sich dann eine Ebene auswählen lässt.
\item[Absätze] \hfill \\
  Blöcke von zusammenhängendem Text werden als Absatz ausgezeichnet, das entsprechende HTML-Äquivalent ist das \texttt{<p>}-Element.
\item[Listen] \hfill \\
  Strukturiert zusammenhängende Daten werden als sortierte oder unsortierte Liste (\texttt{<ol>} bzw. \texttt{<ul>} in HTML) gruppiert. Wie schon bei den Überschriften soll an dieser Stelle nur ein logisches Listenelement zum Einsatz kommen, dass dafür über eine Eigenschaft ``Nummerierung'' verfügt.      
\item[Bilder] \hfill \\
  Zwar technisch betrachtet ein eher simples Unterfangen, für Schüler erfahrungsgemäß aber motivierend, ist die Einbindung von Bildern. Diese entsprechen einem HTML \texttt{<img>}-Element, für welches das \texttt{src}-Attribut gesetzt werden muss.
\end{description}

Spätestens durch die Aufnahme von Bildern ist auch für statische Webseiten eindeutig, dass es möglich sein muss diese Bedienelemente über die Angabe von Parametern zu beinflussen. In den meisten Fällen werden diese Parameter zur Laufzeit durch die Ausführung einer Abfrage mit konkreten Werten bestückt, die Angabe von konstanten Werten ist aber ebenfalls möglich.

\subsubsection{Bindung von Daten an die Oberfläche}
\label{sec:design-ui-bind-output}

Ein Großteil der Oberflächenelemente wird also einige Eigenschaften aufweisen, deren Inhalte erst zur Laufzeit dynamisch gebunden werden. Dabei stellt sich die Frage, wie die Struktur der zur Verfügung stehenden Daten eigentlich definiert wird bzw. aussehen soll?

Auf abstrakter Ebene ist die Beantwortung dieser Frage eindeutig: Da das Ergebnis einer SQL-Abfrage grundsätzlich eine Tabellenstruktur mit Zeilen, Spalten und Zellen ist, wird dieses Datenmodell auch für die Oberfläche zugrunde gelegt. Die an die Bedienelemente zu bindenden Daten sind also grundsätzlich tabellarischer Natur.

Um im Falle von mehreren in Frage kommenden Abfragen eine eindeutige Zuordnung vornehmen zu können, müssen Abfragen, deren Ergebnis an die Oberfläche gebunden werden soll, mit einem eindeutigen Namen bezeichnet werden. Dieser Name kann von den Entwicklern frei gewählt werden, sollte aber, analog zu einem Funktionsbezeichner, sinnvoll darüber Auskunft geben was eine Abfrage bewirkt. Der Umgang mit den Zeilen erfordert keine Sonderbehandlung, hier genügt die Angabe eines Zeilenindex. Für die Spalten entfällt glücklicherweise eine ``künstliche'' Benennung, diese werden trivialerweise über ihren Namen entsprechend der \texttt{SELECT}-Komponente angesprochen. Mit diesen drei Werten (Name der Abfrage, Zeilenindex, Spaltenname) lässt sich jede Zelle einer Menge von Abfragen eindeutig identifizieren.

Aus praktischen Gründen sollten Abfragen allerdings bezüglich der Anzahl der erwarteten Zeilen annotiert werden können, konkret unterschieden werden muss dabei zwischen ``exakt eine Zeile'' und ``beliebig viele (auch keine) Zeilen''. Im Falle einer Abfrage mit nur einer Zeile entfällt dann Notwendigkeit einen Zeilenindex anzugeben, dieser bezieht sich trivalerweise immer auf die einzige, zur Verfügung stehende Zeile. Wenn Daten an die Oberflächen gebunden werden entfällt durch diese Abkürzung die permanente Angabe einer redundanten Information.

Listing \ref{lst:sql:person-days} zeigt eine Abfrage die in Listing \ref{lst:html:person-days-index} und \ref{lst:html:person-days} einmal mit und einmal ohne Index gebunden wird. Das ist zwar ein kleiner Vorgriff, demonstriert aber wie diese Annotation der Lesbarkeit zu gute kommt.

\begin{lstlisting}[language=SQL, caption=Abfrage mit garantiert einer Ergebniszeile,label=lst:sql:person-days]
SELECT name, geburtstag
FROM   person
LIMIT  1;
\end{lstlisting}

\begin{lstlisting}[language=HTML, caption=String-Interpolation mit Indexzugriff, label=lst:html:person-days-index]
<!-- benutzer = Listing (*@\ref{lst:sql:person-days}: \nameref{lst:sql:person-days}@*) -->
<p>Hallo {{ benutzer[0].name }}, du bist am {{ benutzer[0].geburtstag }} geboren.</p>
\end{lstlisting}

\begin{lstlisting}[language=HTML, caption=String-Interpolation mit implizitem Index, label=lst:html:person-days]
<!-- benutzer = Listing (*@\ref{lst:sql:person-days}: \nameref{lst:sql:person-days}@*) -->
<p>Hallo {{ benutzer.name }}, du bist am {{ benutzer.geburtstag }} geboren.</p>
\end{lstlisting}

Diese Annotation muss jedoch vom Entwickler manuell gesetzt werden, eine automatische Berechnung wäre für beliebige Abfragen unmöglich\footnote{Viele SQL-Dialekte sind Turing-Vollständig und schon das Halteproblem ist nicht entscheidbar.} und auch für speziell strukturierte Abfragen ist diese Unterscheidung zumindest nicht trivial. Oder um es anders auszudrücken: Die Erweiterung der Entwicklungsumgebung um diese automatische Vorhersage wäre Stoff für eine weitere wissenschaftliche Arbeit, keinesfalls aber Gegenstand dieser Arbeit. Praktischerweise lässt sich eine auftretende Verletzung dieser Annotation aber sehr gut erkennen und daher auch sinnvoll an den Entwickler kommunizieren. Und letzten Endes könnte ein Entwickler die Garantie durch die Einführung einer \texttt{LIMIT 1}-Komponente selber geben, was möglicherweise aber zu überraschendem Verhalten zur Laufzeit führt.

Diese gesonderte Behandlung von bestimmten Tabellendimensionen ist übrigens keine Erfindung dieser Thesis: Auch SQL selbst hebt einige Abfragen mit speziellen Strukturen hervor! Tabellen mit nur einer Zelle können als skalarer Wert eingesetzt werden, bei nur einer Spalte können Abfragen z.B. für \texttt{IN}-Ausdrücke verwendet werden.

Der Umgang mit den Zeilen einer Abfrage ist im Rahmen der Oberfläche ein bisschen komplizierter, da hier technisch gesehen mindestens zwei verschiedene Bedienelemente zum Einsatz kommen müssen: Ein Containerelement und ein sich wiederholendes Element. Auch an dieser Stelle tätigen wir daher mit Listing \ref{lst:sql:people-days} und \ref{lst:html:people-days} einen kurzen Vorgriff, um den grundsätzlichen Unterbau zu demonstrieren. An dieser Stelle sei noch einmal angemerkt, dass diese Listings nur die \textit{Struktur} der Oberfläche wiedergeben sollen, die Schüler werden diesen HTML-Code im Normalfall nicht zu Gesicht bekommen.

\begin{lstlisting}[language=SQL, caption=Abfrage mit beliebig vielen Ergebniszeilen,label=lst:sql:people-days]
SELECT name, geburtstag
FROM   person;
\end{lstlisting}

\begin{lstlisting}[language=HTML, caption=Containerelemente mit Kindern, label=lst:html:people-days]
<!-- alleBenutzer = Listing (*@\ref{lst:sql:people-days}: \nameref{lst:sql:people-days}@*) -->
<ul>
  <template ngFor #benutzer [ngForOf]="alleBenutzer">
    <li>
      {{ benutzer.name }} wurde am {{ benutzer.geburtstag }} geboren.
    </li>
  </template>
</ul>
\end{lstlisting}

Es fällt auf, dass die Angabe des Containerelementes \texttt{<ul>} ohne besondere Auszeichnung auskommt. Die exakte Syntax der Wiederholung, also das \texttt{<template>} Element ist die spezifische Syntax von Angular 2 und weder für Entwickler noch Endanwender von Belang. Der Inhalt des sich wiederholenden Elements ist wiederum trivial und entspricht der Bindung an eine Abfrage mit exakt einer Ergebniszeile. Eine mögliche Alternative zu dieser Vorgehensweise wäre wiederum das hochzählen eines Index, statt dem impliziten erneuten Binden der \texttt{benutzer} Variable in jedem Durchlauf.

\unsure[inline]{Es gibt 1001-HTML-Templating Sprachen da draußen, von denen Angular 2 nicht gerade die einsteigerfreundlichste Syntax hat. Allerdings verwende ich für den Editor Angular 2, die Engine ist also ``schon da''.  Ich muss nochmal schauen ob ich das ändern mag, dann wäre für die ``normale'' Endbenutzeranzeige der Projekte aber noch ein weiteres Programm zu schreiben.}

\subsubsection{Eingabe von Daten über die Oberfläche}
\label{sec:design-ui-bind-input}

Immer wenn eine Abfrage mit Platzhaltern an eine Seite gebunden wird, ist es notwendig dem Benutzer eine Möglichkeit zu geben diese Platzhalter zu füllen. Im einfachsten Fall kann diese Bindung über einfache \texttt{<input type=text>}-Elemente hergestellt werden. Das ist für viele Arten von Such- und Eingabemasken auch sicherlich ausreichend, führt jedoch zu Problemen bei Beziehungen zwischen Entitäten. In diesem Fall müsste in das Textfeld vom Benutzer ein Primärschlüssel angegeben werden, was ganz sicher kein besonders Benutzerfreundliches Konzept darstellt.

In einem solchen Fall soll es also möglich sein auch Bedienelememente wie Comboboxen zu verwenden, die ihrerseits wieder eine Abfrage zur Anzeige der zur Verfügung stehenden Daten benötigen. Dabei muss klar zwischen zwei benötigten Informationen unterschieden werden. Auf der einen Seite muss das Resultat eines solchen Bedienelements vermutlich einen Primärschlüssel zurückliefern. Auf der anderen Seite erwarten die Benutzer aber die Anzeige von ``intuitiven'' Informationen, anhand derer Sie ihre Auswahl treffen können. Der Primärschlüssel ist typischerweise eine künstliche Spalte, die für den Endanwender keinerlei Informationsgehalt hat. Praktischerweise ist auch diese Unterscheidung in HTML vorgesehen

\begin{lstlisting}[language=HTML, caption=Containerelemente mit Kindern, label=lst:html:select-example]
<!-- alleBenutzer = Listing (*@\ref{lst:sql:people-days}: \nameref{lst:sql:people-days}@*) -->
<select>
  <template ngFor #benutzer [ngForOf]="alleBenutzer">
    <option [value]="benutzer.rowid">
      {{ benutzer.rowid }}: {{ benutzer.name }}
    </option>
  </template>
</select>
\end{lstlisting}

\subsubsection{Seitenstruktur und Navigation}

Zunächst ist festzuhalten, dass die serverseitige Erzeugung von Seiten zur Laufzeit nicht vorgesehen ist. Stattdessen definiert der Entwickler eine Reihe von serverseitig statischen HTML-Seiten samt einiger Datenquellen, die dann vom Client des Anwenders ausgeführt werden. Der Client analysiert dann zunächst die in der Seite hinterlegten Datenquellen und führt ggfs. weitere HTTP-Anfragen aus, um alle Daten einzusammenln.

Die Seiten können beliebig in einer Ordnerstruktur organisiert werden, die Dateipfade bilden nicht die Grundlage für das Routing. Diese Entkoppelung der Pfade im Dateisystem von den URLs erlaubt es den Schülern die Übergabe von Parametern und auch die sonstige Navigationsstruktur frei zu gestalten.

Zu jeder dieser Seiten können eigene Datenquellen hinterlegt werden. Dabei sind bisher die folgenden Möglichkeiten vorgesehen:

\begin{description}
  \item[Parameter aus URLs] \hfill \\
    Zur Übergabe von Informationen zwischen Seiten können Teile der URL als Parameter für die Seite markiert werden.
  \item[Abfragen] \hfill \\
    Die ``typische'' Datenquelle für Seiten.
  \item[Variablen \& Konstanten] \hfill \\
    Manchmal kann es nötig sein, verschiedene Konstrukte ohne eine Datenbank im Hintergrund zu erproben. In diesem Fall bietet sich die Verwendung von vordefinierten Werten an.
\end{description}

Damit für jede Seite eindeutig ist, über welche kanonischen URLs sie angesprochen werden kann muss der Entwickler eine Routing-Konfiguration hinterlegen. In dieser wird einem einfachen Pfad, ggfs. auch mit Platzhaltern für Parameter, ein eindeutiger Name sowie eine anzuzeigende HTML-Seite zugewiesen. Der Name ist nötig um, unabhängig von einer sich möglicherweise ändernden Pfadangabe, ein eindeutiges Ziel für Links angeben zu können. Nur wenn diese Eindeutigkeit hergestellt ist, kann der Entwickler auf z.B. nicht existierende Linkziele oder fehlende Parameter hingewiesen werden.

Für einen einfachen Blog wäre z.B. die folgende interne Repräsentation der Konfiguration denkbar:

\begin{lstlisting}[language=JavaScript, caption=Einfache Routen für einen Blog, label=lst:json:simple-routing-blog]
{ path : "/",            name : "Hauptseite", page : "index.html",
  path : "/beitrag/:id", name : "Beitrag",    page : "beitrag.html" }
\end{lstlisting}

\subsection{Editor für Oberflächen}
\label{sec:design-ui-editor}

Der Editor für Oberflächen nimmt die Konzepte aus dem vorigen Kapitel und macht sie den Entwicklern auf eine intuitive Art zugänglich. Abbildung \ref{fig:ui-editor-sketch-first} zeigt den grundsätzlichen Aufbau der Oberfläche aus einem zentralen Designbereich und einer Seitenleiste. Die Verbindungen zwischen den Bedienelementen und den Daten aus der Seitenleiste verdeutlichen zu Illustrationszwecken deren Zusammenhang. In der endgültigen Fassung des Editors soll der Zusammenhang nicht durch Linien dargestellt werden, sondern durch einen Hervorhebungseffekt wenn einer der beiden Seiten aktuell ausgewählt ist.

\unsure[inline]{Eigentlich würde ich gerne auf Tabs in der Seitenleiste verzichten, die Daten könnte man ja auch innerhalb des Designbereiches darstellen. Diesen Weg der Platzierung von nicht sichtbaren Elementen hat auch schon die Delphi IDE betreten und so dramatisch schlimm fand ich das nicht.}

\begin{figure}[h]
  \includegraphics[width=\textwidth]{sketches/ui-sketch-first}
  \caption{Skizze des Oberflächeneditors mit Tabs}
  \label{fig:ui-editor-sketch-first}
\end{figure}

Wenn der Benutzer auf den Tab für die Bedienelemente wechselt, bekommt er eine Übersicht aller verfügbaren Bedienelemente. Diese können dann per Drag \& Drop auf dem Designbereich platziert werden. Einige Bedienelemente verfügen über komplexere Eigenschaften, diese werden dann separat dargestellt.

\unsure[inline]{Ich bin noch absolut unentschlossen, wie bzw. wo ich die weiteren Eigenschaften von komplexen Eigenschaften unterbringen soll. Die ``klassische'' Variante aller aktuellen Entwicklungsumgebungen ist eine weitere Seitenleiste, die hätte dann in diesem Fall dann allerdings drei Tabs. Daher überlege ich alternativ möglichst viele Informationen unmittelbar im Designbereich unterzubringen. Wenn man dann ein Bedienelement anklickt klappt sich darunter (oder sonstwo) ein Bereich auf, in dem die weiteren Einstellungen vorgenommen werden können.}

\subsection{Beispielhafte Datenbasen}
\label{sec:example-queries}

%%% Local Variables:
%%% mode: latex
%%% TeX-master: "thesis"
%%% End:
